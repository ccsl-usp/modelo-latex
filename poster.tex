% Authors: Nelson Lago, Arthur Del Esposte and Eduardo Zambom Santana
% Portions of the example contents: Arthur Del Esposte
% This file is distributed under the MIT Licence

%%%%%%%%%%%%%%%%%%%%%%%%%%%%%%%%%%%%%%%%%%%%%%%%%%%%%%%%%%%%%%%%%%%%%%%%%%%%%%%%
%%%%%%%%%%%%%%%%%%%%%%%%%%%%%%%%% PREÂMBULO %%%%%%%%%%%%%%%%%%%%%%%%%%%%%%%%%%%%
%%%%%%%%%%%%%%%%%%%%%%%%%%%%%%%%%%%%%%%%%%%%%%%%%%%%%%%%%%%%%%%%%%%%%%%%%%%%%%%%

% A língua padrão é a última citada
\documentclass[brazilian,english]{article}

% Vários pacotes e opções de configuração genéricos
\usepackage{imegoodies}
\usepackage[poster]{imelooks}
\usepackage[poster,skins]{tcolorbox}
\usepackage{evenragged}
\evenRaggedRight

\hidelinks % desabilita cor/sublinhado dos links (URLs, refs etc.)

% Diretórios onde estão as figuras; com isso, não é necessário (mas
% é permitido) colocar o caminho completo em \includegraphics. Note
% que a extensão nunca é necessária (mas é permitida), ou seja, o
% resultado é o mesmo com "\includegraphics{figuras/foto.jpeg}",
% "\includegraphics{foto.jpeg}", "\includegraphics{figuras/foto}"
% ou "\includegraphics{foto}".
\graphicspath{{figuras/},{fig/},{logos/},{img/},{images/},{imagens/}}

% Comandos rápidos para mudar de língua:
% \en -> muda para o inglês
% \br -> muda para o português
% \texten{blah} -> o texto "blah" é em inglês
% \textbr{blah} -> o texto "blah" é em português
\babeltags{br = brazilian, en = english}

% Para formatar código-fonte (ex. em Java). listings funciona bem mas
% tem algumas limitações (https://tex.stackexchange.com/a/153915 ).
% Se isso for um problema, a package minted pode oferecer resultados
% (muito) melhores; a desvantagem é que ela depende de um programa
% externo, o pygments (escrito em python).
%
% listings também não tem suporte específico a pseudo-código, mas
% incluímos uma configuração para isso que deve ser suficiente.
% Caso contrário, há diversas packages específicas para a criação
% de pseudocódigo:
%
%  * a mais comum é algorithmicx ("\usepackage{algpseudocode}");
%
%  * algorithm2e é bastante flexível, mas um tanto complexa;
%
%  * clrscode3e foi usada no livro "Introduction to Algorithms",
%    de Cormen, Leiserson, Rivert e Stein;
%
%  * pseudocode foi usada no livro "Combinatorial Algorithms",
%    de Kreher e Stinson;
%
%  * algpseudocodex é uma package relativamente nova similar
%    a algorithmicx/algpseudocode mas com diversas melhorias;
%
%  * pseudo também é relativamente nova; ela funciona de forma
%    um pouco diferente das demais e é bastante customizável.
%
% A diferença entre essas packages e listings/minted é que estas
% últimas "entendem" o código e aplicam a formatação automaticamente,
% enquanto com as packages acima o usuário precisa usar comandos LaTeX
% para definir a formatação.
%
% algorithmicx/algpseudocode, algorithm2e, clrscode3e, pseudocode
% e algpseudocodex usam uma "linguagem" própria baseada em comandos
% LaTeX que pode ser facilmente modificada pelo usuário (ou seja,
% é fácil fazer pseudocódigo em português). Segue um exemplo com
% algpseudocodex, provavelmente a opção mais interessante dentre
% este grupo (note o comando "\While", que imprime automaticamente
% a palavra-chave "while" e ajusta a indentação):
%
% \begin{algorithmic}[1]
%   \Function{Euclid}{$a, b$} \Comment{The g.c.d. of a and b}
%     \State $r\gets a\bmod b$
%     \While{$r\not=0$} \Comment{We have the answer if r is 0}
%       \State $a\gets b$
%       \State $b\gets r$
%       \State $r\gets a\bmod b$
%     \EndWhile
%     \State \textbf{return} $b$ \Comment{The gcd is b}
%   \EndFunction
% \end{algorithmic}
%
% pseudo não usa uma "linguagem" própria desse tipo; ao invés disso,
% ela oferece comandos para a formatação direta de palavras-chave,
% variáveis, indentação etc. Um exemplo ("\\", "\\+" e "\\-" controlam
% as quebras de linha e a indentação):
%
% \begin{pseudo}
%    \kw{Function} \fn{Euclid}(a, b) \ct{The g.c.d. of a and b} \\+
%      $r\gets a\bmod b$ \\
%      \kw{while} $r\not=0$ \ct{We have the answer if r is 0} \\+
%        $a\gets b$ \\
%        $b\gets r$ \\
%        $r\gets a\bmod b$ \\-
%      \kw{end} \\
%      \kw{return} $b$ \ct{The gcd is b} \\-
%    \kw{end}
% \end{pseudo}

\usepackage{listings}
\usepackage{lstautogobble}
% Carrega a "linguagem" pseudocode para listings
\appto{\lstaspectfiles}{,lstpseudocode.sty}
\appto{\lstlanguagefiles}{,lstpseudocode.sty}
% Estes dois são carregados do diretório extras (veja basics.tex)
\lstloadaspects{simulatex,invisibledelims,pseudocode}
\lstloadlanguages{[base]pseudocode,[english]pseudocode,[brazilian]pseudocode}

% O pacote listings não lida bem com acentos! No caso dos caracteres acentuados
% usados em português é possível contornar o problema com a tabela abaixo.
% From https://en.wikibooks.org/wiki/LaTeX/Source_Code_Listings#Encoding_issue
\lstset{literate=
  {á}{{\'a}}1 {é}{{\'e}}1 {í}{{\'i}}1 {ó}{{\'o}}1 {ú}{{\'u}}1
  {Á}{{\'A}}1 {É}{{\'E}}1 {Í}{{\'I}}1 {Ó}{{\'O}}1 {Ú}{{\'U}}1
  {à}{{\`a}}1 {è}{{\`e}}1 {ì}{{\`i}}1 {ò}{{\`o}}1 {ù}{{\`u}}1
  {À}{{\`A}}1 {È}{{\'E}}1 {Ì}{{\`I}}1 {Ò}{{\`O}}1 {Ù}{{\`U}}1
  {ä}{{\"a}}1 {ë}{{\"e}}1 {ï}{{\"i}}1 {ö}{{\"o}}1 {ü}{{\"u}}1
  {Ä}{{\"A}}1 {Ë}{{\"E}}1 {Ï}{{\"I}}1 {Ö}{{\"O}}1 {Ü}{{\"U}}1
  {â}{{\^a}}1 {ê}{{\^e}}1 {î}{{\^i}}1 {ô}{{\^o}}1 {û}{{\^u}}1
  {Â}{{\^A}}1 {Ê}{{\^E}}1 {Î}{{\^I}}1 {Ô}{{\^O}}1 {Û}{{\^U}}1
  {Ã}{{\~A}}1 {ã}{{\~a}}1 {Õ}{{\~O}}1 {õ}{{\~o}}1
  {œ}{{\oe}}1 {Œ}{{\OE}}1 {æ}{{\ae}}1 {Æ}{{\AE}}1 {ß}{{\ss}}1
  {ű}{{\H{u}}}1 {Ű}{{\H{U}}}1 {ő}{{\H{o}}}1 {Ő}{{\H{O}}}1
  {ç}{{\c c}}1 {Ç}{{\c C}}1 {ø}{{\o}}1 {å}{{\r a}}1 {Å}{{\r A}}1
  {€}{{\euro}}1 {£}{{\pounds}}1 {«}{{\guillemotleft}}1
  {»}{{\guillemotright}}1 {ñ}{{\~n}}1 {Ñ}{{\~N}}1 {¿}{{?`}}1
}

% Opções default para o pacote listings
% Ref: http://en.wikibooks.org/wiki/LaTeX/Packages/Listings
\lstset{
  columns=[l]fullflexible,            % do not try to align text with proportional fonts
  basicstyle=\footnotesize\ttfamily,  % the font that is used for the code
  numbers=left,                       % where to put the line-numbers
  numberstyle=\footnotesize\ttfamily, % the font that is used for the line-numbers
  stepnumber=1,                       % the step between two line-numbers. If it's 1 each line will be numbered
  numbersep=20pt,                     % how far the line-numbers are from the code
  autogobble,                         % ignore irrelevant indentation
  commentstyle=\color{Brown}\upshape,
  stringstyle=\color{black},
  identifierstyle=\color{DarkBlue},
  keywordstyle=\color{cyan},
  showspaces=false,                   % show spaces adding particular underscores
  showstringspaces=false,             % underline spaces within strings
  showtabs=false,                     % show tabs within strings adding particular underscores
  %frame=single,                       % adds a frame around the code
  framerule=0.6pt,
  tabsize=2,                          % sets default tabsize to 2 spaces
  captionpos=b,                       % sets the caption-position to bottom
  breaklines=true,                    % sets automatic line breaking
  breakatwhitespace=false,            % sets if automatic breaks should only happen at whitespace
  escapeinside={\%*}{*)},             % if you want to add a comment within your code
  backgroundcolor=\color[rgb]{1.0,1.0,1.0}, % choose the background color.
  rulecolor=\color{darkgray},
  extendedchars=true,
  inputencoding=utf8,
  xleftmargin=30pt,
  xrightmargin=10pt,
  framexleftmargin=25pt,
  framexrightmargin=5pt,
  framesep=5pt,
}

% Um exemplo de estilo personalizado para listings (tabulações maiores)
\lstdefinestyle{wider} {
  tabsize = 4,
  numbersep=15pt,
  xleftmargin=25pt,
  framexleftmargin=20pt,
}

% Outro exemplo de estilo personalizado para listings (sem cores)
\lstdefinestyle{nocolor} {
  commentstyle=\color{darkgray}\upshape,
  stringstyle=\color{black},
  identifierstyle=\color{black},
  keywordstyle=\color{black}\bfseries,
}

% Um exemplo de definição de linguagem para listings (XML)
\lstdefinelanguage{XML}{
  morecomment=[s]{<!--}{-->},
  morecomment=[s]{<!-- }{ -->},
  morecomment=[n]{<!--}{-->},
  morecomment=[n]{<!-- }{ -->},
  morestring=[b]",
  morestring=[s]{>}{<},
  morecomment=[s]{<?}{?>},
  morekeywords={xmlns,version,type}% list your attributes here
}

% Estilo padrão para a "linguagem" pseudocode
\lstdefinestyle{pseudocode}{
  basicstyle=\rmfamily\small,
  commentstyle=\itshape,
  keywordstyle=\bfseries,
  identifierstyle=\itshape,
  % as palavras "function" e "procedure"
  procnamekeystyle=\bfseries\scshape,
  % funções precedidas por function/procedure ou com \func{}
  procnamestyle=\ttfamily,
  specialidentifierstyle=\ttfamily\bfseries,
}
\lstset{defaultdialect=[english]{pseudocode}}

% A package listings tem seu próprio mecanismo para a criação de
% captions, lista de programas etc. Neste modelo não usamos esses
% recursos (veja mais abaixo), mas utilizamos estes nomes:
\addto\extrasbrazil{%
  \gdef\lstlistlistingname{Lista de programas}%
  \gdef\lstlistingname{Programa}%
}
\addto\extrasbrazilian{%
  \gdef\lstlistlistingname{Lista de programas}%
  \gdef\lstlistingname{Programa}%
}
\addto\extrasenglish{%
  \gdef\lstlistlistingname{List of Programs}%
  \gdef\lstlistingname{Program}%
}

% Novo tipo de float para programas, possível graças à package float
% ou floatrow.
% Observe que a lista de floats de cada tipo é criada automaticamente
% pela package float/floatrow, mas precisamos:
%  1. Definir o nome do comando ("\begin{program}")
%  2. Definir o nome do float em cada língua ("Figura X", "Programa X")
%  3. Definir a extensão do arquivo temporário a ser usada. Pode ser
%     qualquer coisa, desde que não haja repetições. Aqui, usamos "lop";
%     lembre-se que LaTeX já usa várias outras, como "lof", "lot" etc.,
%     então seja cuidadoso na escolha!
%  4. Acrescentar os comandos correspondentes em paginas-preliminares.tex

\makeatletter
\@ifpackageloaded{floatrow}
  {
    \ifcsundef{chapter}
        % O novo ambiente se chama "program" ("\begin{program}") e a extensão
        % temporária é "lop"
        {\DeclareNewFloatType{program}{placement=htbp,fileext=lop}}
        {\DeclareNewFloatType{program}{placement=htbp,fileext=lop,within=chapter}}

    % Ajusta ligeiramente o espaçamento do estilo "ruled".
    \DeclareFloatVCode{customrule}{{\kern 0pt\hrule\kern 2.5pt\relax}}
    \floatsetup[program]{style=ruled,precode=customrule}
  }
  {
    % Não temos a package floatrow; vamos assumir que temos a package float.

    % O estilo padrão do novo float a ser criado (veja mais sobre isso na
    % documentação da package float). Para "program" usamos "ruled", mas
    % para outros floats provavelmente é melhor usar o mesmo formato de
    % Figuras e Tables (plain).
    \floatstyle{ruled}

    \ifcsundef{chapter}
        % O novo ambiente se chama "program" ("\begin{program}") e a extensão
        % temporária é "lop"
        {\newfloat{program}{htbp}{lop}}
        {\newfloat{program}{htbp}{lop}[chapter]}

    % Retorna o estilo dos floats para o padrão
    \floatstyle{plain}
  }
\makeatother

\captionsetup*[program]{style=ruled,position=top}

% "Program X / Programa X" e "Lista de programas / List of Programs"
\floatname{program}{\lstlistingname}
\gdef\programlistname{\lstlistlistingname}

% Se um programa é maior que uma página, ele não pode ser inserido em
% um float. Nesse caso, vamos criar o ambiente "programruledcaption",
% que cria a mesma estrutura visual e os mesmos captions que os floats
% do tipo "program", mas sem ser um float. Para isso, vamos usar recursos
% da package framed (a package tcolorbox poderia ter sido usada também).
%
% Observe que "programruledcaption" funciona *apenas* para os floats do
% tipo "program". Se quiser criar algo similar para outro tipo de float,
% você vai precisar criar um novo comando ("myfloatruledcaption")
% copiando os comandos abaixo e modificando-os conforme necessário.
\newsavebox{\programCaptionTextBox}
\usepackage{framed}
\newenvironment{programruledcaption}[2][]{
  % All spacing measurements were adjusted to visually reproduce
  % the float captions
  \setlength\fboxsep{0pt}

  % topsep means space before AND after
  \setlength\topsep{.28\baselineskip plus .3\baselineskip minus 0pt}

  \vspace{.3\baselineskip} % Some extra top space

  % For whatever reason, the framed package actually calls "\captionof"
  % multiple times, messing up the counter. We need to prevent this,
  % so we put the caption in a box once and reuse the box.

  \savebox{\programCaptionTextBox}{%
    \parbox[b]{\textwidth}{%
      \ifstrempty{#1}
        {\captionof{program}[#2]{#2}}%
        {\captionof{program}[#1]{#2}}%
    }
  }

  \def\fullcaption{
    \vspace*{-.325\baselineskip}
    \noindent\usebox{\programCaptionTextBox}%
    \vspace*{-.56\baselineskip}%
    \kern 2pt\hrule\kern 2pt\relax
  }

  \def\FrameCommand{
    \hspace{-.007\textwidth}%
    \CustomFBox
      {\fullcaption}
      {\vspace{.13\baselineskip}}
      {.8pt}{.4pt}{0pt}{0pt}
  }

  \def\FirstFrameCommand{
    \hspace{-.007\textwidth}%
    \CustomFBox
      {\fullcaption}
      {\hfill\textit{cont}\enspace$\longrightarrow$}
      {.8pt}{0pt}{0pt}{0pt}
  }

  \def\MidFrameCommand{
    \hspace{-.007\textwidth}%
    \CustomFBox
      {$\longrightarrow$\enspace\textit{cont}\par\vspace*{.3\baselineskip}}
      {\hfill\textit{cont}\enspace$\longrightarrow$}
      {0pt}{0pt}{0pt}{0pt}
  }

  \def\LastFrameCommand{
    \hspace{-.007\textwidth}%
    \CustomFBox
      {$\longrightarrow$\enspace\textit{cont}\par\vspace*{.3\baselineskip}}
      {\vspace{.13\baselineskip}}
      {0pt}{.4pt}{0pt}{0pt}
  }

  \MakeFramed{\FrameRestore}

}{
  \endMakeFramed
}





%%%%%%%%%%%%%%%%%%%%%%%%%%%%%%%%%%%%%%%%%%%%%%%%%%%%%%%%%%%%%%%%%%%%%%%%%%%%%%%%
%%%%%%%%%%%%%%%%%%%%%%%%%%%%%%%%%% METADADOS %%%%%%%%%%%%%%%%%%%%%%%%%%%%%%%%%%%
%%%%%%%%%%%%%%%%%%%%%%%%%%%%%%%%%%%%%%%%%%%%%%%%%%%%%%%%%%%%%%%%%%%%%%%%%%%%%%%%

% O arquivo com os dados bibliográficos para biblatex; você pode usar
% este comando mais de uma vez para acrescentar múltiplos arquivos
\addbibresource{bibliografia.bib}

% Este comando permite acrescentar itens à lista de referências sem incluir
% uma referência de fato no texto (pode ser usado em qualquer lugar do texto)
%\nocite{bronevetsky02,schmidt03:MSc, FSF:GNU-GPL, CORBA:spec, MenaChalco08}
% Com este comando, todos os itens do arquivo .bib são incluídos na lista
% de referências
%\nocite{*}

%\title[CCSL]{An example poster using \LaTeX{}\\by CCSL-IME/USP}
%
%\institute{Department of Computer Science --- University of São Paulo}
%
%\author[ccsl@ime.usp.br]{\textbf{ccsl.ime.usp.br}}
%
%\date{Month and day, year}
%
%

%%%%%%%%%%%%%%%%%%%%%%%%%%%%%%%%%%%%%%%%%%%%%%%%%%%%%%%%%%%%%%%%%%%%%%%%%%%%%%%%
%%%%%%%%%%%%%%%%%%%%%%%%%%%%%%% INÍCIO DO POSTER %%%%%%%%%%%%%%%%%%%%%%%%%%%%%%%
%%%%%%%%%%%%%%%%%%%%%%%%%%%%%%%%%%%%%%%%%%%%%%%%%%%%%%%%%%%%%%%%%%%%%%%%%%%%%%%%

\geometry{a1paper,left=1.5cm,right=1.5cm,top=0cm,bottom=0cm}
\renewcommand*\familydefault{\sfdefault}
\definecolor{InterSCity}{HTML}{00998C}
\definecolor{DarkInterSCity}{HTML}{0D6D65}
\definecolor{PastelGreen}{HTML}{73937E}

\definecolor{imeblue}{RGB}{20,45,105}
\definecolor{imeyellow}{RGB}{230,175,60}
\definecolor{imered}{RGB}{130,20,60}
\definecolor{imesoftblue}{RGB}{0,100,160}

\definecolor{DarkRed}{HTML}{872718}
\definecolor{Blackish}{HTML}{232830}
\definecolor{SandyPaper}{HTML}{FFFCE2}

\makeatletter
% Os triângulos ficam de tamanhos diferentes com pdflatex e lualatex;
% para contornar isso, vamos (1) redimensionar para o tamanho do
% \strut e (2) redimensionar para o tamanho desejado.
\newcommand\@IMEcharscale[2]{%
    \bgroup
    \scalebox{#1}{\resizebox*{!}{\totalheightof{\strut}}{#2}}%
    \egroup
}

\newcommand{\footimage}[1]{\def\footim@ge{#1}}
\def\insertfootimage{
    \ifdefined\footim@ge
    \tikz[overlay,remember picture]
    \node[anchor=south east,inner sep=12pt] at (current page.south east)
    {\footim@ge};
  \fi
}


\setlist{
}

\setlist[itemize,1]{
    left = 0pt .. .8em,
    label={\color{imeblue}\raisebox{.1em}{\@IMEcharscale{.26}{\ensuremath{\blacktriangleright}}}}
}
\setlist[itemize,2]{
    before*={\footnotesize},
    left = -5pt .. .8em,
    label={\color{imeblue}\raisebox{-.1em}{\@IMEcharscale{.4}{\ensuremath{\bullet}}}}
}
\setlist[itemize,3]{
    before*={\footnotesize\itshape},
    left = -1pt .. .95em,
    label={\color{imeblue}\raisebox{.05em}{\@IMEcharscale{.3}{\normalfont\textbf{\guillemotright}}}}
}
\makeatother

%% Optional foot image
\footimage{\includegraphics[width=13cm]{ccsl-logo}}

\setstretch{1.1}
\renewcommand\bibfont{\scriptsize}
\parindent=0pt

\begin{document}

% Em um poster não há \maketitle

\begin{tcbposter}[
  no coverage,
  fontsize = 32pt,
  poster = {
    %showframe,
    rows = 6,
    % Queremos ter algumas coisas divididas em 1, 2, 3 e 4 colunas;
    % o mínimo múltiplo comum é 12.
    columns = 12,
    colspacing = 1.2cm,
    rowspacing = .8cm,
  },
  boxes = {
    colframe = imesoftblue,
    colback = SandyPaper,
    fonttitle = \large,
    top=.3\baselineskip,
    bottom=.3\baselineskip,
    toptitle=.1\baselineskip,
    bottomtitle=.1\baselineskip,
    left=.4\baselineskip,
    right=.6\baselineskip,
    valign=center,
    skin=enhanced,
  }
]

\tcbsubskin{titlebox}{enhanced}{colback=imeblue, coltext=white, borderline south={15pt}{0pt}{imeyellow}, boxsep=10mm, grow sidewards by = 2.3cm}
\tcbsubskin{footer}{enhanced}{colback=imesoftblue, coltext=white, boxsep=4mm, left=30pt, borderline north={15pt}{0pt}{imeyellow}, grow sidewards by = 2.3cm}

\posterbox[skin=titlebox]
          {name=title, column=1, row=1, span=12, rowspan=.5, below = top}{

    \huge
    \rmfamily\bfseries
    \centering
    An example poster using \LaTeX{}\\
    by CCSL-IME/USP
}

\posterbox[skin=footer]{name=footer, column=1, above=bottom, span=12}{
    \rmfamily\bfseries\large
    ccsl.ime.usp.br\par
    \vspace{4pt}
    \small\ttfamily
    ccsl@ime.usp.br\par
    \vspace{4pt}
    \footnotesize\rmfamily
    \textcolor{imesoftblue!30!white}{Department of Computer Science --- University of São Paulo}\relax
    \insertfootimage\par
    \vspace{-.4\baselineskip}
}




\posterbox[adjusted title = {The CCSL logo (full width)}, left*=.8\baselineskip, right*=.8\baselineskip, top=.8\baselineskip, bottom=.8\baselineskip]
          {name=widelogo, column = 1, below=title, span = 12}{

    \centering
    \includegraphics[width=.95\textwidth]{ccsl-logo}
}

%%%%%%%% Quatro colunas %%%%%%%%
\posterbox[adjusted title = One fourth width]
          {name=quarterlogo, column=1, below=widelogo, span=3}{

    \centering
    \includegraphics[width=.9\textwidth]{ccsl-logo}
}

\posterbox[adjusted title = One fourth width]
          {column=4, below=widelogo, span=3}{

    \centering
    \includegraphics[width=.9\textwidth]{ccsl-logo}
}

\posterbox[adjusted title = One fourth width]
          {column=7, below=widelogo, span=3}{

    \centering
    \includegraphics[width=.9\textwidth]{ccsl-logo}
}

\posterbox[adjusted title = One fourth width]
          {column=10, below=widelogo, span=3}{

    \centering
    \includegraphics[width=.9\textwidth]{ccsl-logo}
}

\posterbox[adjusted title = Pangrams (half width)]
          {name=pangramdef,column=1, below=quarterlogo,span=6, rowspan=.9}{

    \begin{itemize}
      \item A \textbf{pangram} is a sentence using every letter of a given
            alphabet at least once.
      \item Pangrams have been used to:

      \begin{itemize}
        \item display typefaces
        \item test equipment
        \item develop skills in handwriting, calligraphy, and keyboarding
      \end{itemize}

    \end{itemize}
}

\posterbox[adjusted title = Examples of pangrams (half width)]
          {name=pangramexamples,below=pangramdef,column=1,span=6, rowspan=1.2}{

    \begin{itemize}
      \item In English

      \begin{itemize}
        \item The quick brown fox jumps over the lazy dog
        \item Sphinx of black quartz, judge my vow
        \item How vexingly quick daft zebras jump
        \item Pack my box with five dozen liquor jugs
      \end{itemize}

      \item In Portuguese

      \begin{itemize}
        \item Vejo xá gritando que fez show sem playback
        \item Já fiz vinho com toque de kiwi para belga sexy
        \item Dê já multa ao punk sexy que fez viação chegar à web
        \item Vejo galã sexy pôr quinze kiwis à força em baú achatado
      \end{itemize}

    \end{itemize}
}

\posterbox[adjusted title = The IME/USP logo (half width)]
          {name=halflogo,column=7, below=quarterlogo,span=6, rowspan=2.1}{

    \centering
    \includegraphics[width=.9\textwidth,trim=0 0 70 0,clip]{ime-logo}\par
}

%%%%%%% Colunas assimétricas %%%%%%%%

\makeatletter
\defbibenvironment{bibliography}
{\list
{\color{imeblue}\raisebox{.1em}{\@IMEcharscale{.26}{\ensuremath{\blacktriangleright}}}}
{\setlength{\leftmargin}{\bibhang}%
\setlength{\itemindent}{0pt}%
\setlength{\itemsep}{\bibitemsep}%
\setlength{\parsep}{\bibparsep}}}
{\endlist}
{\item}
\makeatother


\posterbox[adjusted title = Bibliography (assimetrical columns -- two thirds width), equal height group = bib-and-table]
          {name=bibliography,column=1, below=halflogo,span=8}{

    \nocite{FSF:GNU-GPL, MenaChalco08, biblatex}
    \printbibliography[heading=none]
}

\posterbox[adjusted title = A Table (one third width), equal height group = bib-and-table]
          {column=9, below=halflogo,span=4}{

    \begin{table}[H] % [H] é obrigatório com beamer!
      \centering
      \singlespacing\vspace{-\baselineskip}
      \begin{tabular}{ccl}
        \toprule
        Code        & Abbreviation & Name       \\
        \midrule
        \texttt{A}  & Ala          & Alanine    \\
        \texttt{C}  & Cys          & Cysteine   \\
        \texttt{W}  & Trp          & Tryptophan \\
        \texttt{Y}  & Tyr          & Tyrosine   \\
        \bottomrule
      \end{tabular}
    \end{table}
}

%%%%%%%% Três colunas %%%%%%%%
\posterbox[adjusted title = Sponsors (one third width)]
          {name=sponsors,column=1, between=bibliography and footer,span=4}{

    \centering
    \begin{figure}[H] % [H] é obrigatório com beamer!
      \begin{subfigure}[c]{.3\textwidth}
        \centering
        %\raisebox{.0972\baselineskip}{\includegraphics[height=.927\baselineskip]{fapesp-logo}}
        \includegraphics[width=.9\textwidth]{ccsl-logo}
      \end{subfigure}
      \begin{subfigure}[c]{.3\textwidth}
        \centering
        %\raisebox{-.53496\baselineskip}{\includegraphics[height=2.16\baselineskip]{capes-logo}}
        \includegraphics[width=.9\textwidth]{ccsl-logo}
      \end{subfigure}
      \begin{subfigure}[c]{.3\textwidth}
        \centering
        %\includegraphics[height=1.08\baselineskip]{cnpq-logo}
        \includegraphics[width=.9\textwidth]{ccsl-logo}
      \end{subfigure}
    \end{figure}
}

\posterbox[adjusted title = Sponsors (one third width)]
          {column=5, between=bibliography and footer,span=4}{

    \centering
    \begin{figure}[H] % [H] é obrigatório com beamer!
      \begin{subfigure}[c]{.3\textwidth}
        \centering
        %\raisebox{.0972\baselineskip}{\includegraphics[height=.927\baselineskip]{fapesp-logo}}
        \includegraphics[width=.9\textwidth]{ccsl-logo}
      \end{subfigure}
      \begin{subfigure}[c]{.3\textwidth}
        \centering
        %\raisebox{-.53496\baselineskip}{\includegraphics[height=2.16\baselineskip]{capes-logo}}
        \includegraphics[width=.9\textwidth]{ccsl-logo}
      \end{subfigure}
      \begin{subfigure}[c]{.3\textwidth}
        \centering
        %\includegraphics[height=1.08\baselineskip]{cnpq-logo}
        \includegraphics[width=.9\textwidth]{ccsl-logo}
      \end{subfigure}
    \end{figure}
}

\posterbox[adjusted title = Sponsors (one third width)]
          {column=9, between=bibliography and footer,span=4}{

    \centering
    \begin{figure}[H] % [H] é obrigatório com beamer!
      \begin{subfigure}[c]{.3\textwidth}
        \centering
        %\raisebox{.0972\baselineskip}{\includegraphics[height=.927\baselineskip]{fapesp-logo}}
        \includegraphics[width=.9\textwidth]{ccsl-logo}
      \end{subfigure}
      \begin{subfigure}[c]{.3\textwidth}
        \centering
        %\raisebox{-.53496\baselineskip}{\includegraphics[height=2.16\baselineskip]{capes-logo}}
        \includegraphics[width=.9\textwidth]{ccsl-logo}
      \end{subfigure}
      \begin{subfigure}[c]{.3\textwidth}
        \centering
        %\includegraphics[height=1.08\baselineskip]{cnpq-logo}
        \includegraphics[width=.9\textwidth]{ccsl-logo}
      \end{subfigure}
    \end{figure}
}

\end{tcbposter}

\end{document}
