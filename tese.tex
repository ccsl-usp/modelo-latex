% Arquivo LaTeX de exemplo de dissertação/tese a ser apresentada à CPG do IME-USP
%
% Criação: Jesús P. Mena-Chalco
% Revisão: Fabio Kon e Paulo Feofiloff
% Adaptação para UTF8, biblatex e outras melhorias: Nelson Lago
%
% Except where otherwise indicated, these files are distributed under
% the MIT Licence. The example text, which includes the tutorial and
% examples as well as the explanatory comments in the source, are
% available under the Creative Commons Attribution International
% Licence, v4.0 (CC-BY 4.0) - https://creativecommons.org/licenses/by/4.0/


%%%%%%%%%%%%%%%%%%%%%%%%%%%%%%%%%%%%%%%%%%%%%%%%%%%%%%%%%%%%%%%%%%%%%%%%%%%%%%%%
%%%%%%%%%%%%%%%%%%%%%%%%%%%%%%% PREÂMBULO LaTeX %%%%%%%%%%%%%%%%%%%%%%%%%%%%%%%%
%%%%%%%%%%%%%%%%%%%%%%%%%%%%%%%%%%%%%%%%%%%%%%%%%%%%%%%%%%%%%%%%%%%%%%%%%%%%%%%%

% A opção twoside (frente-e-verso) significa que a aparência das páginas pares
% e ímpares pode ser diferente. Por exemplo, as margens podem ser diferentes ou
% os números de página podem aparecer à direita ou à esquerda alternadamente.
% Mas nada impede que você crie um documento "só frente" e, ao imprimir, faça
% a impressão frente-e-verso.
%
% Aqui também definimos a língua padrão do documento
% (a última da lista) e línguas adicionais.
%\documentclass[12pt,twoside,brazilian,english]{book}
\documentclass[12pt,twoside,english,brazilian]{book}

% Ao invés de definir o tamanho das margens, vamos definir os tamanhos do
% texto, do cabeçalho e do rodapé, e deixamos a package geometry calcular
% o tamanho das margens em função do tamanho do papel. Assim, obtemos o
% mesmo resultado impresso, mas com margens diferentes, se o tamanho do
% papel for diferente.
\usepackage[a4paper]{geometry}

\geometry{
  textwidth=152mm,
  hmarginratio=12:17, % 24:34 -> com papel A4, 24mm + 152mm + 34mm = 210mm
  textheight=237mm,
  vmarginratio=8:7, % 32:28 -> com papel A4, 32mm + 237mm + 28mm = 297mm
  headsep=11mm, % distância entre a base do cabeçalho e o texto
  headheight=21mm, % qualquer medida grande o suficiente, p.ex., top - headsep
  footskip=10mm,
  marginpar=20mm,
  marginparsep=5mm,
}

% Vários pacotes e opções de configuração genéricos; para personalizar o
% resultado, modifique estes arquivos.
%%%%%%%%%%%%%%%%%%%%%%%%%%%%%%%%%%%%%%%%%%%%%%%%%%%%%%%%%%%%%%%%%%%%%%%%%%%%%%%%
%%%%%%%%%%%%%%%%%%%%%%% CONFIGURAÇÕES E PACOTES BÁSICOS %%%%%%%%%%%%%%%%%%%%%%%%
%%%%%%%%%%%%%%%%%%%%%%%%%%%%%%%%%%%%%%%%%%%%%%%%%%%%%%%%%%%%%%%%%%%%%%%%%%%%%%%%

% Vários comandos auxiliares para o desenvolvimento de packages e classes;
% aqui, usamos em alguns comandos de formatação e condicionais.
\usepackage{etoolbox}
\usepackage{xstring}
\usepackage{xparse}
\usepackage{regexpatch}
% Algumas packages dependem de xpatch e tentam carregá-la, causando conflitos
% com regexpatch. Como regexpatch oferece todos os recursos de xpatch (ela
% é uma versão estendida de xpatch, mas ainda considerada experimental), vamos
% fazê-las acreditar que xpatch já foi carregada.
\expandafter\xdef\csname ver@xpatch.sty\endcsname{2012/10/02}

\usepackage{calc}
% Sempre que possível, é melhor usar os recursos de etoolbox ao invés de
% ifthen; no entanto, várias packages dependem dela.
%\usepackage{ifthen}
% Estas não estão em uso mas podem ser úteis.
%\usepackage{ltxcmds}
%\usepackage{letltxmacro}

% Esta package permite detectar XeTeX, LuaTeX e pdfTeX, mas pode não estar
% disponível em todas as instalações de TeX.
%\usepackage{iftex}
% Por conta disso, usaremos estas (que não detectam pdfTeX):
\usepackage{ifxetex}
\usepackage{ifluatex}

\newbool{unicodeengine}
\ifboolexpr{bool{xetex} or bool{luatex}}
  {\booltrue{unicodeengine}}
  {\boolfalse{unicodeengine}}

% Detecta se estamos produzindo um arquivo PDF ou DVI (lembrando que tanto
% pdfTeX quanto LuaTeX podem gerar ambos)
\usepackage{ifpdf}

% Algumas packages "padrão" da AMS, que são praticamente  obrigatórias.
% Algumas delas devem ser carregadas antes de unicode-math ou das
% definições das fontes do documento.
\usepackage{amssymb}
\usepackage{amsthm}
\usepackage{amsmath}
\usepackage{mathtools}

% "fontenc" é um parâmetro do NFSS (sistema de gestão de fontes do
% LaTeX; consulte "texdoc fntguide" e "texdoc fontenc"). Mesmo
% usando fontspec (alternativa a ele compatível apenas com lualatex
% e xelatex), é aconselhável configurar fontenc corretamente. O
% default é OT1, mas ele tem algumas limitações; a mais importante
% é que, com ele, palavras acentuadas não podem ser hifenizadas.
% Por conta disso, quase todos os documentos LaTeX utilizam o
% fontenc T1. A escolha do fontenc tem consequências para as fontes
% que podem ser usadas com NFSS; hoje em dia T1 tem mais opções
% de qualidade, então não se perde nada em usá-lo.
\usepackage[T1]{fontenc}

\ifunicodeengine
  % Não é preciso carregar inputenc com LuaTeX e XeTeX, pois
  % com eles utf8 é obrigatório.
  \usepackage{fontspec}
  \usepackage{unicode-math}
\else
  % O texto está escrito em utf8.
  \usepackage[utf8]{inputenc}
\fi

% Internacionalização dos nomes das seções ("chapter" X "capítulo" etc.),
% hifenização e outras convenções tipográficas. babel deve ser um dos
% primeiros pacotes carregados. É possível passar a língua do documento
% como parâmetro aqui, mas já fizemos isso ao carregar a classe, no início
% do documento.
\usepackage{babel}

% É possível personalizar as palavras-chave que babel utiliza, por exemplo:
%\addto\extrasbrazil{\renewcommand{\chaptername}{Chap.}}
% Com BibTeX, isso vale também para a bibliografia; com BibLaTeX, é melhor
% usar o comando "DefineBibliographyStrings".

% Para línguas baseadas no alfabeto latino, como o inglês e o português,
% o pacote babel funciona muito bem, mas com outros alfabetos ele às vezes
% falha. Por conta disso, o pacote polyglossia foi criado para substituí-lo.
% polyglossia só funciona com LuaTeX e XeTeX; como babel também funciona com
% esses sistemas, provavelmente não há razão para usar polyglossia, mas é
% possível que no futuro esse pacote se torne o padrão.
%\usepackage{polyglossia}
%\setdefaultlanguage{brazil}
%\setotherlanguage{english}

% Alguns pacotes (espeficicamente, tikz) usam, além de babel, este pacote
% como auxiliar para a tradução de palavras-chave, como os meses do ano.
\usepackage{translator}

% microajustes no tamanho das letras, espaçamento etc. para melhorar
% a qualidade visual do resultado. LaTeX tradicional não dá suporte a
% nenhum tipo de microajuste; pdfLaTeX dá suporte a todos. LuaLaTeX
% e XeLaTeX dão suporte a alguns:
%
% * expansion não funciona com XeLaTeX
% * tracking não funciona com XeLaTeX; é possível obter o mesmo resultado
%   com a opção "LetterSpace" do pacote fontspec, mas a configuração é
%   totalmente manual. Por padrão, aumenta o afastamento entre caracteres
%   nas fontes "small caps"; o resultado não se presta ao uso na
%   bibliografia ou citações, então melhor desabilitar.
% * kerning e spacing só funcionam com pdfLaTex; ambas são funções
%   consideradas experimentais e nem sempre produzem resultados vantajosos.

\newcommand\microtypeopts{
  protrusion=true,
  tracking=false,
  kerning=false,
  spacing=false
}

% TeXLive 2018 inclui a versão 2.7a da package microtype e a versão
% 1.07 de luatex. Essa combinação faz aparecer um bug:
% https://tex.stackexchange.com/questions/476740/microtype-error-with-lualatex-attempt-to-call-field-warning-a-nil-value
% Aqui, aplicamos a solução sugerida, que não tem "contra-indicações".
\ifluatex
  \usepackage{luatexbase}
\fi

\ifxetex
  \usepackage[expansion=false,\microtypeopts]{microtype}
\else
  \usepackage[expansion=true,\microtypeopts]{microtype}
\fi

% Alguns "truques" (sujos?) para minimizar over/underfull boxes.
%
% Para fazer um texto justificado, é preciso modificar o tamanho dos espaços
% em cada linha para mais ou para menos em relação ao seu tamanho ideal. Para
% escolher as quebras de linha, TeX vai percorrendo o texto procurando lugares
% possíveis para quebrar as linhas considerando essa flexibilidade mas dentro
% de um certo limite mínimo/máximo. Nesse processo, ele associa a cada possível
% linha o valor *badness*, que é o nível de distorção do tamanho dos espaços
% daquela linha em relação ao ideal, e ignora opções que tenham badness muito
% grande (esse limite é dado por \tolerance). Depois de encontradas todas
% as possíveis quebras de linha e a badness de cada uma, TeX calcula as
% *penalties* das quebras encontradas, que são uma medida de quebras "ruins".
% Por exemplo, na configuração padrão, quebrar uma linha hifenizando uma
% palavra gera uma penalty de 50; já uma quebra que faça a última linha
% do parágrafo ficar sozinha na página seguinte gera uma penalty de 150.
% Finalmente, TeX calcula a "feiúra" de cada possível linha (demerits)
% com base na badness e nas penalties e escolhe a solução que minimiza os
% demerits totais do parágrafo. Os comandos \linebreak e \pagebreak funcionam
% simplesmente acrescentando uma penalty negativa ao lugar desejado para a
% quebra.
%
% Para cada fonte, o espaço em TeX tem um tamanho ideal, um tamanho mínimo e um
% tamanho máximo. TeX nunca reduz um espaço para menos que o mínimo da fonte,
% mas pode aumentá-lo para mais que o máximo. Se os espaços de uma linha ficam
% com o tamanho ideal, a badness da linha é 0; se o tamanho é
% reduzido/aumentado 50% do mínimo/máximo, a badness da linha é 12; se o
% tamanho é reduzido/aumentado para o mínimo/máximo, a badness é 100. Se esse
% aumento for de 30% além do máximo, a badness da linha é 200; se for de 45%
% além do máximo, a badness é 300; se for de 60% além do máximo, a badness é
% 400; se for de 100% além do máximo, a badness é 800. O valor máximo possível
% para badness é 10.000, que significa "badness infinita".
%
% \tolerance indica a badness máxima que TeX aceita para uma linha; seu valor
% default é 200. Assim, aumentar para, digamos, 300 ou 400, permite que
% TeX escolha parágrafos com maior variação no espaçamento entre as linhas.
% No entanto, no cálculo de demerits, a badness e as penalties de cada linha
% são elevadas ao quadrado, então TeX geralmente prefere escolher outras
% opções no lugar de uma linha ruim. Por exemplo, órfãs/viúvas têm demerit
% de 22.500 e dois hífens seguidos têm demerit de 10.000; já uma linha com
% badness 400 tem demerit 160.000. Portanto, não é surpreendente que a maioria
% dos parágrafos tenha demerits abaixo de 40.000, quase todos abaixo de 100.000
% e praticamente nenhum acima de 1.000.000. Isso significa que, para a grande
% maioria dos parágrafos, aumentar \tolerance não faz diferença: uma linha com
% badness 400 nunca será efetivamente escolhida se houver qualquer outra opção
% com badness menor. Também fica claro que não há muita diferença real entre
% definir \tolerance como 800 ou 9.999.
%
% O problema muda de figura se TeX não consegue encontrar uma solução. Isso
% pode acontecer em dois casos: (1) o parágrafo tem ao menos uma linha que não
% pode ser quebrada com badness < 10.000 ou (2) o parágrafo tem ao menos uma
% linha que não pode ser quebrada com badness < tolerance (mas essa badness é
% menor que 10.000).
%
% No primeiro caso, se houver várias possibilidades de linhas que não podem ser
% quebradas, TeX não vai ser capaz de compará-las e escolher a melhor: todas
% têm a badness máxima (10.000) e, portanto, a que gerar menos deméritos no
% restante do parágrafo será a escolhida. Na realidade, no entanto, essas
% linhas *não* são igualmente ruins entre si, o que pode levar TeX a fazer uma
% má escolha. Para evitar isso, TeX tenta novamente aplicando
% \emergencystretch, que "faz de conta" que o tamanho máximo ideal dos espaços
% da linha é maior que o definido na fonte. Isso reduz a badness de todas as
% linhas, o que soa parecido com aumentar \tolerance. Há três diferenças, no
% entanto: (1) essa mudança só afeta os parágrafos que falharam; (2) soluções
% que originalmente teriam badness = 10.000 (e, portanto, seriam vistas como
% equivalentes) podem ser avaliadas e comparadas entre si; e (3) como a badness
% de todas as linhas diminui, a possibilidade de outras linhas que
% originalmente tinham badness alta serem escolhidas aumenta. Esse último ponto
% significa que \emergencystretch pode fazer TeX escolher linhas mais
% espaçadas, fazendo o espaçamento do parágrafo inteiro aumentar e, portanto,
% tornando o resultado mais homogêneo mesmo com uma linha particularmente ruim.
%
% É esse último ponto que justifica o uso de \emergencystretch no segundo caso
% também: apenas aumentar a tolerância, nesse caso, poderia levar TeX a
% diagramar uma linha ruim em meio a um parágrafo bom, enquanto
% \emergencystretch pode fazer TeX aumentar o espaçamento de maneira geral no
% parágrafo, minimizando o contraste da linha problemática com as demais.
% Colocando a questão de outra maneira, aumentar \tolerance para lidar com
% esses parágrafos problemáticos pode fazê-los ter uma linha especialmente
% ruim, enquanto \emergencystretch pode dividir o erro entre várias linhas.
% Assim, definir \tolerance em torno de 800 parece razoável: no caso geral,
% não há diferença e, se um desses casos difíceis não pode ser resolvido com
% uma linha de badness até 800, \emergencystretch deve ser capaz de gerar um
% resultado igual ou melhor.
%
% Penalties & demerits: https://tex.stackexchange.com/a/51264
% Definições (fussy, sloppy etc.): https://tex.stackexchange.com/a/241355
% Mais definições (hfuzz, hbadness etc.): https://tex.stackexchange.com/a/50850
% Donald Arseneau defendendo o uso de \sloppy: https://groups.google.com/d/msg/comp.text.tex/Dhf0xxuQ66E/QTZ7aLYrdQUJ
% Artigo detalhado sobre \emergencystretch: https://www.tug.org/TUGboat/tb38-1/tb118wermuth.pdf
% Esse artigo me leva a crer que algo em torno de 1.5em é suficiente

\tolerance=800
\hyphenpenalty=100 % Default 50; se o texto é em 2 colunas, 50 é melhor
\setlength{\emergencystretch}{1.5em}

% Não gera warnings para Overfull menor que 0.5pt
\hfuzz=.5pt
\vfuzz\hfuzz

% Não gera warnings para Underfull com badness < 1000
\hbadness=1000
\vbadness=1000

% Por padrão, o algoritmo LaTeX para textos não-justificados é (muito) ruim;
% este pacote implementa um algoritmo bem melhor
\usepackage[newcommands]{ragged2e}

% Com ragged2e e a opção "newcommands", textos curtos não-justificados
% podem gerar warnings sobre "underfull \hbox". Não há razão para pensar
% muito nesses warnings, então melhor desabilitá-los.
% https://tex.stackexchange.com/questions/17659/ragged2e-newcommands-option-produces-underfull-hbox-warnings
\makeatletter
\g@addto@macro{\centering}{\hbadness=\@M}
\g@addto@macro{\Centering}{\hbadness=\@M}
\g@addto@macro{\raggedright}{\hbadness=\@M}
\g@addto@macro{\RaggedRight}{\hbadness=\@M}
\g@addto@macro{\raggedleft}{\hbadness=\@M}
\g@addto@macro{\RaggedLeft}{\hbadness=\@M}
\g@addto@macro{\center}{\hbadness=\@M}
\g@addto@macro{\Center}{\hbadness=\@M}
\g@addto@macro{\flushleft}{\hbadness=\@M}
\g@addto@macro{\FlushLeft}{\hbadness=\@M}
\g@addto@macro{\flushright}{\hbadness=\@M}
\g@addto@macro{\FlushRight}{\hbadness=\@M}
\makeatother

% Espaçamento entre linhas configurável (\singlespacing, \onehalfspacing etc.)
\usepackage{setspace}

% LaTeX às vezes coloca notas de rodapé logo após o final do texto da
% página ao invés de no final da página; este pacote evita isso e faz
% notas de rodapé funcionarem corretamente em títulos de seções.
% Esta package deve ser carregada depois de setspace.
\usepackage[stable,bottom]{footmisc}

% Se uma página está vazia, não imprime número de página ou cabeçalho
\usepackage{emptypage}

% Carrega nomes de cores disponíveis (podem ser usados com hyperref e listings)
\usepackage[hyperref,svgnames,x11names,table]{xcolor}

% LaTeX define os comandos "MakeUppercase" e "MakeLowercase", mas eles têm
% algumas limitações; esta package define os comandos MakeTextUppercase e
% MakeTextLowercase que resolvem isso.
\usepackage{textcase}

% Em documentos frente-e-verso, LaTeX faz o final da página terminar sempre
% no mesmo lugar (exceto no final dos capítulos). Esse comportamento pode ser
% ativado explicitamente com o comando "\flushbottom". Mas se, por alguma
% razão, o volume de texto na página é "pequeno", essa página vai ter espaços
% verticais artificialmente grandes. Uma solução para esse problema é utilizar
% "\raggedbottom" (padrão em documentos que não são frente-e-verso): com essa
% opção, as páginas podem terminar em alturas ligeiramente diferentes. Outra
% opção é corrigir manualmente cada página problemática, por exemplo com o
% comando "\enlargethispage".
%\raggedbottom
\flushbottom

% Por padrão, LaTeX coloca uma espaço aumentado após sinais de pontuação;
% Isso não é tão bom quanto alguns TeX-eiros defendem :) .
% Esta opção desabilita isso e, consequentemente, evita problemas com
% "id est" (i.e.) e "exempli gratia" (e.g.)
\frenchspacing

% Trechos de texto "puro" (tabs, quebras de linha etc. não são modificados)
\usepackage{verbatim}

% LaTeX procura por arquivos adicionais no diretório atual e nos diretórios
% padrão do sistema. Assim, é preciso usar caminhos relativos para incluir
% arquivos de subdiretórios: "\input{diretorio/arquivo}". No entanto, há
% duas limitações:
%
% 1. É necessário dizer "\input{diretorio/arquivo} mesmo quando o arquivo
%    que contém esse comando já está dentro do subdiretório.
%
% 2. Isso não deve ser usado para packages ("\usepackage{diretorio/package}"),
%    embora na prática funcione.
%
% O modo recomendado de resolver esses problemas é modificando o arquivo
% texmf.cnf ou a variável de ambiente TEXINPUTS ou colocando os arquivos
% compartilhados na árvore TEXMF (geralmente, no diretório texmf dentro do
% diretório do usuário), o que é um tanto complicado para usuários menos
% experientes.
%
% O primeiro problema pode ser solucionado também com a package import,
% mas não há muita vantagem pois é preciso usar outro comando no lugar de
% "\input". O segundo problema é mais importante, pois torna muito difícil
% colocar packages adicionais em um diretório separado. Para contorná-lo,
% vamos usar um truque que é suficiente para nossa necessidade, embora
% *não* seja normalmente recomendado.
%\usepackage{import}

\newcommand\dowithsubdir[2]{
    \csletcs{@oldinput@path}{input@path}
    \csappto{input@path}{{#1}}
    #2
    \csletcs{input@path}{@oldinput@path}
}

%%%%%%%%%%%%%%%%%%%%%%%%%%%%%%%%%%%%%%%%%%%%%%%%%%%%%%%%%%%%%%%%%%%%%%%%%%%%%%%%
%%%%%%%%%%%%%%%%%%%%%%%%%%%%%%%%%%% LÍNGUAS %%%%%%%%%%%%%%%%%%%%%%%%%%%%%%%%%%%%
%%%%%%%%%%%%%%%%%%%%%%%%%%%%%%%%%%%%%%%%%%%%%%%%%%%%%%%%%%%%%%%%%%%%%%%%%%%%%%%%

\makeatletter
\ExplSyntaxOn

% We need to have at least some variant of Portuguese and of English
% loaded to generate the abstract/resumo, palavras-chave/keywords etc.
% We will make sure that both languages are present in the class options
% list by adding them if needed. With this, these options become global
% and therefore are seen by all packages (among them, babel).
%
% babel traditionally uses "portuguese", "brazilian", "portuges", or
% "brazil" to support the Portuguese language, using .ldf files. babel
% is also in the process of implementing a new scheme, using .ini
% files, based on the concept of "locales" instead of "languages". This
% mechanism uses the names "portuguese-portugal", "portuguese-brazil",
% "portuguese-pt", "portuguese-br", "portuguese", "brazilian", "pt",
% "pt-PT", and "pt-BR" (i.e., neither "portuges" nor "brazil"). To avoid
% compatibility problems, let's stick with "brazilian" or "portuguese"
% by substituting portuges and brazil if necessary.

\NewDocumentCommand\@IMEportugueseAndEnglish{m}{

  % Make sure any instances of "portuges" and "brazil" are replaced
  % by "portuguese" e "brazilian"; other options are unchanged.
  \seq_gclear_new:N \l_tmpa_seq
  \seq_gclear_new:N \l_tmpb_seq
  \seq_gset_from_clist:Nc \l_tmpa_seq {#1}

  \seq_map_inline:Nn \l_tmpa_seq{
    \def\@tempa{##1}
    \ifstrequal{portuges}{##1}
      {
        \GenericInfo{sbc2019}{}{Substituting~language~portuges~->~portuguese}
        \def\@tempa{portuguese}
      }
      {}
    \ifstrequal{brazil}{##1}
      {
        \GenericInfo{}{Substituting~language~brazil~->~brazilian}
        \def\@tempa{brazilian}
      }
      {}
    \seq_gput_right:NV \l_tmpb_seq {\@tempa}
  }

  % Remove the leftmost duplicates (default is to remove the rightmost ones).
  % Necessary in case the user did "portuges,portuguese", "brazil,brazilian"
  % or some variation: When we substitute the language, we end up with the
  % exact same language twice, which may mess up the main language selection.
  \seq_greverse:N \l_tmpb_seq
  \seq_gremove_duplicates:N \l_tmpb_seq
  \seq_greverse:N \l_tmpb_seq

  % If the user failed to select some variation of English and Portuguese,
  % we add them here. We also remember which ones of portuguese/brazilian,
  % english/american/british etc. were selected.
  \exp_args:Nnx \regex_extract_all:nnNTF
    {\b(portuguese|brazilian)\b}
    {\seq_use:Nn \l_tmpb_seq {,}}
    \l_tmpa_tl
    {
      \tl_reverse:N \l_tmpa_tl
      \xdef\@IMEpt{\tl_head:N \l_tmpa_tl}
    }
    {
      \seq_gput_left:Nn \l_tmpb_seq {brazilian}
      \gdef\@IMEpt{brazilian}
    }

  \exp_args:Nnx \regex_extract_all:nnNTF
    {\b(english|american|USenglish|canadian|british|UKenglish|australian|newzealand)\b}
    {\seq_use:Nn \l_tmpb_seq {,}}
    \l_tmpa_tl
    {
      \tl_reverse:N \l_tmpa_tl
      \xdef\@IMEen{\tl_head:N \l_tmpa_tl}
    }
    {
      \seq_gput_left:Nn \l_tmpb_seq {english}
      \gdef\@IMEen{english}
    }

  \exp_args:Nc \xdef {#1} {\seq_use:Nn \l_tmpb_seq {,}}
}


% https://tex.stackexchange.com/a/43541
% This message is part of a larger thread that discusses some
% limitations of this method, but it is enough for us here.
\def\@getcl@ss#1.cls#2\relax{\def\@currentclass{#1}}
\def\@getclass{\expandafter\@getcl@ss\@filelist\relax}
\@getclass

% The three class option lists we need to update: \@unusedoptionlist,
% \@classoptionslist and one of \opt@book.cls, \opt@article.cls etc.
% according to the current class. Note that beamer.cls (and maybe
% others) does not use \@unusedoptionlist; with it, we incorrectly
% add "english,brazilian" to \@unusedoptionlist, but that does not
% cause problems.
\@IMEportugueseAndEnglish{@unusedoptionlist}
\@IMEportugueseAndEnglish{@classoptionslist}
\@IMEportugueseAndEnglish{opt@\@currentclass .cls}

\ExplSyntaxOff
\makeatother

% Internacionalização dos nomes das seções ("chapter" X "capítulo" etc.),
% hifenização e outras convenções tipográficas. babel deve ser um dos
% primeiros pacotes carregados. É possível passar a língua do documento
% como parâmetro aqui, mas já fizemos isso ao carregar a classe, no início
% do documento.
\usepackage{babel}
\usepackage{iflang}

% É possível personalizar as palavras-chave que babel utiliza, por exemplo:
%\addto\extrasbrazilian{\renewcommand{\chaptername}{Chap.}}
% Com BibTeX, isso vale também para a bibliografia; com BibLaTeX, é melhor
% usar o comando "DefineBibliographyStrings".

% Para línguas baseadas no alfabeto latino, como o inglês e o português,
% o pacote babel funciona muito bem, mas com outros alfabetos ele às vezes
% falha. Por conta disso, o pacote polyglossia foi criado para substituí-lo.
% polyglossia só funciona com LuaTeX e XeTeX; como babel também funciona com
% esses sistemas, provavelmente não há razão para usar polyglossia, mas é
% possível que no futuro esse pacote se torne o padrão.
%\usepackage{polyglossia}
%\setdefaultlanguage{brazilian}
%\setotherlanguage{english}

% Alguns pacotes (espeficicamente, tikz) usam, além de babel, este pacote
% como auxiliar para a tradução de palavras-chave, como os meses do ano.
\usepackage{translator}

%%%%%%%%%%%%%%%%%%%%%%%%%%%%%%%%%%%%%%%%%%%%%%%%%%%%%%%%%%%%%%%%%%%%%%%%%%%%%%%%
%%%%%%%%%%%%%%%%%%%%%%%%%%%%%%%%%%% FONTE %%%%%%%%%%%%%%%%%%%%%%%%%%%%%%%%%%%%%%
%%%%%%%%%%%%%%%%%%%%%%%%%%%%%%%%%%%%%%%%%%%%%%%%%%%%%%%%%%%%%%%%%%%%%%%%%%%%%%%%

% LaTeX normalmente usa quatro tipos de fonte:
%
% * uma fonte serifada, para o corpo do texto;
% * uma fonte com design similar à anterior, para modo matemático;
% * uma fonte sem serifa, para títulos ou "entidades". Por exemplo, "a classe
%   \textsf{TimeManager} é responsável..." ou "chamamos \textsf{primos} os
%   números que...". Observe que em quase todos os casos desse tipo é mais
%   adequado usar negrito ou itálico;
% * uma fonte "teletype", para trechos de programas.
%
% A escolha de uma família de fontes para o documento normalmente é feita
% carregando uma package específica que, em geral, seleciona as quatro fontes
% de uma vez.
%
% LaTeX usa por default a família de fontes "Computer Modern". Essas fontes
% precisaram ser re-criadas diversas vezes em formatos diferentes, então há
% diversas variantes dela. Com o fontenc OT1 (default "ruim" do LaTeX), a
% versão usada é a BlueSky Computer Modern, que é de boa qualidade, mas com
% os problemas do OT1. Com fontenc T1 (padrão deste modelo e recomendado), o
% LaTeX usa o conjunto "cm-super". Com fontspec (ou seja, com LuaLaTeX e
% XeLaTeX), LaTeX utiliza a versão "Latin Modern". Ao longo do tempo, versões
% diferentes dessas fontes foram recomendadas como "a melhor"; atualmente, a
% melhor opção para usar a família Computer Modern é a versão "Latin Modern".
%
% Você normalmente não precisa lidar com isso, mas pode ser útil saber: O
% mecanismo tradicionalmente usado por LaTeX para gerir fontes é o NFSS
% (veja "texdoc fntguide"). Ele funciona com todas as versões de LaTeX,
% mas só com fontes que foram adaptadas para funcionar com LaTeX. LuaLaTeX
% e XeLaTeX podem usar NFSS mas também são capazes de utilizar um outro
% mecanismo (através da package fontspec), que permite utilizar quaisquer
% fontes instaladas no computador.

\ifPDFTeX
    % Usando pdfLaTeX

    % Ativa Latin Modern como a fonte padrão.
    \usepackage{lmodern}

    % Alguns truques para melhorar a aparência das fontes Latin Modern;
    % eles não funcionam com LuaLaTeX e XeLaTeX.

    % Latin Modern não tem fontes bold + Small Caps, mas cm-super sim;
    % assim, vamos ativar o suporte às fontes cm-super (sem ativá-las
    % como a fonte padrão do documento) e configurar substituições
    % automáticas para que a fonte Latin Modern seja substituída por
    % cm-super quando o texto for bold + Small Caps.
    \usepackage{fix-cm}

    % Com Latin Modern, é preciso incluir substituições para o encoding TS1
    % também por conta dos números oldstyle, porque para inclui-los nas fontes
    % computer modern foi feita uma hack: os dígitos são declarados como sendo
    % os números itálicos da fonte matemática e, portanto, estão no encoding TS1.
    %
    % Primeiro forçamos o LaTeX a carregar a fonte Latin Modern (ou seja, ler
    % o arquivo que inclui "DeclareFontFamily") e, a seguir, definimos a
    % substituição
    \fontencoding{TS1}\fontfamily{lmr}\selectfont
    \DeclareFontShape{TS1}{lmr}{b}{sc}{<->ssub * cmr/bx/n}{}
    \DeclareFontShape{TS1}{lmr}{bx}{sc}{<->ssub * cmr/bx/n}{}

    \fontencoding{T1}\fontfamily{lmr}\selectfont
    \DeclareFontShape{T1}{lmr}{b}{sc}{<->ssub * cmr/bx/sc}{}
    \DeclareFontShape{T1}{lmr}{bx}{sc}{<->ssub * cmr/bx/sc}{}

    % Latin Modern não tem "small caps + itálico", mas tem "small caps + slanted";
    % vamos definir mais uma substituição aqui.
    \fontencoding{T1}\fontfamily{lmr}\selectfont % já feito acima, mas tudo bem
    \DeclareFontShape{T1}{lmr}{m}{scit}{<->ssub * lmr/m/scsl}{}
    \DeclareFontShape{T1}{lmr}{bx}{scit}{<->ssub * lmr/bx/scsl}{}

    % Se fizermos mudanças manuais na fonte Latin Modern, estes comandos podem
    % vir a ser úteis
    %\newcommand\lmodern{%
    %  \renewcommand{\oldstylenums}[1]{{\fontencoding{TS1}\selectfont ##1}}%
    %  \fontfamily{lmr}\selectfont%
    %}
    %
    %\DeclareRobustCommand\textlmodern[1]{%
    %  {\lmodern #1}%
    %}
\else
    % Com LuaLaTex e XeLaTeX, Latin Modern é a fonte padrão. Existem
    % diversas packages e "truques" para melhorar alguns aspectos de
    % Latin Modern, mas eles foram feitos para pdflatex (veja mais
    % acima). Assim, se você pretende usar Latin Modern como a fonte
    % padrão do documento, é melhor usar pdfLaTeX. Deve ser possível
    % implementar essas melhorias com fontspec também, mas este modelo
    % não faz isso, apenas ativamos Small Caps aqui.

    \ifLuaTeX
      % Com LuaTeX, basta indicar o nome de cada fonte; para descobrir
      % o nome "certo", use o comando "otfinfo -i" e veja os itens
      % "preferred family" e "full name"
      \setmainfont{Latin Modern Roman}[
        SmallCapsFont = {LMRomanCaps10-Regular},
        ItalicFeatures = {
          SmallCapsFont = {LMRomanCaps10-Oblique},
        },
        SlantedFont = {LMRomanSlant10-Regular},
        SlantedFeatures = {
          SmallCapsFont = {LMRomanCaps10-Oblique},
          BoldFont = {LMRomanSlant10-Bold}
        },
      ]
    \fi

    \ifXeTeX
      % Com XeTeX, é preciso informar o nome do arquivo de cada fonte.
      \setmainfont{lmroman10-regular.otf}[
        SmallCapsFont = {lmromancaps10-regular.otf},
        ItalicFeatures = {
          SmallCapsFont = {lmromancaps10-oblique.otf},
        },
        SlantedFont = {lmromanslant10-regular.otf},
        SlantedFeatures = {
          SmallCapsFont = {lmromancaps10-oblique.otf},
          BoldFont = {lmromanslant10-bold.otf}
        },
      ]
    \fi
\fi

% Algumas packages mais novas que tratam de fontes funcionam corretamente
% tanto com fontspec (LuaLaTeX/XeLaTeX) quanto com NFSS (qualquer versão
% de LaTeX, mas menos poderoso que fontspec). No entanto, muitas funcionam
% apenas com NFSS. Nesse caso, em LuaLaTeX/XeLaTeX é melhor usar os
% comandos de fontspec, como exemplificado mais abaixo.

% É possível mudar apenas uma das fontes. Em particular, a fonte
% teletype da família Computer Modern foi criada para simular
% as impressoras dos anos 1970/1980. Sendo assim, ela é uma fonte (1)
% com serifas e (2) de espaçamento fixo. Hoje em dia, é mais comum usar
% fontes sem serifa para representar código-fonte. Além disso, ao imprimir,
% é comum adotar fontes que não são de espaçamento fixo para fazer caber
% mais caracteres em uma linha de texto. Algumas opções de fontes para
% esse fim:
%\usepackage{newtxtt} % Não funciona com fontspec (lualatex / xelatex)
%\usepackage{DejaVuSansMono}
% inconsolata é uma boa fonte, mas não tem variante itálico
%\ifPDFTeX
%  \usepackage[narrow]{inconsolata}
%\else
%  \setmonofont{inconsolatan}
%\fi
\usepackage[scale=.85]{sourcecodepro}

% Ao invés da família Computer Modern, é possível usar outras como padrão.
% Uma ótima opção é a libertine, similar (mas não igual) à Times mas com
% suporte a Small Caps e outras qualidades. A fonte teletype da família
% é serifada, então é melhor definir outra; a opção "mono=false" faz
% o pacote não carregar sua própria fonte, mantendo a escolha anterior.
% Versões mais novas de LaTeX oferecem um fork desta fonte, libertinus.
% As packages libertine/libertinus funcionam corretamente com pdfLaTeX,
% LuaLaTeX e XeLaTeX.
% TODO: remover suporte a Libertine no final de 2022
\makeatletter
\IfFileExists{libertinus.sty}
    {
      \usepackage[mono=false]{libertinus}
      % Com LuaLaTeX/XeLaTeX, Libertinus configura também
      % a fonte matemática; aqui só precisamos corrigir \mathit
      \ifLuaTeX
        \setmathfontface\mathit{Libertinus Serif Italic}
      \fi
      \ifXeTeX
        % O nome de arquivo da fonte mudou na versão 2019-04-04
        \@ifpackagelater{libertinus-otf}{2019/04/03}
            {\setmathfontface\mathit{LibertinusSerif-Italic.otf}}
            {\setmathfontface\mathit{libertinusserif-italic.otf}}
      \fi
    }
    {
      % Libertinus não está disponível; vamos usar libertine
      \usepackage[mono=false]{libertine}

      % Com Libertine, é preciso modificar também a fonte
      % matemática, além de \mathit
      \ifLuaTeX
        \setmathfont{Libertinus Math}
        \setmathfontface\mathit{Linux Libertine O Italic}
      \fi

      \ifXeTeX
        \setmathfont{libertinusmath-regular.otf}
        \setmathfontface\mathit{LinLibertine_RI.otf}
      \fi
    }
\makeatother

\ifPDFTeX
  % A família libertine por padrão não define uma fonte matemática
  % específica para pdfLaTeX; uma opção que funciona bem com ela:
  %\usepackage[libertine]{newtxmath}
  % Outra, provavelmente melhor:
  \usepackage{libertinust1math}
\fi

% Ativa apenas a fonte biolinum, que é a fonte sem serifa da família.
%\IfFileExists{libertinus.sty}
%  \usepackage[sans]{libertinus}
%\else
%  \usepackage{biolinum}
%\fi

% Também é possível usar a Times como padrão; nesse caso, a fonte
% sem serifa usualmente é a Helvetica. Mas provavelmente libertine
% é uma opção melhor.
%\ifPDFTeX
%  \usepackage[helvratio=0.95,largesc]{newtxtext}
%  \usepackage{newtxtt} % Fonte teletype
%  \usepackage{newtxmath}
%\else
%  % Clone da fonte Times como fonte principal
%  \setmainfont{TeX Gyre Termes}
%  \setmathfont[Scale=MatchLowercase]{TeX Gyre Termes Math}
%  % TeX Gyre Termes Math tem um bug e não define o caracter
%  % \setminus; Vamos contornar esse problema usando apenas
%  % esse caracter da fonte STIX Two Math
%  \setmathfont[range=\setminus]{STIX Two Math}
%  % Clone da fonte Helvetica como fonte sem serifa
%  \setsansfont{TeX Gyre Heros}
%  % Clone da Courier como fonte teletype, mas provavelmente
%  % é melhor utilizar sourcecodepro
%  %\setmonofont{TeX Gyre Cursor}
%\fi

% Cochineal é outra opção de qualidade; ela define apenas a fonte
% com serifa.
%
% Com NFSS (recomendado no caso de cochineal):
%\usepackage{cochineal}
%\usepackage[cochineal,vvarbb]{newtxmath}
%\usepackage[cal=boondoxo]{mathalfa}
%
% Com fontspec (até a linha "setmathfontface..."):
%
%\setmainfont{Cochineal}[
%  Extension=.otf,
%  UprightFont=*-Roman,
%  ItalicFont=*-Italic,
%  BoldFont=*-Bold,
%  BoldItalicFont=*-BoldItalic,
%  %Numbers={Proportional,OldStyle},
%]
%
%\DeclareRobustCommand{\lfstyle}{\addfontfeatures{Numbers=Lining}}
%\DeclareTextFontCommand{\textlf}{\lfstyle}
%\DeclareRobustCommand{\tlfstyle}{\addfontfeatures{Numbers={Tabular,Lining}}}
%\DeclareTextFontCommand{\texttlf}{\tlfstyle}
%
%% Cochineal não tem uma fonte matemática; com fontspec, provavelmente
%% o melhor a fazer é usar libertinus.
%\setmathfont{Libertinus Math}
%\setmathfontface\mathit{Cochineal-Italic.otf}

% gentium inclui apenas uma fonte serifada, similar a Garamond, que busca
% cobrir todos os caracteres unicode
%\usepackage{gentium}

% LaTeX normalmente funciona com fontes que foram adaptadas para ele, ou
% seja, ele não usa as fontes padrão instaladas no sistema: para usar
% uma fonte é preciso ativar o pacote correspondente, como visto acima.
% É possível escapar dessa limitação e acessar as fontes padrão do sistema
% com XeTeX ou LuaTeX. Com eles, além dos pacotes de fontes "tradicionais",
% pode-se usar o pacote fontspec para usar fontes do sistema.
%\usepackage{fontspec}
%\setmainfont{DejaVu Serif}
%\setmainfont{Charis SIL}
%\setsansfont{DejaVu Sans}
%\setsansfont{Libertinus Sans}[Scale=1.1]
%\setmonofont{DejaVu Sans Mono}

% fontspec oferece vários recursos interessantes para manipular fontes.
% Por exemplo, Garamond é uma fonte clássica; a versão EBGaramond é muito
% boa, mas não possui versões bold e bold-italic; aqui, usamos
% CormorantGaramond ou Gentium para simular a versão bold.
%\setmainfont{EBGaramond12}[
%  Numbers        = {Lining,} ,
%  Scale          = MatchLowercase ,
%  UprightFont    = *-Regular ,
%  ItalicFont     = *-Italic ,
%  BoldFont       = gentiumbasic-bold ,
%  BoldItalicFont = gentiumbasic-bolditalic ,
%%  BoldFont       = CormorantGaramond Bold ,
%%  BoldItalicFont = CormorantGaramond Bold Italic ,
%]
%
%\newfontfamily\garamond{EBGaramond12}[
%  Numbers        = {Lining,} ,
%  Scale          = MatchLowercase ,
%  UprightFont    = *-Regular ,
%  ItalicFont     = *-Italic ,
%  BoldFont       = gentiumbasic-bold ,
%  BoldItalicFont = gentiumbasic-bolditalic ,
%%  BoldFont       = CormorantGaramond Bold ,
%%  BoldItalicFont = CormorantGaramond Bold Italic ,
%]

% Crimson tem Small Caps, mas o recurso é considerado "em construção".
% Vamos utilizar Gentium para Small Caps
%\setmainfont{Crimson}[
%  Numbers           = {Lining,} ,
%  Scale             = MatchLowercase ,
%  UprightFont       = *-Roman ,
%  ItalicFont        = *-Italic ,
%  BoldFont          = *-Bold ,
%  BoldItalicFont    = *-Bold Italic ,
%  SmallCapsFont     = Gentium Plus ,
%  SmallCapsFeatures = {Letters=SmallCaps} ,
%]
%
%\newfontfamily\crimson{Crimson}[
%  Numbers           = {Lining,} ,
%  Scale             = MatchLowercase ,
%  UprightFont       = *-Roman ,
%  ItalicFont        = *-Italic ,
%  BoldFont          = *-Bold ,
%  BoldItalicFont    = *-Bold Italic ,
%  SmallCapsFont     = Gentium Plus ,
%  SmallCapsFeatures = {Letters=SmallCaps} ,
%]

% Com o pacote fontspec, também é possível usar o comando "\fontspec" para
% selecionar uma fonte temporariamente, sem alterar as fontes-padrão do
% documento.

%%%%%%%%%%%%%%%%%%%%%%%%%%%%%%%%%%%%%%%%%%%%%%%%%%%%%%%%%%%%%%%%%%%%%%%%%%%%%%%%
%%%%%%%%%%%%%%%%%%%%%%%%%%%%% FIGURAS / FLOATS %%%%%%%%%%%%%%%%%%%%%%%%%%%%%%%%%
%%%%%%%%%%%%%%%%%%%%%%%%%%%%%%%%%%%%%%%%%%%%%%%%%%%%%%%%%%%%%%%%%%%%%%%%%%%%%%%%

% Permite importar figuras. LaTeX "tradicional" só é capaz de trabalhar com
% figuras EPS. Hoje em dia não há nenhuma boa razão para usar essa versão;
% pdfTeX, XeTeX, e LuaTeX podem usar figuras nos formatos PDF, JPG e PNG; EPS
% também pode funcionar em algumas instalações mas não é garantido, então é
% melhor evitar.
\usepackage{graphicx}

% Mais tipos de float e mais opções para personalização; este pacote
% também acrescenta a possibilidade de definir "H" como opção de
% posicionamento do float, que significa "aqui, incondicionalmente".
\usepackage{float}

% Por padrão, LaTeX prefere colocar floats no topo da página que
% onde eles foram definidos; vamos mudar isso. Este comando depende
% do pacote "float", carregado logo acima.
\floatplacement{table}{htbp}
\floatplacement{figure}{htbp}

% Garante que floats (tabelas e figuras) só apareçam após as seções a que
% pertencem. Por padrão, se a seção começa no meio da página, LaTeX pode
% colocar a figura no topo dessa página
\usepackage{flafter}
% Às vezes um float pode ser adiado por muitas páginas; é possível forçar
% LaTeX a imprimir todos os floats pendentes com o comando \clearpage.
% Esta package acrescenta o comando \FloatBarrier, que garante que floats
% definidos anteriormente sejam impressos e garante que floats subsequentes
% não apareçam antes desse ponto. A opção "section" faz o comando ser
% aplicado automaticamente a cada nova seção. "above" e "below" desabilitam
% a barreira quando os floats estão na mesma página.
\usepackage[section,above,below]{placeins}

% LaTeX escolhe automaticamente o "melhor" lugar para colocar cada float.
% Por padrão, ele tenta colocá-los no topo da página e depois no pé da
% página; se não tiver sucesso, vai para a página seguinte e recomeça.
% Se esse algoritmo não tiver sucesso "logo", LaTeX cria uma página só
% com floats. É possível modificar esse comportamento com as opções de
% posicionamento: "tp", por exemplo, instrui LaTeX a não colocar floats
% no pé da página, e "htbp" o instrui para tentar "aqui" como a primeira
% opção. O pacote "float" acrescenta a opção "H", que significa "aqui,
% incondicionalmente".
%
% A escolha do "melhor" lugar leva em conta os parâmetros abaixo, mas é
% possível ignorá-los com a opção de posicionamento "!". Dado que os
% valores default não são muito bons para floats "grandes" ou documentos
% com muitos floats, é muito comum usar "!" ou "H". No entanto, modificando
% esses parâmetros o algoritmo automático tende a funcionar melhor. Ainda
% assim, vale ler a discussão a respeito na seção "Limitações do LaTeX"
% deste modelo.

% Fração da página que pode ser ocupada por floats no topo. Default: 0.7
\renewcommand{\topfraction}{.85}
% Idem para documentos em colunas e floats que tomam as 2 colunas. Default: 0.7
\renewcommand{\dbltopfraction}{.66}
% Fração da página que pode ser ocupada por floats no pé. Default: 0.3
\renewcommand{\bottomfraction}{.7}
% Fração mínima da página que deve conter texto. Default: 0.2
\renewcommand{\textfraction}{.15}
% Numa página só de floats, fração mínima que deve ser ocupada. Default: 0.5
\renewcommand{\floatpagefraction}{.66}
% Idem para documentos em colunas e floats que tomam as 2 colunas. Default: 0.5
\renewcommand{\dblfloatpagefraction}{.66}
% Máximo de floats no topo da página. Default: 2
\setcounter{topnumber}{9}
% Idem para documentos em colunas e floats que tomam as 2 colunas. Default: 2
\setcounter{dbltopnumber}{9}
% Máximo de floats no pé da página. Default: 1
\setcounter{bottomnumber}{9}
% Máximo de floats por página. Default: 3
\setcounter{totalnumber}{20}

% Define o ambiente "\begin{landscape} -- \end{landscape}"; o texto entre
% esses comandos é impresso em modo paisagem, podendo se estender por várias
% páginas. A rotação não inclui os cabeçalhos e rodapés das páginas.
% O principal uso desta package é em conjunto com a package longtable: se
% você precisa mostrar uma tabela muito larga (que precisa ser impressa em
% modo paisagem) e longa (que se estende por várias páginas), use
% "\begin{landscape}" e "\begin{longtable}" em conjunto. Note que o modo
% landscape entra em ação imediatamente, ou seja, "\begin{landscape}" gera
% uma quebra de página no local em que é chamado. Na maioria dos casos, o
% que se quer não é isso, mas sim um "float paisagem"; isso é o que a
% package rotating oferece (veja abaixo).
\usepackage{pdflscape}

% Define dois novos tipos de float: sidewaystable e sidewaysfigure, que
% imprimem a figura ou tabela sozinha em uma página em modo paisagem. Além
% disso, permite girar elementos na página de diversas outras maneiras.
\usepackage[figuresright,clockwise]{rotating}

% Captions com fonte menor, indentação normal, corpo do texto
% negrito e nome do caption itálico
\usepackage[
  font=small,
  format=plain,
  labelfont=bf,up,
  textfont=it,up]{caption}

% Sub-figuras (e seus captions) - observe que existe uma package chamada
% "subfigure", mas ela é obsoleta; use esta no seu lugar.
\usepackage{subcaption}

% Permite criar imagens com texto ao redor
\usepackage{wrapfig}

% Permite incorporar um arquivo PDF como uma página adicional. Útil se
% for necessário importar uma imagem ou tabela muito grande ou ainda
% para definir uma capa personalizada.
\usepackage{pdfpages}

% Caixas de texto coloridas
%\usepackage{tcolorbox}


%%%%%%%%%%%%%%%%%%%%%%%%%%%%%%%%%%%%%%%%%%%%%%%%%%%%%%%%%%%%%%%%%%%%%%%%%%%%%%%%
%%%%%%%%%%%%%%%%%%%%%%%%%%%%%%%%%% TABELAS %%%%%%%%%%%%%%%%%%%%%%%%%%%%%%%%%%%%%
%%%%%%%%%%%%%%%%%%%%%%%%%%%%%%%%%%%%%%%%%%%%%%%%%%%%%%%%%%%%%%%%%%%%%%%%%%%%%%%%

% Tabelas simples são fáceis de fazer em LaTeX; tabelas com alguma sofisticação
% são trabalhosas, pois é difícil controlar alinhamento, largura das colunas,
% distância entre células etc. Ou seja, é muito comum que a tabela final fique
% "torta". Por isso, em muitos casos, vale mais a pena gerar a tabela em uma
% planilha, como LibreOffice calc ou excel, transformar em PDF e importar como
% figura, especialmente se você quer controlar largura/altura das células
% manualmente etc. No entanto, se você quiser fazer as tabelas em LaTeX para
% garantir a consistência com o tipo e o tamanho das fontes, é possível e o
% resultado é muito bom. Aqui há alguns pacotes que incrementam os recursos de
% tabelas do LaTeX e alguns comandos pré-prontos que podem facilitar um pouco
% seu uso.

% LaTeX por padrão não permite notas de rodapé dentro de tabelas;
% este pacote acrescenta essa funcionalidade.
\usepackage{tablefootnote}

% Estende o ambiente tabular para que, além de "l", "c", "r" para definir se uma
% coluna deve ser alinhada à esquerda, centralizada ou à direita, seja possível
% definir a largura das colunas (além de outras pequenas modificações). Isso é
% muito útil porque LaTeX não "percebe" automaticamente quando é mais
% interessante fazer uma coluna mais estreita e forçar quebras de linha nas
% células correspondentes.
\usepackage{array}

% Se você quer ter um pouco mais de controle sobre o tamanho de cada coluna da
% tabela, utilize estes tipos de coluna (criados com base nos recursos do pacote
% array). É só usar algo como M{número}, onde "número" (por exemplo, 0.4) é a
% fração de \textwidth que aquela coluna deve ocupar. "M" significa que o
% conteúdo da célula é centralizado; "L", alinhado à esquerda; "J", justificado;
% "R", alinhado à direita. Obviamente, a soma de todas as frações não pode ser
% maior que 1, senão a tabela vai ultrapassar a linha da margem.
\newcolumntype{M}[1]{>{\centering}m{#1\textwidth}}
\newcolumntype{L}[1]{>{\RaggedRight}m{#1\textwidth}}
\newcolumntype{R}[1]{>{\RaggedLeft}m{#1\textwidth}}
\newcolumntype{J}[1]{m{#1\textwidth}}

% Permite alinhar os elementos de uma coluna pelo ponto decimal
\usepackage{dcolumn}

% Define tabelas do tipo "longtable", similares a "tabular" mas que podem ser
% divididas em várias páginas. "longtable" também funciona corretamente com
% notas de rodapé. Note que, como uma longtable pode se estender por várias
% páginas, não faz sentido colocá-las em um float "table". Por conta disso,
% longtable define o comando "\caption" internamente.
\usepackage{longtable}

% Permite agregar linhas de tabelas, fazendo colunas "compridas"
\usepackage{multirow}

% Cria comando adicional para possibilitar a inserção de quebras de linha
% em uma célula de tabela, entre outros
\usepackage{makecell}

% Às vezes a tabela é muito larga e não cabe na página. Se os cabeçalhos da
% tabela é que são demasiadamente largos, uma solução é inclinar o texto das
% células do cabeçalho. Para fazer isso, use o comando "\rothead".
\renewcommand{\rothead}[2][60]{\makebox[11mm][l]{\rotatebox{#1}{\makecell[c]{#2}}}}

% Se quiser criar uma linha mais grossa no meio de uma tabela, use
% o comando "\thickhline".
\newlength\savedwidth
\newcommand\thickhline{
  \noalign{
    \global\savedwidth\arrayrulewidth
    \global\arrayrulewidth 1.5pt
  }
  \hline
  \noalign{\global\arrayrulewidth\savedwidth}
}

% Modifica (melhora) o layout default das tabelas e acrescenta os comandos
% \toprule, \bottomrule, \midrule e \cmidrule
\usepackage{booktabs}

% Permite colorir linhas, colunas ou células
%\usepackage{colortbl}


%%%%%%%%%%%%%%%%%%%%%%%%%%%%%%%%%%%%%%%%%%%%%%%%%%%%%%%%%%%%%%%%%%%%%%%%%%%%%%%%
%%%%%%%%%%%%%%% CAPA E PÁGINAS PRELIMINARES (TESE/DISSERTAÇÃO)  %%%%%%%%%%%%%%%%
%%%%%%%%%%%%%%%%%%%%%%%%%%%%%%%%%%%%%%%%%%%%%%%%%%%%%%%%%%%%%%%%%%%%%%%%%%%%%%%%

\usepackage{trimspaces}

% Formatação de datas de acordo com a língua
\usepackage[useregional]{datetime2}

\makeatletter

%%%%%%%%%%%%%%%%%%%%%%%%%%%%%%%%%%%%%%%%%%%%%%%%%%%%%%%%%%%%%%%%%%%%%%%%%%%%%%%%
%%%%%%%%%%%%%%%%%%%%% TEXTOS PADRÃO EM PT E EN PARA A CAPA %%%%%%%%%%%%%%%%%%%%%
%%%%%%%%%%%%%%%%%%%%%%%%%%%%%%%%%%%%%%%%%%%%%%%%%%%%%%%%%%%%%%%%%%%%%%%%%%%%%%%%

% \extrasLANGUAGE vs \captionsLANGUAGE: https://tex.stackexchange.com/a/354197/217608

% Palavras fixas a serem traduzidas
\providecommand\keywordsname{} % Keywords / Palavras-chave
\providecommand\programname{} % Program / Programa
\providecommand\committeename{} % Examining committee / Comissão julgadora
\providecommand\advisorname{} % Advisor / Orientador(a)
\providecommand\coadvisorname{} % Co-advisor / Coorientador(a)
\providecommand\workname{} % Report, Thesis / Tese, Dissertação, Monografia
\providecommand\degreename{} % Masters, Doctorate, Bachelor / Mestrado, Doutorado, Bacharelado
\providecommand\titlename{} % Master, Doctor, Bachelor / Mestre(a), Doutor(a), Bacharel

% Textos longos a serem traduzidos
\providecommand\@coverTCCText{}
\providecommand\@coverQualiText{}
\providecommand\@coverThesisText{}
\providecommand\@institutionBlockText{} % Só para TCC
\providecommand\@provisionalFrontmatterText{}
\providecommand\@finalFrontmatterText{}
\providecommand\@institution{}

% Este não precisa ser traduzido, o texto em inglês não utiliza
\providecommand\@bywhom{%
  \ifdefstring{\@authorGender}{masc}
    {pelo candidato \@author}
    {pela candidata \@author}%
}

%%%%%%%%%% PORTUGUÊS %%%%%%%%%%
\expandafter\addto\csname captions\@IMEpt\endcsname{%
  \let\@title\@titlept
  \let\@subtitle\@subtitlept
  \let\@keywords\@keywordspt
  \renewcommand\keywordsname{Palavras-chave}%
  \renewcommand\programname{Programa}%
  \renewcommand\committeename{Comissão julgadora}%
  \renewcommand\advisorname{%
    \iftoggle{@tcc}{%
      \ifdefstring{\@advisorGender}{masc}
        {Supervisor}
        {Supervisora}%
    }{%
      \ifdefstring{\@advisorGender}{masc}
        {Orientador}
        {Orientadora}%
    }%
  }%
  \renewcommand\coadvisorname[1]{%
    \iftoggle{@tcc}{%
      \ifcsstring{@coadvisor#1Gender}{masc}
        {Cossupervisor}
        {Cossupervisora}%
    }{%
      \ifcsstring{@coadvisor#1Gender}{masc}
        {Coorientador}
        {Coorientadora}%
    }%
  }%
  \renewcommand\workname{%
    \iftoggle{@tcc}
      {Monografia}
      {\iftoggle{@qualificacao}
        {Exame de Qualificação}
        {\iftoggle{@doutorado}
          {Tese}
          {Dissertação}%
        }%
      }%
  }%
  \renewcommand\degreename{%
    \iftoggle{@doutorado}
      {Doutorado}
      {\iftoggle{@mestrado}
        {Mestrado}
        {\iftoggle{@tcc}
          {Bacharelado}
          {Nível não definido!}%
        }%
      }%
  }%
  \renewcommand\titlename{%
    \iftoggle{@doutorado}
      {\ifdefstring{\@authorGender}{masc}{Doutor}{Doutora}}
      {\iftoggle{@mestrado}
        {\ifdefstring{\@authorGender}{masc}{Mestre}{Mestra}}
        {\iftoggle{@tcc}
          {Bacharel}{Nível não definido!}%
        }%
      }%
  }%
  %
  %
  \renewcommand\@coverTCCText{%
    Monografia Final\vspace{.5\baselineskip}\\
    \@macCDXCIX{} --- Trabalho de\\
    Formatura Supervisionado%
  }%
  \renewcommand\@coverQualiText{%
    Relatório apresentado ao\\
    Instituto de Matemática e Estatística\\
    da Universidade de São Paulo\\
    para exame de qualificação de\\
    \degreename{} em Ciências%
  }%
  \renewcommand\@coverThesisText{%
    \workname{} apresentada ao\\
    Instituto de Matemática e Estatística\\
    da Universidade de São Paulo\\
    para obtenção do título de\\
    \titlename{} em Ciências%
  }%
  \renewcommand\@institutionBlockText{%
    Universidade de São Paulo\\
    Instituto de Matemática e Estatística\\
    Bacharelado em Ciência da Computação%
  }%
  \renewcommand\@provisionalFrontmatterText{%
    \iftoggle{@qualificacao}{%
      Esta é a versão original do texto de qualificação elaborado
      \@bywhom{}, tal como submetido à Comissão Julgadora.%
    }{%
      Esta é a versão original da \MakeLowercase{\workname} elaborada
      \@bywhom{}, tal como submetida à Comissão Julgadora.%
    }%
  }%
  \renewcommand\@finalFrontmatterText{%
    Esta versão da \MakeLowercase{\workname} contém as correções e alterações
    sugeridas pela Comissão Julgadora durante a defesa da versão
    original do trabalho, realizada em \DTMusedate{@defensedate}.\\[1\baselineskip]
    Uma cópia da versão original está disponível no Instituto de
    Matemática e Estatística da Universidade de São Paulo.%
  }%
  \renewcommand\@institution{%
    Instituto de Matemática e Estatística,
    Universidade de São Paulo%
  }%
}


%%%%%%%%%% INGLÊS %%%%%%%%%%
\expandafter\addto\csname captions\@IMEen\endcsname{%
  \let\@title\@titleen
  \let\@subtitle\@subtitleen
  \let\@keywords\@keywordsen
  \renewcommand\keywordsname{Keywords}%
  \renewcommand\programname{Program}%
  \renewcommand\committeename{Examining Committee}
  \renewcommand\advisorname{%
    \iftoggle{@tcc}{Supervisor}{Advisor}%
  }%
  \renewcommand\coadvisorname[1]{%
    \iftoggle{@tcc}{Co-supervisor}{Coadvisor}%
  }%
  % "Tese" e "dissertação" têm sentido contrário em língua inglesa:
  % http://guides.lib.berkeley.edu/dissertations_theses
  % https://www.grad.ubc.ca/handbook-graduate-supervision/graduate-thesis
  % Como "Thesis" é o nome genérico, vamos usar para mestrado e doutorado
  %
  %%%%%
  %
  % Nomes possíveis para o TCC em inglês:
  %
  % * monograph/monography
  %     usado para trabalho de alto nível de um autor "senior",
  %     então não faz sentido para um trabalho de graduação.
  %
  % * undergraduate thesis / bachelor's thesis
  %     plausível, mas no nosso caso report parece melhor.
  %
  % * senior project / senior thesis / honor thesis
  %     usado para "TCCs" de caráter fortemente acadêmico;
  %     não é o caso aqui.
  %
  % * essay / report
  %     razoável, porque trata-se de um texto/relato
  %     sobre o projeto de TCC.
  \renewcommand\workname{%
    \iftoggle{@tcc}
      {Capstone Project Report}
      {\iftoggle{@qualificacao}
        {Qualifying Exam}
        {Thesis}%
      }%
  }%
  \renewcommand\degreename{%
    \iftoggle{@doutorado}
      {Doctorate}
      {\iftoggle{@mestrado}
        {Master's}
        {\iftoggle{@tcc}
          {Bachelor}
          {Nível não definido!}%
        }%
      }%
  }%
  \renewcommand\titlename{%
    \iftoggle{@doutorado}
      {Doctor}
      {\iftoggle{@mestrado}
        {Master}
        {\iftoggle{@tcc}
          {Bachelor}%
          {Nível não definido!}%
        }%
      }%
  }%
  %
  %
  \renewcommand\@coverTCCText{%
    Final Essay\vspace{.5\baselineskip}\\
    \@macCDXCIX{} --- Capstone Project%
  }%
  \renewcommand\@coverQualiText{%
    Report presented to the\\
    Institute of Mathematics and Statistics\\
    of the University of São Paulo\\
    for the \titlename{} of Science\\
    qualifying examination\\%
  }%
  \renewcommand\@coverThesisText{%
    \workname{} presented to the\\
    Institute of Mathematics and Statistics\\
    of the University of São Paulo\\
    in partial fulfillment\\
    of the requirements\\
    for the degree of\\
    \titlename{} of Science%
  }%
  \renewcommand\@institutionBlockText{%
    University of São Paulo\\
    Institute of Mathematics and Statistics\\
    Bachelor of Computer Science%
  }%
  \renewcommand\@provisionalFrontmatterText{%
    \iftoggle{@qualificacao}{%
      This is the original version of the qualifying text prepared
      by candidate \@author, as submitted to the Examining Committee.%
    }{%
      This is the original version of the \MakeLowercase{\workname} prepared
      by candidate \@author, as submitted to the Examining Committee.%
    }%
  }%
  \renewcommand\@finalFrontmatterText{%
    This version of the \MakeLowercase{\workname} includes the corrections
    and modifications suggested by the Examining Committee during
    the defense of the original version of the work, which took
    place on \DTMusedate{@defensedate}.\\[1\baselineskip]
    A copy of the original version is available at the Institute of
    Mathematics and Statistics of the University of São Paulo.%
  }%
  \renewcommand\@institution{%
    Institute of Mathematics and Statistics,
    University of São Paulo%
  }%
}


%%%%%%%%%%%%%%%%%%%%%%%%%%%%%%%%%%%%%%%%%%%%%%%%%%%%%%%%%%%%%%%%%%%%%%%%%%%%%%%%
%%%%%%%%%%%%%%%%%%%%%%% COLETA E DEFINIÇÃO DE METADADOS %%%%%%%%%%%%%%%%%%%%%%%%
%%%%%%%%%%%%%%%%%%%%%%%%%%%%%%%%%%%%%%%%%%%%%%%%%%%%%%%%%%%%%%%%%%%%%%%%%%%%%%%%

\renewcommand\author[2][masc]{
  \gdef\@author{#2}
  \gdef\@authorGender{#1}
}

\NewDocumentCommand{\orientador}{O{masc} m}{
  \gdef\@advisor{#2}
  \gdef\@advisorGender{#1}
}

% Mais de um coorientador é raro, mas acontece
\ExplSyntaxOn
\newcounter{numberOfCoadvisors}
\NewDocumentCommand\coorientador{O{masc} m}{
    \stepcounter{numberOfCoadvisors}
    \tl_gclear_new:c {@coadvisor\Roman{numberOfCoadvisors}}
    \tl_gclear_new:c {@coadvisor\Roman{numberOfCoadvisors}Gender}

    \tl_set:cn {@coadvisor\Roman{numberOfCoadvisors}} {#2}
    \tl_set:cn {@coadvisor\Roman{numberOfCoadvisors}Gender} {#1}
}

\seq_gclear_new:N \@committeeMembers

\newtoggle{@mestrado}
\newtoggle{@doutorado}
\newtoggle{@tcc}
\newtoggle{@qualificacao}
\newtoggle{@finalversion}

% Opções usando LaTeX3 (veja texdoc l3keys).
\keys_define:nn { IME / defense }
  {
    % Chaves à esquerda definem as variáveis à direita
    data .code:n= {\DTMsavedate{@defensedate}{#1}},
    data .value_required:n = true,
    nivel .choice:,
    nivel / mestrado .code:n = {\@mestrado},
    nivel / masters .code:n = {\@mestrado},
    nivel / dissertacao .code:n = {\@mestrado},
    nivel / doutorado .code:n = {\@doutorado},
    nivel / phd .code:n = {\@doutorado},
    nivel / tese .code:n = {\@doutorado},
    nivel / graduacao .code:n = {\@tcc},
    nivel / bachelor .code:n = {\@tcc},
    nivel / tcc .code:n = {\@tcc},
    nivel .value_required:n = true,
    quali .code:n = {\ifstrequal{#1}{true}{\toggletrue{@qualificacao}}{\togglefalse{@qualificacao}}},
    quali .default:n = {true},
    definitiva .code:n = {\ifstrequal{#1}{true}{\toggletrue{@finalversion}}{\togglefalse{@finalversion}}},
    definitiva .default:n = {true},
    provisoria .code:n = {\ifstrequal{#1}{true}{\togglefalse{@finalversion}}{\toggletrue{@finalversion}}},
    provisoria .default:n = {true},
    programa .tl_gset:N = \@program,
    program .value_required:n = true,
    apoio .tl_gset:N = \@financing,
    apoio .value_required:n = true,
    local .tl_gset:N = \@defenselocation,
    local .value_required:n = true,
    direitos .tl_gset:N = \@license,
    direitos .value_required:n = true,
    fichacatalografica .tl_gset:N = \@catalogindata,
    fichacatalografica .value_required:n = true,
    membrobanca .code:n = {\seq_gput_right:Nn \@committeeMembers {#1}},
    membrobanca .value_required:n = true,
  }

\NewDocumentCommand\defesa{m}{\keys_set:nn {IME/defense}{#1}}

\seq_gclear_new:N \@seqkeywordspt
\seq_gclear_new:N \@seqkeywordsen
\newcommand*{\palavrachave}[1]{\seq_gput_right:Nn \@seqkeywordspt {#1}}
\newcommand*{\keyword}[1]{\seq_gput_right:Nn \@seqkeywordsen {#1}}

% Na impressão, as palavras-chave são separadas por pontos
\newcommand*{\@keywordspt}{\seq_use:Nn \@seqkeywordspt {.\space}.}
\newcommand*{\@keywordsen}{\seq_use:Nn \@seqkeywordsen {.\space}.}

% Para inclusão nos metadados com hyperxmp, são separadas por vírgulas
\newcommand*{\@commakeywordspt}{\seq_use:Nn \@seqkeywordspt {,}}
\newcommand*{\@commakeywordsen}{\seq_use:Nn \@seqkeywordsen {,}}

\ExplSyntaxOff

\NewDocumentCommand{\@doutorado}{}{
  \toggletrue{@doutorado}
  \togglefalse{@mestrado}
  \togglefalse{@tcc}
}

\NewDocumentCommand{\@mestrado}{}{
  \togglefalse{@doutorado}
  \toggletrue{@mestrado}
  \togglefalse{@tcc}
}

\NewDocumentCommand{\@tcc}{}{
  \togglefalse{@mestrado}
  \togglefalse{@doutorado}
  \toggletrue{@tcc}
}

% Defaults quando o usuário não define alguma dessas variáveis

\author{Autor não definido!}
\orientador{Orientador não definido!}
\DTMsavedate{@defensedate}{1970-01-01}
\providecommand\@program{Programa não definido!}
\providecommand\@financing{}
\providecommand\@defenselocation{Local não definido!}
\providecommand\@license{Direitos não definidos!}
\providecommand\@title{Título não definido!}
\providecommand\@titlept{Título em português não definido!}
\providecommand\@titleen{Título em inglês não definido!}
\providecommand\@shorttitle{título curto não definido!}
\providecommand\@resumo{Resumo não definido!}
\providecommand\@abstract{Abstract não definido!}


%%%%%%%%%%%%%%%%%%%%%%%%%%%%%%%%%%%%%%%%%%%%%%%%%%%%%%%%%%%%%%%%%%%%%%%%%%%%%%%%
%%%%%%%%%%%%%%%%%%%%%%%%%%%%%% TÍTULO E SUBTÍTULO %%%%%%%%%%%%%%%%%%%%%%%%%%%%%%
%%%%%%%%%%%%%%%%%%%%%%%%%%%%%%%%%%%%%%%%%%%%%%%%%%%%%%%%%%%%%%%%%%%%%%%%%%%%%%%%

\ExplSyntaxOn

% Opções usando LaTeX3 (veja texdoc l3keys).
\keys_define:nn { IME / title }
  {
    % Chaves à esquerda definem as variáveis à direita
    shorttitle .tl_gset:N = \@shorttitle,
    shorttitle .value_required:n = true,
    titlept .tl_gset:N = \@titlept,
    titlept .value_required:n = true,
    titleen .tl_gset:N = \@titleen,
    titleen .value_required:n = true,
    subtitlept .tl_gset:N = \@subtitlept,
    subtitlept .value_required:n = true,
    subtitleen .tl_gset:N = \@subtitleen,
    subtitleen .value_required:n = true,
  }

\RenewDocumentCommand\title{m}{
  \keys_set:nn {IME/title}{#1}

  % Ambos devem existir. Este é o default, mas o valor de fato é definido
  % por \captionsLANGUAGE.
  \ifdefvoid{\@titlept}
    {\let\@title\@titleen}
    {\let\@title\@titlept}

  % Estes talvez não existam, mas se um existe o outro deve existir também.
  % Este é o default, mas o valor de fato é definido por \captionsLANGUAGE.
  \ifdefvoid{\@subtitlept}
    {\let\@subtitle\@subtitleen}
    {\let\@subtitle\@subtitlept}

  \tl_if_blank:VT \@shorttitle
    {
      \let\@shorttitle\@title
      \@IMEremoveLinebreaksEtc{\@shorttitle}
    }
}

\ExplSyntaxOff


%%%%%%%%%%%%%%%%%%%%%%%%%%%%%%%%%%%%%%%%%%%%%%%%%%%%%%%%%%%%%%%%%%%%%%%%%%%%%%%%
%%%%%%%%%%%%%%%%%%%%%%%%%%%%%%%%%% DEDICATÓRIA %%%%%%%%%%%%%%%%%%%%%%%%%%%%%%%%%
%%%%%%%%%%%%%%%%%%%%%%%%%%%%%%%%%%%%%%%%%%%%%%%%%%%%%%%%%%%%%%%%%%%%%%%%%%%%%%%%

% A dedicatória vai em uma página separada, sem numeração,
% com o texto alinhado à direita e margens esquerda e
% superior muito grandes. Vamos fazer isso com uma minipage.
\newenvironment{dedicatoria} {
  \hypersetup{pageanchor=false} % Veja comentário em \maketitle

  \if@openright\cleardoublepage\else\clearpage\fi

  \thispagestyle{empty}
  \vspace*{140mm plus 0mm minus 100mm}
  \noindent
  \begin{FlushRight}
     \begin{minipage}[b][100mm][b]{100mm}
       \begin{FlushRight}
         \itshape
} {
       \end{FlushRight}
     \end{minipage}\hspace*{3em}
  \end{FlushRight}
  \vspace*{50mm plus 0mm minus 10mm}
  \if@openright\cleardoublepage\else\clearpage\fi

  \hypersetup{pageanchor=true}
}


%%%%%%%%%%%%%%%%%%%%%%%%%%%%%%%%%%%%%%%%%%%%%%%%%%%%%%%%%%%%%%%%%%%%%%%%%%%%%%%%
%%%%%%%%%%%%%%%%%%%%%%%%%%%%%%%%%%% RESUMO %%%%%%%%%%%%%%%%%%%%%%%%%%%%%%%%%%%%%
%%%%%%%%%%%%%%%%%%%%%%%%%%%%%%%%%%%%%%%%%%%%%%%%%%%%%%%%%%%%%%%%%%%%%%%%%%%%%%%%

% A página de resumo deve existir em português e inglês; ambas as versões
% utilizam o mesmo environment.

\NewDocumentCommand{\resumo}{+m}{\long\gdef\@resumo{#1}}
\DeclareDocumentCommand{\abstract}{+m}{\long\gdef\@abstract{#1}}

\newcommand\printResumoAbstract{
  \bgroup\bgroup % Dois grupos aninhados, veja a documentação da package babel
  \expandafter\selectlanguage\expandafter{\@IMEpt}
  \begin{IMEabstract}\@resumo\end{IMEabstract}
  \expandafter\selectlanguage\expandafter{\@IMEen}
  \begin{IMEabstract}\@abstract\end{IMEabstract}
  \egroup\egroup
}


\NewDocumentEnvironment{IMEabstract}{} {
  \if@openright\cleardoublepage\else\clearpage\fi
  \thispagestyle{empty}

    \begin{Center}\Large\bfseries\abstractname\end{Center}

  \vspace*{2em plus 1em minus 1em}

  \footnotesize

  % Esse é o jeito mais simples de mudar as margens de um parágrafo:
  % faz de conta que é uma lista
  \begin{list}{}{\rightmargin 4em \leftmargin 4em}
    \item\@selfReference
  \end{list}

  \vspace*{1em plus 1em minus 0em}
} {
  % Impede uma quebra de página entre esta linha e a próxima, ou seja,
  % entre a última linha do resumo/abstract e as palavras-chave.
  \@afterheading

  \vspace*{1em plus 1em minus .5em}

  \begingroup

      \setlength{\leftmargini}{\widthof{\textbf{\keywordsname:}\quad}}
      \setlength{\labelwidth}{\widthof{\textbf{\keywordsname:}}}
      \setlength{\labelsep}{\widthof{\quad}}

      \begin{description}\item[\keywordsname:]\@keywords\end{description}

  \endgroup
}


%%%%%%%%%%%%%%%%%%%%%%%%%%%%%%%%%%%%%%%%%%%%%%%%%%%%%%%%%%%%%%%%%%%%%%%%%%%%%%%%
%%%%%%%%%%%%%%%%%%%%%% IMPRIME A CAPA E A FOLHA DE ROSTO %%%%%%%%%%%%%%%%%%%%%%%
%%%%%%%%%%%%%%%%%%%%%%%%%%%%%%%%%%%%%%%%%%%%%%%%%%%%%%%%%%%%%%%%%%%%%%%%%%%%%%%%

\RenewDocumentCommand\maketitle{}{
  % Embora as páginas iniciais *pareçam* não ter numeração, a numeração
  % existe, só não é impressa. Os comandos \frontmatter, \mainmatter,
  % \pagenumbering etc. reiniciam a contagem de páginas quando os números
  % passam a ser impressos. Isso significa que há mais de uma página com
  % o número "1". O pacote hyperref não lida bem com essa situação, então
  % vamos desabilitar hyperlinks para números de páginas aqui.
  \hypersetup{pageanchor=false}
  \bgroup
  \onehalfspacing
  \@IMEcover
  \iftoggle{@tcc}{}{\@IMEtitlePage}
  \egroup
  \hypersetup{pageanchor=true}
}

% Layout da capa
\NewDocumentCommand{\@IMEcover}{} {
  \@calculateCoverMargins

  \cleardoublepage

  \thispagestyle{empty}

  \begin{hyphenrules}{nohyphenation}

    \iftoggle{@tcc}{\@institutionBlock}

    \@titleBlock

    \vfill

    \@detailsBlock

  \end{hyphenrules}

  \if@openright\cleardoublepage\else\clearpage\fi
}

% Layout para a página de rosto (duas versões, de acordo
% com a Resolução CoPGr 6018 de 13/10/2011)
\NewDocumentCommand{\@IMEtitlePage}{} {
  \@calculateCoverMargins

  \cleardoublepage

  \thispagestyle{empty}

  \begin{hyphenrules}{nohyphenation}

    \@titleBlock

    \vspace*{2cm plus 2cm minus 1cm}

    \@versionInfoBlock

    \vspace*{3.5cm plus 3cm minus 3.5cm}

    \iftoggle{@finalversion}{\@committeeBlock}{}

    \vspace*{2cm plus 2cm minus 2cm}

  \end{hyphenrules}

  \clearpage

  \thispagestyle{empty}

  \vspace*{4cm plus 4cm minus 2cm}

  \@versoPageBlock

  \vspace*{8cm plus 5cm minus 6cm}

  \if@openright\cleardoublepage\else\clearpage\fi
}


%%%%%%%%%%%%%%%%%%%%%%%%%%%%%%%%%%%%%%%%%%%%%%%%%%%%%%%%%%%%%%%%%%%%%%%%%%%%%%%%
%%%%%%%%%%%%%%%%%%%%%%%% POSIÇÃO DOS ELEMENTOS NA CAPA %%%%%%%%%%%%%%%%%%%%%%%%%
%%%%%%%%%%%%%%%%%%%%%%%%%%%%%%%%%%%%%%%%%%%%%%%%%%%%%%%%%%%%%%%%%%%%%%%%%%%%%%%%

% O IME usa uma capa padrão de cartolina para todas as teses/dissertações.
% Essa capa tem uma janela recortada por onde se vê o título e o autor do
% trabalho. Ela fica centralizada na página, tem 100m de largura, 60mm de
% altura e começa 47mm abaixo do topo da página. Como o documento já tem
% margens definidas pelo usuário, precisamos calcular quanto precisamos
% acrescentar ou subtrair dessas margens para colocar o título e autor
% na posição exata (na verdade, com uma pequena folga: 49mm abaixo do topo
% da página, 96mm de largura e 56mm de altura).
%
% Para centralizar horizontalmente, poderíamos pensar em usar "\center",
% mas isso não funciona porque ele centraliza o texto em relação à coluna
% de texto, não à página. Assim, como as margens esquerda e direita do
% documento podem ser diferentes, a janela não ficaria na posição correta.
% O que faremos, então, é colocar essa janela em uma minipage e calcular
% a margem esquerda para que essa minipage fique centralizada.
%
% Além disso, outros elementos da capa também não podem ser centralizados
% com "\center", porque eles ficariam desalinhados em relação à janela
% com o título e autor. Vamos colocar esses outros elementos em uma
% minipage também, mas de tamanho diferente da anterior.
%
% Então, precisamos calcular três valores: a margem adicional em relação ao
% topo da página, a margem esquerda da janela com título e autor e a margem
% esquerda para os demais elementos centralizados da página.

\newcommand*{\@calculateCoverMargins}{
  % Calcula o valor das margens; chamando este comando explicitamente
  % quando necessário ao invés de calcular os valores durante a
  % inicialização garante que vamos calcular após o pacote geometry
  % ter definido as margens.

  % A distância entre o topo da página e o início do texto (fora o cabeçalho)
  % é dada por (1in + \voffset + \headsep + \topmargin + \headheight).
  % Queremos colocar a caixa com o título 49mm abaixo do topo, então:
  \dimgdef\@topTitleBlockMargin{49mm - (1in + \voffset + \headsep + \topmargin + \headheight)}

  % Quando \vspace é usado no início da página, ele não tem efeito; como
  % não é isso que queremos, vamos usar \vspace*. No entanto, \vspace*
  % é implementado inserindo uma \hrule de espessura zero e depois
  % acrescentando o espaço solicitado. O resultado não é exatamente
  % o esperado, pois \topskip, \parskip e \baselineskip interagem com
  % \vspace* de maneira um tanto complexa:
  % https://tex.stackexchange.com/a/247516/183146
  %
  % Aqui, vamos compensar essa diferença. Note que, se a primeira linha
  % da página tivesse um tamanho de fonte especial, seria necessário
  % usar o valor de \baselineskip correspondente a essa fonte. Além
  % disso, definimos espaçamento simples porque o \vspace* mencionado
  % acima é executado com espaçamento simples.
  \bgroup
  \setstretch {\setspace@singlespace}% \singlespacing adds \baselineskip
  \dimgdef\@topTitleBlockMargin{\@topTitleBlockMargin - \baselineskip - \parskip}
  \egroup

  % Queremos colocar a caixa com o título centralizada na página. "\center"
  % centraliza em função da área de texto, não da página inteira, então
  % não podemos usá-lo, pois as margens esquerda e direita podem ser
  % diferentes. A distância entre a borda esquerda/interna do papel e o
  % início do texto é dada por (1in + \hoffset + \oddsidemargin), então:
  \dimgdef\@leftTitleBlockMargin{(\paperwidth - 96mm)/2 - (1in + \hoffset + \oddsidemargin)}
  \dimgdef\@coverLeftMargin{(\paperwidth - 160mm)/2 - (1in + \hoffset + \oddsidemargin)}
}


%%%%%%%%%%%%%%%%%%%%%%%%%%%%%%%%%%%%%%%%%%%%%%%%%%%%%%%%%%%%%%%%%%%%%%%%%%%%%%%%
%%%%%%%%%%%%% OS ELEMENTOS QUE COMPÕEM A CAPA E A FOLHA DE ROSTO %%%%%%%%%%%%%%%
%%%%%%%%%%%%%%%%%%%%%%%%%%%%%%%%%%%%%%%%%%%%%%%%%%%%%%%%%%%%%%%%%%%%%%%%%%%%%%%%

% Com fontspec (ou seja, lualatex/xelatex), o comando \oldstylenums funciona
% com qualquer fonte que tenha suporte a números old-style. Já com pdflatex,
% o comando para escolher números old style depende da fonte em uso. Nesse
% caso, se não soubermos qual a fonte atual (ou seja, não é nem libertine
% nem libertinus), vamos usar latin modern e torcer para o resultado não ser
% muito discrepante do restante do texto.

% 499 = CDXCIX
\@ifpackageloaded{fontspec}
  {\providecommand{\@macCDXCIX}{mac~\oldstylenums{499}}}
  {
    \providecommand{\@macCDXCIX}{{\fontfamily{lmr}\selectfont mac~\oldstylenums{499}}}

    \@ifpackageloaded{libertinus}
      {\renewcommand{\@macCDXCIX}{\LibertinusSerifOsF mac~499}}
      {}

    \@ifpackageloaded{libertine}
      {\renewcommand{\@macCDXCIX}{\libertineOsF mac~499}}
      {}
  }

\newcommand{\@coverText}{
  \bgroup
  \setstretch{.9}

  \iftoggle{@tcc}
    {\@coverTCCText}
    {\iftoggle{@qualificacao}{\@coverQualiText}{\@coverThesisText}}
  \par
  \egroup
}

\ExplSyntaxOn
\newcounter{@IMEtmpcnt}
\newcommand*{\@coverPeople} {%
  \begin{tabular}{rl}
    \iftoggle{@tcc}{}{\programname : & \@program \tabularnewline}
    \advisorname : & \@advisor \tabularnewline
    \setcounter{@IMEtmpcnt}{0}%
    \int_while_do:nNnn {\value{@IMEtmpcnt}} < {\value{numberOfCoadvisors}} {%
      \stepcounter{@IMEtmpcnt}%
      \coadvisorname{\Roman{@IMEtmpcnt}}: & \csuse{@coadvisor\Roman{@IMEtmpcnt}} \tabularnewline
    }%
  \end{tabular}
}
\ExplSyntaxOff

\newcommand{\@selfReference} {%
  \bgroup
  \@IMEremoveLinebreaksEtc{\@title}%
  \@IMEremoveLinebreaksEtc{\@subtitle}%
  \@IMEremoveLinebreaksEtc{\@author}%
  \@author.
  \textbf{\@title\ifdefvoid{\@subtitle}{}{: \textit{\@subtitle}}}.
  \workname{} (\degreename).
  \@institution,
  São Paulo, \DTMfetchyear{@defensedate}.%
  \egroup
}

\NewDocumentCommand{\@versoPageBlock}{} {
  \bgroup
  \onehalfspacing
  \begin{list}{}{\rightmargin 3em \leftmargin 3em}
    \item
      \@license

      \ifcsvoid{@catalogingData} {} {
        \vspace*{3cm plus 3cm minus 1cm}
        \setlength{\fboxsep}{20pt}
        \begin{Center}
        \fbox{
          \begin{minipage}[t]{120mm}
            \setlength\parskip{1em}

            \@catalogingData

          \end{minipage}
        }
        \end{Center}
      }
  \end{list}
  \egroup
}

% Só para TCC
\newcommand{\@institutionBlock}{

    % A posição do quadro de título é fixa em relação à página;
    % a posição deste quadro é definida em função da posição do
    % quadro de título. Assim, primeiro vamos encontrar onde
    % deve começar o quadro do título. Veja os comentários em
    % \@titleBlock para entender o mecanismo.
    \bgroup
    \setstretch {\setspace@singlespace}% \singlespacing adds \baselineskip

    \vspace*{\@topTitleBlockMargin}
    \ifdeflength{\@normalstrutheight}
      {}
      {\newlength{\@normalstrutheight}}
    \settoheight{\@normalstrutheight}{\strut}
    \vspace{-\@normalstrutheight}

    % Estamos alinhados com o quadro do título do trabalho,
    % mas não é isso que queremos: a parte inferior deste
    % quadro deve ficar 15mm acima do quadro de título e
    % este quadro tem 20mm de altura, então precisamos subir:
    \vspace{-20mm} % Espaço ocupado por este quadro
    \vspace{-15mm} % Espaço entre este quadro e o quadro de título

    \noindent\strut%
    \hspace*{\@coverLeftMargin}%
%    \fbox{%
      \begin{minipage}[t][20mm][s]{160mm}
        \vspace{0pt plus 20mm}

        \Centering\large

        \textsc{\@institutionBlockText}

        \vspace{0pt plus 20mm}
      \end{minipage}
%    }% fbox
    \par

    % Agora precisamos voltar o "cursor" para o começo da página
    % para que o quadro de título seja inserido no lugar certo.
    % Para isso, vamos:
    %
    % 1. Chegar novamente ao início do quadro de título e
    %
    % 2. Retroceder o tamanho da margem superior

    % compensa o espaço inserido por \par logo acima
    \vspace{-\parskip}
    \egroup

    % A altura da minipage já compensou o \vspace{-20mm} acima;
    % ainda precisamos compensar o \vspace{-15mm}
    \vspace{15mm}

    % Agora estamos no início do quadro de título, então
    % podemos recuar exatamente o tamanho da margem superior.
    \vspace{-\@topTitleBlockMargin}
}

% O quadro com o título e o autor que deve ser visível
% através da janela na capa.
\NewDocumentCommand{\@titleBlock}{} {

    \bgroup
    \setstretch {\setspace@singlespace}% \singlespacing adds \baselineskip

    % Este espaço coloca o topo da próxima linha
    % na posição que queremos:
    \vspace*{\@topTitleBlockMargin}

    % No entanto, a próxima linha contém apenas
    % uma minipage, e definir o topo de uma linha
    % desse tipo é complicado. Assim, vamos:
    %
    % 1. Acrescentar um \strut a essa linha;
    %
    % 2. mover o baseline dessa linha para o topo do \strut;
    %
    % 3. Alinhar o topo da minipage ao baseline da linha.
    %
    % Sobre alinhamento de minipages:
    % https://en.wikibooks.org/wiki/LaTeX/Boxes

    \ifdeflength{\@normalstrutheight}
      {}
      {\newlength{\@normalstrutheight}}
    \settoheight{\@normalstrutheight}{\strut}
    \vspace{-\@normalstrutheight}

    \noindent\strut
    \hspace*{\@leftTitleBlockMargin}%
%    \fbox{%
      \begin{minipage}[t][56mm][s]{96mm}
          \vspace*{2cm plus 1.5cm minus 1.8cm}

          \Centering\large

          \textbf{\@title}

          \vspace{0.3cm plus 0.2cm minus 0.1cm}

          \textbf{\textit{\@subtitle}}

          \vspace{1cm plus 1cm minus 0.6cm}

          \@author

          \vspace*{2cm plus 1.5cm minus 1.8cm}
      \end{minipage}%
%    }% fbox
    \par
    \egroup
}

% As demais informações da capa
\NewDocumentCommand{\@detailsBlock}{} {

  \bgroup
  \onehalfspacing
  \noindent
  \hspace*{\@coverLeftMargin}%
%  \fbox{%
    \begin{minipage}[t][130mm][s]{160mm}
      \begin{Center}
        \Large

        \vspace*{0.3cm plus 0.5cm minus 0.3cm}

        \textsc{\@coverText}

        \vspace*{1.5cm plus 0.5cm minus 0.5cm}

        \large\@coverPeople

        \vspace*{2.5cm plus 1cm minus 1cm}

        \normalsize

        \@financing

        \vspace*{1cm plus 1cm minus 0.3cm}

        \@defenselocation

        \DTMusedate{@defensedate}

      \end{Center}
    \end{minipage}%
%  }% fbox
  \par
  \egroup
}

% As informações da banca que vão apenas na versão definitiva
% da página de rosto
\ExplSyntaxOn
\NewDocumentCommand{\@committeeBlock}{} {
    \bgroup
    \onehalfspacing
    \begin{minipage}[t][][t]{\textwidth}
      \begin{quote}
        \normalsize\noindent\committeename :\par
        \begin{list}{}
        {
          \setlength{\leftmargin}{0pt}
          \setlength{\itemsep}{.1\baselineskip}
          \setlength{\topsep}{\baselineskip}
        }
          \item[] \seq_use:Nn \@committeeMembers {\item[]}
        \end{list}
      \end{quote}
    \end{minipage}
    \par
    \egroup
}
\ExplSyntaxOff

% A informação sobre a versão provisória ou definitiva
\NewDocumentCommand{\@versionInfoBlock}{} {%
  % As diretrizes dizem que "A natureza do trabalho, o grau pretendido, o
  % nome da instituição a que é submetido e a área de concentração devem
  % ser alinhados a partir do meio da parte impressa da página para a
  % margem direita, tanto na folha de rosto como na folha de avaliação."
  %
  % Assim, queremos alinhar o texto à direita com uma grande margem
  % à esquerda. Uma solução simples é alinhar o texto à direita
  % e inserir uma minipage. Dentro dela, definimos o texto
  % também alinhado à direita.

  \bgroup
  \onehalfspacing
  \begin{FlushRight}
    %\fbox{
      % Margem direita + 80mm de largura significa que a minipage
      % começa mais ou menos no meio da página.
      \begin{minipage}[t][50mm][s]{80mm}
        \begin{FlushRight}
          \normalsize
          \iftoggle{@finalversion}{%
            \@finalFrontmatterText%
          } {%
            \@provisionalFrontmatterText%
          }%
        \end{FlushRight}
      \end{minipage}
      \par
    %} % fbox
  \end{FlushRight}
  \egroup
}


%%%%%%%%%%%%%%%%%%%%%%%%%%%%%%%%%%%%%%%%%%%%%%%%%%%%%%%%%%%%%%%%%%%%%%%%%%%%%%%%
%%%%%%%%%%%%%%%%%%%%%%%%%%%%%% METADADOS XMP %%%%%%%%%%%%%%%%%%%%%%%%%%%%%%%%%%%
%%%%%%%%%%%%%%%%%%%%%%%%%%%%%%%%%%%%%%%%%%%%%%%%%%%%%%%%%%%%%%%%%%%%%%%%%%%%%%%%

% Insere os metadados XMP no arquivo PDF final. Alguns desses valores são
% definidos normalmente por hyperref/hyperxmp ou em hyperlinks.tex, mas
% para teses/dissertações vamos sobrescrevê-los aqui com AtEndPreamble.
% \@IMEremoveLinebreaksEtc está definida em hyperlinks.tex.
\AtEndPreamble{

  % Remove quebras de linha, notas de rodapé etc.
  \let\@cleantitleen\@titleen
  \let\@cleansubtitleen\@subtitleen
  \let\@cleantitlept\@titlept
  \let\@cleansubtitlept\@subtitlept
  \let\@cleanabstract\@abstract
  \let\@cleanresumo\@resumo

  \@IMEremoveLinebreaksEtc{\@cleantitleen}
  \@IMEremoveLinebreaksEtc{\@cleansubtitleen}
  \@IMEremoveLinebreaksEtc{\@cleantitlept}
  \@IMEremoveLinebreaksEtc{\@cleansubtitlept}
  \@IMEremoveLinebreaksEtc{\@cleanabstract}
  \@IMEremoveLinebreaksEtc{\@cleanresumo}

  \hypersetup{
    pdfauthor={\@author},
    % TODO: Seria ótimo apontar para uma licença, mas qual?
    %pdflicenseurl={https://creativecommons.org/licenses/by-nc-nd/4.0/},
  }

  % TODO: Com versões recentes de hyperxmp (final de 2020), não é
  %       recomendado definir pdflang; no futuro, isto deve ser mudado.
  \IfLanguagePatterns{brazilian}
    {
      \hypersetup{
        pdflang={pt},
        pdfmetalang={pt},
        pdftitle={\@cleantitlept\ifdefvoid{\@cleansubtitlept}{}{: \@cleansubtitlept}},
        pdfsubject={\@cleanresumo},
        pdfkeywords={\@commakeywordspt},
      }
      % XMPLangAlt redefines "\do"; this may cause
      % problems with biblatex, so let's use a group.
      % https://github.com/plk/biblatex/issues/1105
      \bgroup
      \XMPLangAlt{en}{pdfsubject={\@cleanabstract}}
      \XMPLangAlt{en}{pdftitle={\@cleantitleen\ifdefvoid{\@cleansubtitleen}{}{: \@cleansubtitleen}}}
      \egroup
      % o item "keywords" não pode ser traduzido
    }
    {
      \hypersetup{
        pdflang={en},
        pdfmetalang={en},
        pdftitle={\@cleantitleen\ifdefvoid{\@cleansubtitleen}{}{: \@cleansubtitleen}},
        pdfsubject={\@cleanabstract},
        pdfkeywords={\@commakeywordsen},
      }
      % XMPLangAlt redefines "\do"; this may cause
      % problems with biblatex, so let's use a group.
      % https://github.com/plk/biblatex/issues/1105
      \bgroup
      \XMPLangAlt{pt}{pdfsubject={\@cleanresumo}}
      \XMPLangAlt{pt}{pdftitle={\@cleantitlept\ifdefvoid{\@cleansubtitlept}{}{: \@cleansubtitlept}}}
      \egroup
      % o item "keywords" não pode ser traduzido
    }
}


%%%%%%%%%%%%%%%%%%%%%%%%%%%%%%%%%%%%%%%%%%%%%%%%%%%%%%%%%%%%%%%%%%%%%%%%%%%%%%%%
%%%%%%%%%%%%%%%%%%%%%%%%%%%%% SUMÁRIO E SEÇÕES %%%%%%%%%%%%%%%%%%%%%%%%%%%%%%%%%
%%%%%%%%%%%%%%%%%%%%%%%%%%%%%%%%%%%%%%%%%%%%%%%%%%%%%%%%%%%%%%%%%%%%%%%%%%%%%%%%

% Coloca as linhas "Apêndices" e "Anexos" no sumário. Com a opção "inline",
% cada apêndice/anexo aparece como "Apêndice X" ao invés de apenas "X".
\usepackage{appendixlabel} % carregado do diretório extras (veja basics.tex)

% titlesec permite definir formatação personalizada de títulos, seções etc.
% Observe que titlesec é incompatível com os comandos refsection
% e refsegment do pacote biblatex!
% Esta package utiliza titlesec e implementa a possibilidade de incluir
% uma imagem no título dos capítulos com o comando \imgchapter (leia
% os comentários no arquivo da package).
\usepackage{imagechapter} % carregado do diretório extras (veja basics.tex)

\makeatother
 % capa, páginas preliminares e alguns detalhes
%%%%%%%%%%%%%%%%%%%%%%%%%%%%%%%%%%%%%%%%%%%%%%%%%%%%%%%%%%%%%%%%%%%%%%%%%%%%%%%%
%%%%%%%%%%%%%%%%%%%%%%% SUMÁRIO, CABEÇALHOS, SEÇÕES %%%%%%%%%%%%%%%%%%%%%%%%%%%%
%%%%%%%%%%%%%%%%%%%%%%%%%%%%%%%%%%%%%%%%%%%%%%%%%%%%%%%%%%%%%%%%%%%%%%%%%%%%%%%%

% Formatação personalizada do sumário, lista de tabelas/figuras etc.
%\usepackage{titletoc}

% titlesec permite definir formatação personalizada de títulos, seções etc.
% Observe que titlesec é incompatível com os comandos refsection
% e refsegment do pacote biblatex!
% Vamos usar titlesec apenas
% para fazer títulos, seções etc. não serem justificados.
\usepackage[raggedright]{titlesec}

% Permite saber o número total de páginas; útil para colocar no
% rodapé algo como "página 3 de 10" com "\thepage\ de \zpageref{LastPage}"
%\usepackage{zref-lastpage,zref-user}

% Permite definir cabeçalhos e rodapés
%\usepackage{fancyhdr}

% Personalização de cabeçalhos e rodapés com o estilo deste modelo
\usepackage{imeusp-headers} % carregado do diretório extras (veja basics.tex)

% biblatex pode ser configurado para inserir a bibliografia no sumário;
% bibtex não oferece essa possibilidade. Com esta package, resolvemos
% esse problema.
\usepackage[nottoc,notlot,notlof]{tocbibind}

% Só olha até o nível 2 (subseções) para gerar o sumário e os
% cabeçalhos, ou seja, não coloca nomes de subsubseções (nível 3)
% no sumário nem nos cabeçalhos.
\setcounter{tocdepth}{2}

% Só numera até o nível 2 (subseções, como 2.3.1), ou seja, não numera
% sub-subseções (como 2.3.1.1). Veja que isso afeta referências
% cruzadas: se você fizer \ref{uma-sub-subsecao} sem que ela seja
% numerada, a referência vai apontar para a seção um nível acima.
\setcounter{secnumdepth}{2}

% Normalmente, o capítulo de introdução não deve ser numerado, mas
% deve aparecer no sumário. Por padrão, LaTeX não oferece uma solução
% para isso, então criamos aqui os comandos \unnumberedchapter,
% \unnumberedsection e \unnumberedsubsection.
\newcommand{\unnumberedchapter}[2][]{
  \ifblank{#1}
    {
      \chapter*{#2}
      \phantomsection
      \addcontentsline{toc}{chapter}{#2}
      \chaptermark{#2}
    }
    {
      \chapter*{#2}
      \phantomsection
      \addcontentsline{toc}{chapter}{#1}
      \chaptermark{#1}
    }
}

\newcommand{\unnumberedsection}[2][]{
  \ifblank{#1}
    {
      \section*{#2}
      \phantomsection
      \addcontentsline{toc}{section}{#2}
      \sectionmark{#2}
    }
    {
      \section*{#2}
      \phantomsection
      \addcontentsline{toc}{section}{#1}
      \sectionmark{#1}
    }
}

\newcommand{\unnumberedsubsection}[2][]{
  \ifblank{#1}
    {
      \subsection*{#2}
      \phantomsection
      \addcontentsline{toc}{subsection}{#2}
    }
    {
      \subsection*{#2}
      \phantomsection
      \addcontentsline{toc}{subsection}{#1}
    }
}


%%%%%%%%%%%%%%%%%%%%%%%%%%%%%%%%%%%%%%%%%%%%%%%%%%%%%%%%%%%%%%%%%%%%%%%%%%%%%%%%
%%%%%%%%%%%%%%%%%%%%%%%%%% ESPAÇAMENTO E ALINHAMENTO %%%%%%%%%%%%%%%%%%%%%%%%%%%
%%%%%%%%%%%%%%%%%%%%%%%%%%%%%%%%%%%%%%%%%%%%%%%%%%%%%%%%%%%%%%%%%%%%%%%%%%%%%%%%

% LaTeX por default segue o estilo americano e não faz a indentação da
% primeira linha do primeiro parágrafo de uma seção; este pacote ativa essa
% indentação, como é o estilo brasileiro
\usepackage{indentfirst}

% A primeira linha de cada parágrafo costuma ter um pequeno recuo para
% tornar mais fácil visualizar onde cada parágrafo começa. Além disso, é
% possível colocar um espaço em branco entre um parágrafo e outro. Esta
% package coloca o espaço em branco e desabilita o recuo; como queremos
% o espaço *e* o recuo, é preciso guardar o valor padrão do recuo e
% redefini-lo depois de carregar a package.
% TODO: depois que ubuntu 18.04 se tornar obsoleta (abril/2023), remover
%       as linhas "oldparindent" e carregar a package com a opção "indent".
\newlength\oldparindent
\setlength\oldparindent\parindent
\usepackage[parfill]{parskip}
\setlength\parindent\oldparindent


%%%%%%%%%%%%%%%%%%%%%%%%%%%%%%%%%%%%%%%%%%%%%%%%%%%%%%%%%%%%%%%%%%%%%%%%%%%%%%%%
%%%%%%%%%%%%%%%%%%%%%%%%%% EPÍGRAFE E NOTAS DE RODAPÉ %%%%%%%%%%%%%%%%%%%%%%%%%%
%%%%%%%%%%%%%%%%%%%%%%%%%%%%%%%%%%%%%%%%%%%%%%%%%%%%%%%%%%%%%%%%%%%%%%%%%%%%%%%%

% O formato padrão do pacote epigraph é bem feinho...
% Outra opção para epígrafes é o pacote quotchap
\usepackage{epigraph}

\setlength\epigraphwidth{.85\textwidth}

% Sem linha entre o texto e o autor
\setlength{\epigraphrule}{0pt}

% Ambiente auxiliar para colocar margem à direita da epígrafe
% (como sempre, o modo mais simples de mudar as margens de um
% pagrágrafo é fazer de conta que é uma lista)
\newenvironment{epShiftLeft}
  {
    \par\begin{list}{}
      {
        \leftmargin 0pt
        \labelwidth 0pt
        \labelsep 0pt
        \itemsep 0pt
        \topsep 0pt
        \partopsep 0pt
        \rightmargin 2em
      }
    \item\FlushRight
  }
  {
    \end{list}
    % O espaço padrão que epigraph coloca entre a citação
    % e o autor é muito pequeno; vamos aumentar um pouco
    \vspace*{.3\baselineskip}
  }

\renewcommand\textflush{epShiftLeft}
\renewcommand\sourceflush{epShiftLeft}

\newcommand{\epigrafe}[2] {%
  \ifstrempty{#2}{
    \epigraph{\itshape #1}{}
  }{
    \epigraph{\itshape #1}{--- #2}
  }
}

% Formato personalizado para as notas de rodapé. Copiado quase
% literalmente do exemplo na documentação das classes-padrão de
% LaTeX2e (texdoc classes). Seria possível fazer algo similar
% usando list{} com um único item usando \@thefnmark como label.

% \footnotesep não é um espaço adicional, mas sim um strut que
% existe no começo de cada nota. É por isso que o valor é "grande"
% (\baselineskip) mas a separação de fato é pequena.
\makeatletter
\renewcommand\@makefntext[1]{%
    \setlength{\footnotesep}{1\baselineskip}%
    \@setpar{%
        \@@par
        \@tempdima = \hsize
        \advance\@tempdima-4pt\relax
        \parshape \@ne 4pt \@tempdima
    }%
    \par
    \parindent 1em\noindent
    \parskip .3\baselineskip
    \hbox to \z@{\hss\@makefnmark\,}#1%
}
\makeatother

% \maketitle redefine as notas de rodapé (\thanks) para usar símbolos
% ao invés de números, mas essa não é a única mudança. \maketitle
% também muda \@makefnmark para que a indicação de nota de rodapé
% não ocupe espaço horizontal (isso é feito com \rlap). Isso é feito
% porque a lista de autores em geral é similar a
% \author{Fulano\thanks{instituição 1}, Ciclano\thanks{instituição 2}}.
% Com essa mudança, a nota aparece acima da vírgula entre os autores.
% Mas isso significa que\maketitle precisa também modificar \@makefntext
% para que esse efeito aconteça apenas na lista de autores e não na
% nota em si. Assim, como criamos um novo formato para as notas de
% rodapé, precisamos mudar o formato em \maketitle também.
\makeatletter
\newcommand\@maketitlemakefntext[1]{%
    \setlength{\footnotesep}{1\baselineskip}%
    \@setpar{%
        \@@par
        \@tempdima = \hsize
        \advance\@tempdima-4pt\relax
        \parshape \@ne 4pt \@tempdima
    }%
    \par
    \parindent 1em\noindent
    \parskip .3\baselineskip
    \hbox to \z@{\hss\@textsuperscript{\normalfont\@thefnmark}\,}#1%
}

\patchcmd\maketitle
  {\long\def\@makefntext}
  {\let\@makefntext\@maketitlemakefntext\long\def\@disabledmakefntext}
  {}{}

\makeatother

%%%%%%%%%%%%%%%%%%%%%%%%%%%%%%%%%%%%%%%%%%%%%%%%%%%%%%%%%%%%%%%%%%%%%%%%%%%%%%%%
%%%%%%%%%%%%%%%%%%%%%%%%%%%%%% ÍNDICE REMISSIVO %%%%%%%%%%%%%%%%%%%%%%%%%%%%%%%%
%%%%%%%%%%%%%%%%%%%%%%%%%%%%%%%%%%%%%%%%%%%%%%%%%%%%%%%%%%%%%%%%%%%%%%%%%%%%%%%%

% Cria índice remissivo. Este pacote precisa ser carregado antes de hyperref.
% A criação do índice remissivo depende de um programa auxiliar, que pode ser
% o "makeindex" (default) ou o xindy. xindy é mais poderoso e lida melhor com
% línguas diferentes e caracteres acentuados, mas o programa não está mais
% sendo mantido e índices criados com xindy não funcionam em conjunto com
% hyperref. Se quiser utilizar xindy mesmo assim, é possível contornar esse
% segundo problema configurando hyperref para *não* gerar hyperlinks no
% índice (mais abaixo) e configurando xindy para que ele gere esses hyperlinks
% por conta própria. Para isso, modifique a chamada ao pacote imakeidx (aqui)
% e altere as opções do pacote hyperref.
\providecommand\theindex{} % evita erros de compilação se a classe não tem index
%\usepackage[xindy]{imakeidx} % usando xindy
\usepackage{imakeidx} % usando makeindex

% Cria o arquivo de configuração para xindy lidar corretamente com hyperlinks.
\begin{filecontents*}{hyperxindy.xdy}
(define-attributes ("emph"))
(markup-locref :open "\hyperpage{" :close "}" :attr "default")
(markup-locref :open "\textbf{\hyperpage{" :close "}}" :attr "textbf")
(markup-locref :open "\textit{\hyperpage{" :close "}}" :attr "textit")
(markup-locref :open "\emph{\hyperpage{" :close "}}" :attr "emph")
\end{filecontents*}

% Cria o arquivo de configuração para makeindex colocar um cabeçalho
% para cada letra do índice.
\begin{filecontents*}{mkidxhead.ist}
headings_flag 1
heading_prefix "{\\bfseries "
heading_suffix "}\\nopagebreak\n"
\end{filecontents*}

% Por padrão, o cabeçalho das páginas do índice é feito em maiúsculas;
% vamos mudar isso e deixar fancyhdr definir a formatação
\indexsetup{
  othercode={\chaptermark{\indexname}},
}

\makeindex[
  noautomatic,
  intoc,
  % Estas opções são usadas por xindy
  % "-C utf8" ou "-M lang/latin/utf8.xdy" são truques para contornar este
  % bug, que existe em outras distribuições tambem:
  % https://bugs.launchpad.net/ubuntu/+source/xindy/+bug/1735439
  % Se "-C utf8" não funcionar, tente "-M lang/latin/utf8.xdy"
  %options=-C utf8 -M hyperxindy.xdy,
  %options=-M lang/latin/utf8.xdy -M hyperxindy.xdy,
  % Estas opções são usadas por makeindex
  options=-s mkidxhead.ist -l -c,
]

\PassOptionsToPackage{
  % hyperref não gera hyperlinks corretos em índices remissivos criados
  % com xindy; assim, é possível desabilitar essa função aqui e gerar os
  % os hyperlinks através da configuração de xindy definida anteriormente.
  % Com makeindex (o default), quem precisa criar os hyperlinks é hyperref.
  %hyperindex=false,
}{hyperref}

%%%%%%%%%%%%%%%%%%%%%%%%%%%%%%%%%%%%%%%%%%%%%%%%%%%%%%%%%%%%%%%%%%%%%%%%%%%%%%%%
%%%%%%%%%%%%%%%%%%%%%%%%%%%%%%% BIBLIOGRAFIA %%%%%%%%%%%%%%%%%%%%%%%%%%%%%%%%%%%
%%%%%%%%%%%%%%%%%%%%%%%%%%%%%%%%%%%%%%%%%%%%%%%%%%%%%%%%%%%%%%%%%%%%%%%%%%%%%%%%

% Tradicionalmente, bibliografias no LaTeX são geradas com uma combinação entre
% LaTeX (muitas vezes usando o pacote natbib) e um programa auxiliar chamado
% bibtex. Nesse esquema, LaTeX e natbib são responsáveis por formatar as
% referências ao longo do texto e a formatação da bibliografia fica por conta
% do programa bibtex. A configuração dessa formatação é feita através de um
% arquivo auxiliar de "estilo", com extensão ".bst". Vários journals etc.
% fornecem o arquivo .bst que corresponde ao formato esperado da bibliografia.
%
% bibtex e natbib funcionam bem e, se você tiver alguma boa razão para usá-los,
% obterá bons resultados. No entanto, bibtex tem dois problemas: não lida
% corretamente com caracteres acentuados (embora, na prática, funcione com
% os caracteres usados em português) e o formato .bst, que define a formatação
% da bibliografia, é complexo e pouco flexível.
%
% Por conta disso, a comunidade está migrando para um novo sistema chamado
% biblatex. No biblatex, as formatações da bibliografia e das citações são
% feitas pelo próprio pacote biblatex, dentro do LaTeX. Assim, é bem mais fácil
% modificar e personalizar o estilo da bibliografia. biblatex usa o mesmo
% formato de arquivo de dados do bibtex (".bib") e, portanto, não é difícil
% migrar de um para o outro. biblatex também usa um programa auxiliar (biber),
% mas não para realizar a formatação da bibliografia. A maior desvantagem de
% biblatex é que ele é significativamente mais lento que bibtex.
%
% Observe que biblatex pode criar bibliografias independentes por capítulo
% ou outras divisões do texto. Normalmente é preciso indicar essas seções
% manualmente, mas as opções "refsection" e "refsegment" fazem biblatex
% identificar cada capítulo/seção/etc como uma nova divisão desse tipo.
% No entanto, refsection e refsegment são incompatíveis com o pacote
% titlesec, mencionado em imeusp-formatting.tex. Se você pretende criar
% bibliografias independentes por seções, há duas soluções: (1) desabilitar
% o pacote titlesec; (2) indicar as seções manualmente.
%
% Algumas dicas de configuração:
% https://tex.stackexchange.com/q/12806
% https://github.com/PaulStanley/biblatex-tutorial/releases

\PassOptionsToPackage{
  natbib=true, % Reconhece a sintaxe de natbib (\citet, \citep)
  hyperref=true, % Ativa o suporte ao pacote hyperref
  % Se um item da bibliografia tem língua definida (com langid), permite
  % hifenizar com base na língua selecionada.
  autolang=hyphen,
  % Inclui, em cada item da bibliografia, links para as páginas onde o
  % item foi citado
  backref=true,
}{biblatex}

% Este arquivo é executado antes de carregarmos biblatex, então precisamos
% adiar a execução deste comando. Não carregamos biblatex neste arquivo
% porque o usuário pode querer modificar o estilo bibliográfico, que é
% definido por um parâmetro na hora da carga da package.
\AtEndPreamble{
  % TODO: remover menção ao bug de atendvi/hyperxmp em 2024
  % Impede que um item da bibliografia seja dividido em duas páginas.
  % À parte a estética, isso contorna este bug, que afeta links na
  % úlima página do trabalho, ou seja, pode afetar a bibliografia
  % (atenddvi pode ser carregada por hyperxmp):
  % https://github.com/ho-tex/atenddvi/issues/1
  \AtBeginBibliography{\interlinepenalty=10000\raggedbottom}
}

\makeatletter
\AtEndPreamble{
  % backrefs só fazem sentido com documentos grandes
  \ifboolexpr{test {\@ifclassloaded{book}} or test {\@ifclassloaded{report}}}
    {}
    {\ExecuteBibliographyOptions{backref=false,}}

  % em apresentações e posters, a bibliografia deve ser o mais compacta possível
  \@ifclassloaded{beamer}
    {\ExecuteBibliographyOptions{maxbibnames=2,maxcitenames=2}}
    {}
}
\makeatother

%%%%%%%%%%%%%%%%%%%%%%%%%%%%%%%%%%%%%%%%%%%%%%%%%%%%%%%%%%%%%%%%%%%%%%%%%%%%%%%%
%%%%%%%%%%%%%%%%%%%%%%%%%% HIPERLINKS E REFERÊNCIAS %%%%%%%%%%%%%%%%%%%%%%%%%%%%
%%%%%%%%%%%%%%%%%%%%%%%%%%%%%%%%%%%%%%%%%%%%%%%%%%%%%%%%%%%%%%%%%%%%%%%%%%%%%%%%

% O comando \ref por padrão mostra apenas o número do elemento a que se
% refere; assim, é preciso escrever "veja a Figura~\ref{grafico}" ou
% "como visto na Seção~\ref{sec:introducao}". Usando o pacote hyperref
% (carregado mais abaixo), esse número é transformado em um hiperlink.
%
% Se você quiser mudar esse comportamento, ative as packages varioref
% e cleveref e também as linhas "labelformat" e "crefname" mais abaixo.
% Nesse caso, você deve escrever apenas "veja a \ref{grafico}" ou
% "como visto na \ref{sec:introducao}" etc. e o nome do elemento será
% gerado automaticamente como hiperlink.
%
% Se, além dessa mudança, você quiser usar os recursos de varioref ou
% cleveref, mantenha as linhas labelformat comentadas e use os comandos
% \vref ou \cref, conforme sua preferência, também sem indicar o nome do
% elemento, que é inserido automaticamente. Vale lembrar que o comando
% \vref de varioref pode causar problemas com hyperref, impedindo a
% geração do PDF final.
%
% ATENÇÃO: varioref, hyperref e cleveref devem ser carregadas nessa ordem!
%\usepackage{varioref}

%\labelformat{figure}{Figura~#1}
%\labelformat{table}{Tabela~#1}
%\labelformat{equation}{Equação~#1}
%% Isto não funciona corretamente com os apêndices; o comando seguinte
%% contorna esse problema
%%\labelformat{chapter}{Capítulo~#1}
%\makeatletter
%\labelformat{chapter}{\@chapapp~#1}
%\makeatother
%\labelformat{section}{Seção~#1}
%\labelformat{subsection}{Seção~#1}
%\labelformat{subsubsection}{Seção~#1}

% Usamos "PassOptions" aqui porque outras packages definem opções para
% hyperref também e chamar a package com opções diretamente gera conflitos.
\PassOptionsToPackage{
  unicode=true,
  plainpages=false,
  pdfpagelabels,
  colorlinks=true,
  %citecolor=black,
  %linkcolor=black,
  %urlcolor=black,
  %filecolor=black,
  citecolor=DarkGreen,
  linkcolor=NavyBlue,
  urlcolor=DarkRed,
  filecolor=green,
  bookmarksopen=true,
  % hyperref não gera hyperlinks corretos em índices remissivos criados com
  % xindy; assim, é possível desabilitar essa função aqui e gerar os
  % hyperlinks através da configuração de xindy definida anteriormente. Com
  % makeindex (o default), quem precisa criar os hyperlinks é hyperref.
  %hyperindex=false,
}{hyperref}

% Cria hiperlinks para capítulos, seções, \ref's, URLs etc.
\usepackage{hyperref}

%\usepackage[nameinlink,noabbrev,capitalise]{cleveref}
%% cleveref não tem tradução para o português
%\crefname{figure}{Figura}{Figuras}
%\crefname{table}{Tabela}{Tabelas}
%\crefname{chapter}{Capítulo}{Capítulos}
%\crefname{section}{Seção}{Seções}
%\crefname{subsection}{Seção}{Seções}
%\crefname{subsubsection}{Seção}{Seções}
%\crefname{appendix}{Apêndice}{Apêndices}
%\crefname{subappendix}{Apêndice}{Apêndices}
%\crefname{subsubappendix}{Apêndice}{Apêndices}
%\crefname{line}{Linha}{Linhas}
%\crefname{subfigure}{Figura}{Figuras}
%\crefname{equation}{Equação}{Equações}
%\crefname{listing}{Código-fonte}{Códigos-fonte}
%\crefname{lstlisting}{Código-fonte}{Códigos-fonte}
%\crefname{lstnumber}{Linha}{Linhas}
%\crefrangelabelformat{chapter}{#3#1#4~a~#5#2#6}
%\crefrangelabelformat{section}{#3#1#4~a~#5#2#6}
%\newcommand{\crefrangeconjunction}{ e }
%\newcommand{\crefpairconjunction}{ e }
%\newcommand{\crefmiddleconjunction}{, }
%\newcommand{\creflastconjunction}{ e }
%\crefmultiformat{type}{first}{second}{middle}{last}
%\crefrangemultiformat{type}{first}{second}{middle}{last}

% ao criar uma referência hyperref para um float, a referência aponta
% para o final do caption do float, o que não é muito bom. Este pacote
% faz a referência apontar para o início do float (é possível personalizar
% também). Esta package é incompatível com a classe beamer (usada para
% criar posters e apresentações), então testamos a compatibilidade antes
% de carregá-la.
\ifboolexpr{
  test {\ifcsdef{figure}} and
  test {\ifcsdef{figure*}} and
  test {\ifcsdef{table}} and
  test {\ifcsdef{table*}}
}{\usepackage[all]{hypcap}}{}

% XMP (eXtensible Metadata Platform) is a mechanism proposed by Adobe for
% embedding document metadata within the document itself. The package
% integrates seamlessly with hyperref and requires virtually no modifications
% to documents that already exploit hyperref's mechanisms for specifying PDF
% metadata.
\usepackage{hyperxmp}

% hyperref detecta url's definidas com \url que começam com "http" e
% "www" e cria links adequados. No entanto, quando a url não começa
% com essas strings (por exemplo, "usp.br"), hyperref considera que
% se trata de um link para um arquivo local. Isto força todas as
% \url's que não tem esquema definido a serem do tipo http.
\hyperbaseurl{http://}

%\nocolorlinks % para impressão em P&B
% Para formatar código-fonte (ex. em Java). listings funciona bem mas
% tem algumas limitações (https://tex.stackexchange.com/a/153915 ).
% Se isso for um problema, a package minted pode oferecer resultados
% (muito) melhores; a desvantagem é que ela depende de um programa
% externo, o pygments (escrito em python).
%
% listings também não tem suporte específico a pseudo-código, mas
% incluímos uma configuração para isso que deve ser suficiente.
% Caso contrário, há diversas packages específicas para a criação
% de pseudocódigo:
%
%  * a mais comum é algorithmicx ("\usepackage{algpseudocode}");
%
%  * algorithm2e é bastante flexível, mas um tanto complexa;
%
%  * clrscode3e foi usada no livro "Introduction to Algorithms",
%    de Cormen, Leiserson, Rivert e Stein;
%
%  * pseudocode foi usada no livro "Combinatorial Algorithms",
%    de Kreher e Stinson;
%
%  * algpseudocodex é uma package relativamente nova similar
%    a algorithmicx/algpseudocode mas com diversas melhorias;
%
%  * pseudo também é relativamente nova; ela funciona de forma
%    um pouco diferente das demais e é bastante customizável.
%
% A diferença entre essas packages e listings/minted é que estas
% últimas "entendem" o código e aplicam a formatação automaticamente,
% enquanto com as packages acima o usuário precisa usar comandos LaTeX
% para definir a formatação.
%
% algorithmicx/algpseudocode, algorithm2e, clrscode3e, pseudocode
% e algpseudocodex usam uma "linguagem" própria baseada em comandos
% LaTeX que pode ser facilmente modificada pelo usuário (ou seja,
% é fácil fazer pseudocódigo em português). Segue um exemplo com
% algpseudocodex, provavelmente a opção mais interessante dentre
% este grupo (note o comando "\While", que imprime automaticamente
% a palavra-chave "while" e ajusta a indentação):
%
% \begin{algorithmic}[1]
%   \Function{Euclid}{$a, b$} \Comment{The g.c.d. of a and b}
%     \State $r\gets a\bmod b$
%     \While{$r\not=0$} \Comment{We have the answer if r is 0}
%       \State $a\gets b$
%       \State $b\gets r$
%       \State $r\gets a\bmod b$
%     \EndWhile
%     \State \textbf{return} $b$ \Comment{The gcd is b}
%   \EndFunction
% \end{algorithmic}
%
% pseudo não usa uma "linguagem" própria desse tipo; ao invés disso,
% ela oferece comandos para a formatação direta de palavras-chave,
% variáveis, indentação etc. Um exemplo ("\\", "\\+" e "\\-" controlam
% as quebras de linha e a indentação):
%
% \begin{pseudo}
%    \kw{Function} \fn{Euclid}(a, b) \ct{The g.c.d. of a and b} \\+
%      $r\gets a\bmod b$ \\
%      \kw{while} $r\not=0$ \ct{We have the answer if r is 0} \\+
%        $a\gets b$ \\
%        $b\gets r$ \\
%        $r\gets a\bmod b$ \\-
%      \kw{end} \\
%      \kw{return} $b$ \ct{The gcd is b} \\-
%    \kw{end}
% \end{pseudo}

\usepackage{listings}
\usepackage{lstautogobble}
% Carrega a "linguagem" pseudocode para listings
\appto{\lstaspectfiles}{,lstpseudocode.sty}
\appto{\lstlanguagefiles}{,lstpseudocode.sty}
% Estes dois são carregados do diretório extras (veja basics.tex)
\lstloadaspects{simulatex,invisibledelims,pseudocode}
\lstloadlanguages{[base]pseudocode,[english]pseudocode,[brazilian]pseudocode}

% O pacote listings não lida bem com acentos! No caso dos caracteres acentuados
% usados em português é possível contornar o problema com a tabela abaixo.
% From https://en.wikibooks.org/wiki/LaTeX/Source_Code_Listings#Encoding_issue
\lstset{literate=
  {á}{{\'a}}1 {é}{{\'e}}1 {í}{{\'i}}1 {ó}{{\'o}}1 {ú}{{\'u}}1
  {Á}{{\'A}}1 {É}{{\'E}}1 {Í}{{\'I}}1 {Ó}{{\'O}}1 {Ú}{{\'U}}1
  {à}{{\`a}}1 {è}{{\`e}}1 {ì}{{\`i}}1 {ò}{{\`o}}1 {ù}{{\`u}}1
  {À}{{\`A}}1 {È}{{\'E}}1 {Ì}{{\`I}}1 {Ò}{{\`O}}1 {Ù}{{\`U}}1
  {ä}{{\"a}}1 {ë}{{\"e}}1 {ï}{{\"i}}1 {ö}{{\"o}}1 {ü}{{\"u}}1
  {Ä}{{\"A}}1 {Ë}{{\"E}}1 {Ï}{{\"I}}1 {Ö}{{\"O}}1 {Ü}{{\"U}}1
  {â}{{\^a}}1 {ê}{{\^e}}1 {î}{{\^i}}1 {ô}{{\^o}}1 {û}{{\^u}}1
  {Â}{{\^A}}1 {Ê}{{\^E}}1 {Î}{{\^I}}1 {Ô}{{\^O}}1 {Û}{{\^U}}1
  {Ã}{{\~A}}1 {ã}{{\~a}}1 {Õ}{{\~O}}1 {õ}{{\~o}}1
  {œ}{{\oe}}1 {Œ}{{\OE}}1 {æ}{{\ae}}1 {Æ}{{\AE}}1 {ß}{{\ss}}1
  {ű}{{\H{u}}}1 {Ű}{{\H{U}}}1 {ő}{{\H{o}}}1 {Ő}{{\H{O}}}1
  {ç}{{\c c}}1 {Ç}{{\c C}}1 {ø}{{\o}}1 {å}{{\r a}}1 {Å}{{\r A}}1
  {€}{{\euro}}1 {£}{{\pounds}}1 {«}{{\guillemotleft}}1
  {»}{{\guillemotright}}1 {ñ}{{\~n}}1 {Ñ}{{\~N}}1 {¿}{{?`}}1
}

% Opções default para o pacote listings
% Ref: http://en.wikibooks.org/wiki/LaTeX/Packages/Listings
\lstset{
  columns=[l]fullflexible,            % do not try to align text with proportional fonts
  basicstyle=\footnotesize\ttfamily,  % the font that is used for the code
  numbers=left,                       % where to put the line-numbers
  numberstyle=\footnotesize\ttfamily, % the font that is used for the line-numbers
  stepnumber=1,                       % the step between two line-numbers. If it's 1 each line will be numbered
  numbersep=20pt,                     % how far the line-numbers are from the code
  autogobble,                         % ignore irrelevant indentation
  commentstyle=\color{Brown}\upshape,
  stringstyle=\color{black},
  identifierstyle=\color{DarkBlue},
  keywordstyle=\color{cyan},
  showspaces=false,                   % show spaces adding particular underscores
  showstringspaces=false,             % underline spaces within strings
  showtabs=false,                     % show tabs within strings adding particular underscores
  %frame=single,                       % adds a frame around the code
  framerule=0.6pt,
  tabsize=2,                          % sets default tabsize to 2 spaces
  captionpos=b,                       % sets the caption-position to bottom
  breaklines=true,                    % sets automatic line breaking
  breakatwhitespace=false,            % sets if automatic breaks should only happen at whitespace
  escapeinside={\%*}{*)},             % if you want to add a comment within your code
  backgroundcolor=\color[rgb]{1.0,1.0,1.0}, % choose the background color.
  rulecolor=\color{darkgray},
  extendedchars=true,
  inputencoding=utf8,
  xleftmargin=30pt,
  xrightmargin=10pt,
  framexleftmargin=25pt,
  framexrightmargin=5pt,
  framesep=5pt,
}

% Um exemplo de estilo personalizado para listings (tabulações maiores)
\lstdefinestyle{wider} {
  tabsize = 4,
  numbersep=15pt,
  xleftmargin=25pt,
  framexleftmargin=20pt,
}

% Outro exemplo de estilo personalizado para listings (sem cores)
\lstdefinestyle{nocolor} {
  commentstyle=\color{darkgray}\upshape,
  stringstyle=\color{black},
  identifierstyle=\color{black},
  keywordstyle=\color{black}\bfseries,
}

% Um exemplo de definição de linguagem para listings (XML)
\lstdefinelanguage{XML}{
  morecomment=[s]{<!--}{-->},
  morecomment=[s]{<!-- }{ -->},
  morecomment=[n]{<!--}{-->},
  morecomment=[n]{<!-- }{ -->},
  morestring=[b]",
  morestring=[s]{>}{<},
  morecomment=[s]{<?}{?>},
  morekeywords={xmlns,version,type}% list your attributes here
}

% Estilo padrão para a "linguagem" pseudocode
\lstdefinestyle{pseudocode}{
  basicstyle=\rmfamily\small,
  commentstyle=\itshape,
  keywordstyle=\bfseries,
  identifierstyle=\itshape,
  % as palavras "function" e "procedure"
  procnamekeystyle=\bfseries\scshape,
  % funções precedidas por function/procedure ou com \func{}
  procnamestyle=\ttfamily,
  specialidentifierstyle=\ttfamily\bfseries,
}
\lstset{defaultdialect=[english]{pseudocode}}

% A package listings tem seu próprio mecanismo para a criação de
% captions, lista de programas etc. Neste modelo não usamos esses
% recursos (veja mais abaixo), mas utilizamos estes nomes:
\addto\extrasbrazil{%
  \gdef\lstlistlistingname{Lista de programas}%
  \gdef\lstlistingname{Programa}%
}
\addto\extrasbrazilian{%
  \gdef\lstlistlistingname{Lista de programas}%
  \gdef\lstlistingname{Programa}%
}
\addto\extrasenglish{%
  \gdef\lstlistlistingname{List of Programs}%
  \gdef\lstlistingname{Program}%
}

% Novo tipo de float para programas, possível graças à package float
% ou floatrow.
% Observe que a lista de floats de cada tipo é criada automaticamente
% pela package float/floatrow, mas precisamos:
%  1. Definir o nome do comando ("\begin{program}")
%  2. Definir o nome do float em cada língua ("Figura X", "Programa X")
%  3. Definir a extensão do arquivo temporário a ser usada. Pode ser
%     qualquer coisa, desde que não haja repetições. Aqui, usamos "lop";
%     lembre-se que LaTeX já usa várias outras, como "lof", "lot" etc.,
%     então seja cuidadoso na escolha!
%  4. Acrescentar os comandos correspondentes em paginas-preliminares.tex

\makeatletter
\@ifpackageloaded{floatrow}
  {
    \ifcsundef{chapter}
        % O novo ambiente se chama "program" ("\begin{program}") e a extensão
        % temporária é "lop"
        {\DeclareNewFloatType{program}{placement=htbp,fileext=lop}}
        {\DeclareNewFloatType{program}{placement=htbp,fileext=lop,within=chapter}}

    % Ajusta ligeiramente o espaçamento do estilo "ruled".
    \DeclareFloatVCode{customrule}{{\kern 0pt\hrule\kern 2.5pt\relax}}
    \floatsetup[program]{style=ruled,precode=customrule}
  }
  {
    % Não temos a package floatrow; vamos assumir que temos a package float.

    % O estilo padrão do novo float a ser criado (veja mais sobre isso na
    % documentação da package float). Para "program" usamos "ruled", mas
    % para outros floats provavelmente é melhor usar o mesmo formato de
    % Figuras e Tables (plain).
    \floatstyle{ruled}

    \ifcsundef{chapter}
        % O novo ambiente se chama "program" ("\begin{program}") e a extensão
        % temporária é "lop"
        {\newfloat{program}{htbp}{lop}}
        {\newfloat{program}{htbp}{lop}[chapter]}

    % Retorna o estilo dos floats para o padrão
    \floatstyle{plain}
  }
\makeatother

\captionsetup*[program]{style=ruled,position=top}

% "Program X / Programa X" e "Lista de programas / List of Programs"
\floatname{program}{\lstlistingname}
\gdef\programlistname{\lstlistlistingname}

% Se um programa é maior que uma página, ele não pode ser inserido em
% um float. Nesse caso, vamos criar o ambiente "programruledcaption",
% que cria a mesma estrutura visual e os mesmos captions que os floats
% do tipo "program", mas sem ser um float. Para isso, vamos usar recursos
% da package framed (a package tcolorbox poderia ter sido usada também).
%
% Observe que "programruledcaption" funciona *apenas* para os floats do
% tipo "program". Se quiser criar algo similar para outro tipo de float,
% você vai precisar criar um novo comando ("myfloatruledcaption")
% copiando os comandos abaixo e modificando-os conforme necessário.
\newsavebox{\programCaptionTextBox}
\usepackage{framed}
\newenvironment{programruledcaption}[2][]{
  % All spacing measurements were adjusted to visually reproduce
  % the float captions
  \setlength\fboxsep{0pt}

  % topsep means space before AND after
  \setlength\topsep{.28\baselineskip plus .3\baselineskip minus 0pt}

  \vspace{.3\baselineskip} % Some extra top space

  % For whatever reason, the framed package actually calls "\captionof"
  % multiple times, messing up the counter. We need to prevent this,
  % so we put the caption in a box once and reuse the box.

  \savebox{\programCaptionTextBox}{%
    \parbox[b]{\textwidth}{%
      \ifstrempty{#1}
        {\captionof{program}[#2]{#2}}%
        {\captionof{program}[#1]{#2}}%
    }
  }

  \def\fullcaption{
    \vspace*{-.325\baselineskip}
    \noindent\usebox{\programCaptionTextBox}%
    \vspace*{-.56\baselineskip}%
    \kern 2pt\hrule\kern 2pt\relax
  }

  \def\FrameCommand{
    \hspace{-.007\textwidth}%
    \CustomFBox
      {\fullcaption}
      {\vspace{.13\baselineskip}}
      {.8pt}{.4pt}{0pt}{0pt}
  }

  \def\FirstFrameCommand{
    \hspace{-.007\textwidth}%
    \CustomFBox
      {\fullcaption}
      {\hfill\textit{cont}\enspace$\longrightarrow$}
      {.8pt}{0pt}{0pt}{0pt}
  }

  \def\MidFrameCommand{
    \hspace{-.007\textwidth}%
    \CustomFBox
      {$\longrightarrow$\enspace\textit{cont}\par\vspace*{.3\baselineskip}}
      {\hfill\textit{cont}\enspace$\longrightarrow$}
      {0pt}{0pt}{0pt}{0pt}
  }

  \def\LastFrameCommand{
    \hspace{-.007\textwidth}%
    \CustomFBox
      {$\longrightarrow$\enspace\textit{cont}\par\vspace*{.3\baselineskip}}
      {\vspace{.13\baselineskip}}
      {0pt}{.4pt}{0pt}{0pt}
  }

  \MakeFramed{\FrameRestore}

}{
  \endMakeFramed
}

%%%%%%%%%%%%%%%%%%%%%%%%%%%%%%%%%%%%%%%%%%%%%%%%%%%%%%%%%%%%%%%%%%%%%%%%%%%%%%%%
%%%%%%%%%%%%%%%%%%%%%%%%%%%% OUTROS PACOTES ÚTEIS %%%%%%%%%%%%%%%%%%%%%%%%%%%%%%
%%%%%%%%%%%%%%%%%%%%%%%%%%%%%%%%%%%%%%%%%%%%%%%%%%%%%%%%%%%%%%%%%%%%%%%%%%%%%%%%

% Você provavelmente vai querer ler a documentação de alguns destes pacotes
% para personalizar algum aspecto do trabalho ou usar algum recurso específico.

% melhorias e recursos adicionais para o modo matemático; leia a documentação
\usepackage{mathtools}

% Permite mostrar itens "cancelados" em fórmulas matemáticas, como:
% 2a = 2(b+1)
% \cancel{2}a = \cancel{2}(b+1)
% a = b+1
\usepackage{cancel}

% A classe Book inclui o comando \appendix, que (obviamente) permite inserir
% apêndices no documento. No entanto, não há suporte similar para anexos. Esta
% package acrescenta alguns recursos adicionais para apêndices; vamos usá-la
% para permitir colocar a palavra "Apêndice" no sumário e também para definir
% o comando \annex.
\usepackage{appendix}
\noappendicestocpagenum

\makeatletter

% Altera a formatação da palavra "Apêndice" no sumário
%
% Não queremos a linha "Apêndice" como a última da página no sumário;
% para isso:
%
% 1. Acrescentamos um pouco de espaço elástico logo antes dela
%
% 2. Colocamos uma sugestão de quebra de página
%
% 3. Usamos \@afterheading
%
% 1 e 2 incentivam (mas não forçam) LaTeX a realizar a quebra antes
% dela e 3 força LaTeX a mantê-la na mesma página que a próxima linha.
%
% \addtocontents pode pregar peças se usamos \include; contornamos
% com \immediate: https://tex.stackexchange.com/a/13926
\renewcommand\addappheadtotoc{%
  \begingroup
    \let\origwrite\write
    \def\write{\immediate\origwrite}%
    \addtocontents{toc}{{\large\vspace{0pt plus 2\baselineskip minus 0pt}}}%
    \addtocontents{toc}{\protect\pagebreak[2]}%
    \addtocontents{toc}{\vspace{.8\baselineskip}}%
    \addtocontents{toc}{{\large\bfseries\hspace{-1.3em}\appendixtocname\par}}%
    \addtocontents{toc}{\protect\@afterheading}%
  \endgroup
}

\let\@IMErealAppendixname\appendixname
\let\@IMErealAppendixtocname\appendixtocname
\let\@IMErealAppendixpagename\appendixpagename
\def\appendixname{\@IMErealAppendixname}
\def\appendixtocname{\@IMErealAppendixtocname}
\def\appendixpagename{\@IMErealAppendixpagename}

\providecommand\annexname{Annex}
\providecommand\annextocname{Annexes}
\providecommand\annexpagename{Annexes}

\addto\captionsbrazil{%
  \renewcommand\annexname{Anexo}%
  \renewcommand\annextocname{Anexos}%
  \renewcommand\annexpagename{Anexos}%
  \renewcommand\@IMErealAppendixname{Apêndice}%
  \renewcommand\@IMErealAppendixtocname{Apêndices}%
  \renewcommand\@IMErealAppendixpagename{Apêndices}%
}

\addto\captionsbrazilian{%
  \renewcommand\annexname{Anexo}%
  \renewcommand\annextocname{Anexos}%
  \renewcommand\annexpagename{Anexos}%
  \renewcommand\@IMErealAppendixname{Apêndice}%
  \renewcommand\@IMErealAppendixtocname{Apêndices}%
  \renewcommand\@IMErealAppendixpagename{Apêndices}%
}

\addto\captionsenglish{%
  \renewcommand\annexname{Annex}%
  \renewcommand\annextocname{Annexes}%
  \renewcommand\annexpagename{Annexes}%
  \renewcommand\@IMErealAppendixname{Appendix}%
  \renewcommand\@IMErealAppendixtocname{Appendixes}%
  \renewcommand\@IMErealAppendixpagename{Appendixes}%
}

\let\@IMEorigAppendix\appendix
\renewcommand\appendix{%
    \def\appendixname{\@IMErealAppendixname}
    \def\appendixtocname{\@IMErealAppendixtocname}
    \def\appendixpagename{\@IMErealAppendixpagename}
    \def\Hy@appendixstring{appendix}%
    \@IMEorigAppendix
}

\newcommand\annex{%
    \def\Hy@appendixstring{annex}
    \def\appendixname{\annexname}
    \def\appendixtocname{\annextocname}
    \def\appendixpagename{\annexpagename}
    \@IMEorigAppendix
}

\makeatother

% Para inserir separações no texto que não correspondem a seções com um nome
% definido, é comum usar um ornamento ou florão (em inglês e francês: fleuron).
% Esta package define o comando \froufrou que insere um florão desse tipo.
\usepackage{froufrou} % carregado do diretório extras (veja basics.tex)

% Formatação personalizada das listas "itemize", "enumerate" e
% "description", além de permitir criar novos tipos de listas.
% Com a opção "inline", a package define os novos ambientes "itemize*",
% "description*" e "enumerate*", que fazem os itens da lista como
% parte de um único parágrafo. Como ela causa problemas com
% beamer, apenas a carregamos se não estivermos usando beamer.
\makeatletter
\@ifclassloaded{beamer}
  {}
  {\usepackage[inline]{enumitem}}
\makeatother

% Sublinhado e outras formas de realce de texto
\usepackage{soul}
\usepackage{soulutf8}

% Melhorias e personalização do sublinhado com soul (comando \ul)

% Distância e largura do sublinhado
\setul{1.4pt}{.5pt}

% btul -> "Better Underline" (https://alexwlchan.net/2017/10/latex-underlines/ )
% Sublinhado sem cruzar as linhas descendentes dos caracteres
\usepackage[outline]{contour}
\contourlength{1.1pt}
\newcommand{\btul}[2][white]{%
  \contourlength{1.1pt}%
  \setul{1.4pt}{.5pt}%
  \ul{{\phantom{#2}}}% Faz o sublinhado; precisa das chaves adicionais!
  \llap{\contour{#1}{#2}}% Escreve o texto com fundo branco/colorido
}

% TODO: siunitx removed option "binary-units" in 2021;
%       we can remove this option here after, say, 2025.
% Vários recursos para apresentação de números e grandezas (unidades, notação
% científica, melhor apresentação de números longos etc.), além de permitir
% alinhar números em tabelas pelo ponto decimal (como a package dcolumn)
% através do tipo de coluna "S". Por exemplo, \si{\ohm}, \si{\celsius},
% \si{\milli\litre} (apenas as unidades) ou \SI{10}{\hertz} (a grandeza e a
% unidade), ou \num[round-mode=places,round-precision=4]{3.1415926} -> 3,1416.

\usepackage[binary-units]{siunitx}
\sisetup{
  mode=text,
  round-mode=places,
}

% siunitx usa a package translator para apresentar estas expressões na
% língua do documento; vamos fornecer as traduções para o português.
\providetranslation[to=Portuguese]{to (numerical range)}{a}
\providetranslation[to=Portuguese]{and}{e}
\addto\extrasbrazil{\sisetup{output-decimal-marker = {,}}}
\addto\extrasbrazilian{\sisetup{output-decimal-marker = {,}}}

% Citações melhores; se você pretende fazer citações de textos
% relativamente extensos, vale a pena ler a documentação. biblatex
% utiliza recursos deste pacote.
\usepackage{csquotes}

\usepackage{url}
% URL com fonte sem serifa ao invés de teletype
\urlstyle{sf}

% Permite inserir comentários, muito bom durante a escrita do texto;
% você também pode se interessar pela package pdfcomment.
\usepackage[textsize=scriptsize,colorinlistoftodos,textwidth=2.5cm]{todonotes}
\presetkeys{todonotes}{color=orange!40!white}{}

% Comando para fazer notas com highlight no texto correspondente:
% \hltodo[texto][opções]{comentário}
\makeatletter
\if@todonotes@disabled
  \NewDocumentCommand{\hltodo}{O{} O{} +m}{#1}
\else
  \NewDocumentCommand{\hltodo}{O{} O{} +m}{
    \ifstrempty{#1}{}{\texthl{#1}}%
    \todo[#2]{#3}{}%
  }
\fi
\makeatother

% Vamos reduzir o espaçamento entre linhas nas notas/comentários
\makeatletter
\xpatchcmd{\@todo}
  {\renewcommand{\@todonotes@text}{#2}}
  {\renewcommand{\@todonotes@text}{\begin{spacing}{0.5}#2\end{spacing}}}
  {}
  {}
\makeatother

% Outras ferramentas que podem ser úteis durante a preparação do texto:

% Faz LaTeX mostrar um traço ao lado de linhas "overfull".
%\overfullrule=1mm

% Faz LaTeX mostrar labels e referências bibliográficas:
%\usepackage{showkeys}

% Faz LaTeX mostrar linhas e traços indicando espaçamento, kerning etc.
% Funciona apenas com lualatex.
%\usepackage{lua-visual-debug}

% Além disso, o programa checkcites, instalado juntamente com LaTeX,
% indica problemas com citações bibliográficas.

% Símbolos adicionais: \degree, \celsius, \ohm, \micro, \perthousand.
% Provavelmente é melhor usar os recursos da package siunitx.
%\usepackage{gensymb}

% Permite criar listas como glossários, listas de abreviaturas etc.
% https://en.wikibooks.org/wiki/LaTeX/Glossary
%\usepackage{glossaries}

% Permite formatar texto em colunas
\usepackage{multicol}

% Gantt charts; útil para fazer o cronograma para o exame de
% qualificação, por exemplo.
\usepackage{pgfgantt}

% Estes parâmetros definem a aparência das gantt charts e variam
% em função da fonte do documento.
\ganttset{
    vgrid,
    x unit=1.7em,
    y unit title=3ex,
    y unit chart=4ex,
    % O "strut" é necessário para alinhar o baseline dos nomes dos meses
    title label font=\strut\footnotesize,
    group label font=\footnotesize\bfseries,
    bar label font=\footnotesize,
    milestone label font=\footnotesize\itshape,
    % "align=right" é necessário para \ganttalignnewline funcionar
    group label node/.append style={align=right},
    bar label node/.append style={align=right},
    milestone label node/.append style={align=right},
    group incomplete/.append style={fill=black!50},
    bar/.append style={fill=black!25,draw=black},
    bar incomplete/.append style={fill=white,draw=black},
    % Não é preciso imprimir "0%"
    progress label text=\ifnumequal{#1}{0}{}{(#1\%)},
    % Formato e tamanho dos elementos
    title height=.9,
    group top shift=.4,
    group left shift=0,
    group right shift=0,
    group peaks tip position=0,
    group peaks width=.2,
    group peaks height=.3,
    milestone height=.4,
    milestone top shift=.4,
    milestone left shift=.8,
    milestone right shift=.2,
}

% Em inglês, tanto o nome completo quanto a abreviação do mês de maio
% são "May"; por conta disso, na tradução em português LaTeX erra a
% abreviação. Como talvez usemos o nome inteiro do mês em outro lugar,
% ao invés de forçar a tradução para "Mai" globalmente, fazemos isso
% apenas em ganttchart.
\AtBeginEnvironment{ganttchart}{\deftranslation[to=Portuguese]{May}{Mai}}

% Ilustrações, diagramas, gráficos etc. criados diretamente em LaTeX.
% Também é útil se você quiser importar gráficos gerados com GnuPlot.
\usepackage{tikz}

% Gráficos gerados diretamente em LaTeX; é possível usar tikz para
% isso também.
\usepackage{pgfplots}
% sobre níveis de compatibilidade do pgfplots, veja
% https://tex.stackexchange.com/a/81912
%\pgfplotsset{compat=1.14} % TeXLive 2016
%\pgfplotsset{compat=1.15} % TeXLive 2017
%\pgfplotsset{compat=1.16} % TeXLive 2019
\pgfplotsset{compat=newest}

% Importação direta de arquivos gerados por gnuplot com o
% driver/terminal "lua tikz"; esta package não faz parte da
% instalação padrão do LaTeX, mas sim do gnuplot.
%\usepackage{gnuplot-lua-tikz}

% O formato pdf permite anexar arquivos ao documento, que aparecem
% na página como ícones "clicáveis"; esta package implementa esse
% recurso em LaTeX.
%\usepackage{attachfile}

% Notas de rodapé "órfãs", ou seja, textos que aparecem junto
% das notas de rodapé mas que não têm referência em nenhum lugar.
% "0" desabilita o marcador porque não existe o 0-ésimo símbolo.
\newcommand\detachedfootnote[1]{%
    \bgroup
    \renewcommand\thefootnote{\fnsymbol{footnote}}%
    \renewcommand\thempfootnote{\fnsymbol{mpfootnote}}%
    \footnotetext[0]{#1}%
    \egroup
}

% Os comandos \TeX e \LaTeX são nativos do LaTeX; esta package acrescenta
% comandos para vários outros logos da família TeX. Você provavelmente não
% precisa desse recurso e, portanto, pode removê-la.
\usepackage{hologo}
\providecommand{\XeLaTeX}{\hologo{XeLaTeX}}


% Diretórios onde estão as figuras; com isso, não é preciso colocar o caminho
% completo em \includegraphics (e nem a extensão).
\graphicspath{{figuras/},{logos/}}

% Comandos rápidos para mudar de língua:
% \en -> muda para o inglês
% \br -> muda para o português
% \texten{blah} -> o texto "blah" é em inglês
% \textbr{blah} -> o texto "blah" é em português
\babeltags{br = brazilian, en = english}

% Bibliografia
\usepackage[
  style=extras/plainnat-ime, % variante de autor-data, similar a plainnat
  %style=alphabetic, % similar a alpha
  %style=numeric, % comum em artigos
  %style=authoryear-comp, % autor-data "padrão" do biblatex
  %style=apa, % variante de autor-data, muito usado
  %style=abnt,
]{biblatex}


%%%%%%%%%%%%%%%%%%%%%%%%%%%%%%%%%%%%%%%%%%%%%%%%%%%%%%%%%%%%%%%%%%%%%%%%%%%%%%%%
%%%%%%%%%%%%%%%%%%%%%%%%%%%%%%%%%% METADADOS %%%%%%%%%%%%%%%%%%%%%%%%%%%%%%%%%%%
%%%%%%%%%%%%%%%%%%%%%%%%%%%%%%%%%%%%%%%%%%%%%%%%%%%%%%%%%%%%%%%%%%%%%%%%%%%%%%%%

% O arquivo com os dados bibliográficos para biblatex; você pode usar
% este comando mais de uma vez para acrescentar múltiplos arquivos
\addbibresource{bibliografia.bib}

% Este comando permite acrescentar itens à lista de referências sem incluir
% uma referência de fato no texto (pode ser usado em qualquer lugar do texto)
%\nocite{bronevetsky02,schmidt03:MSc, FSF:GNU-GPL, CORBA:spec, MenaChalco08}
% Com este comando, todos os itens do arquivo .bib são incluídos na lista
% de referências
%\nocite{*}

% É possível definir como determinadas palavras podem (ou não) ser
% hifenizadas; no entanto, a hifenização automática geralmente funciona bem
\babelhyphenation{documentclass latexmk soft-ware clsguide} % todas as línguas
\babelhyphenation[brazilian]{Fu-la-no}
\babelhyphenation[english]{what-ever}

% Estes comandos definem o título e autoria do trabalho e devem sempre ser
% definidos, pois além de serem utilizados para criar a capa, também são
% armazenados nos metadados do PDF.
\title{
    % Obrigatório nas duas línguas
    titlept={Título do trabalho},
    titleen={Title of the document},
    % Opcional, mas se houver deve existir nas duas línguas
    subtitlept={um subtítulo},
    subtitleen={a subtitle},
}

\author[fem]{Nome Completo}

% Para TCCs, este comando define o supervisor
\orientador[fem]{Profª. Drª. Fulana de Tal}

% Se não houver, remova; se houver mais de um, basta
% repetir o comando quantas vezes forem necessárias
\coorientador{Prof. Dr. Ciclano de Tal}
\coorientador[fem]{Profª. Drª. Beltrana de Tal}

% A página de rosto da versão para depósito (ou seja, a versão final
% antes da defesa) deve ser diferente da página de rosto da versão
% definitiva (ou seja, a versão final após a incorporação das sugestões
% da banca).
\defesa{
  nivel=mestrado, % mestrado, doutorado ou tcc
  % É a versão para defesa ou a versão definitiva?
  %definitiva,
  % É qualificação?
  %quali,
  programa={Ciência da Computação},
  membrobanca={Profª. Drª. Fulana de Tal (orientadora) -- IME-USP [sem ponto final]},
  % Em inglês, não há o "ª"
  %membrobanca{Prof. Dr. Fulana de Tal (advisor) -- IME-USP [sem ponto final]},
  membrobanca={Prof. Dr. Ciclano de Tal -- IME-USP [sem ponto final]},
  membrobanca={Profª. Drª. Convidada de Tal -- IMPA [sem ponto final]},
  % Se não houve bolsa, remova
  %
  % Norma sobre agradecimento por auxílios da FAPESP:
  % https://fapesp.br/11789/referencia-ao-apoio-da-fapesp-em-todas-as-formas-de-divulgacao
  %
  % Norma sobre agradecimento por auxílios da CAPES (Portaria 206,
  % de 4 de Setembro de 2018):
  % https://www.in.gov.br/materia/-/asset_publisher/Kujrw0TZC2Mb/content/id/39729251/do1-2018-09-05-portaria-n-206-de-4-de-setembro-de-2018-39729135
  %
  %apoio={O presente trabalho foi realizado com apoio da Coordenação
  %       de Aperfeiçoamento\\ de Pessoal de Nível Superior -- Brasil
  %       (CAPES) -- Código de Financiamento 001}, % o código é sempre 001
  %
  %apoio={This study was financed in part by the Coordenação de
  %       Aperfeiçoamento\\ de Pessoal de Nível Superior -- Brasil
  %       (CAPES) -- Finance Code 001}, % o código é sempre 001
  %
  %apoio={Durante o desenvolvimento deste trabalho, o autor recebeu\\
  %       auxílio financeiro da FAPESP -- processo nº aaaa/nnnnn-d},
  %
  %apoio={During the development if this work, the author received\\
  %       financial support from FAPESP -- grant \#aaaa/nnnnn-d},
  %
  apoio={Durante o desenvolvimento deste trabalho o autor
         recebeu auxílio financeiro da XXXX},
  local={São Paulo},
  data=2017-08-10, % YYYY-MM-DD
  % A licença do seu trabalho. Use CC-BY, CC-BY-NC, CC-BY-ND, CC-BY-SA,
  % CC-BY-NC-SA ou CC-BY-NC-ND para escolher a licença Creative Commons
  % correspondente (o sistema insere automaticamente o texto da licença).
  % Se quiser estabelecer regras diferentes para o uso de seu trabalho,
  % converse com seu orientador e coloque o texto da licença aqui, mas
  % observe que apenas TCCs sob alguma licença Creative Commons serão
  % acrescentados ao BDTA. Se você tem alguma intenção de publicar o
  % trabalho comercialmente no futuro, sugerimos a licença CC-BY-NC-ND.
  direitos={CC-BY}, % Creative Commons Attribution 4.0 International License
  %direitos={Autorizo a reprodução e divulgação total ou parcial
  %          deste trabalho, por qualquer meio convencional ou
  %          eletrônico, para fins de estudo e pesquisa, desde que
  %          citada a fonte.},
  % Para gerar a ficha catalográfica, acesse https://fc.ime.usp.br/,
  % preencha o formulário e escolha a opção "Gerar Código LaTeX".
  % Basta copiar e colar o resultado aqui.
  fichacatalografica={},
}

%%%%%%%%%%%%%%%%%%%%%%%%%%%%%%%%%%%%%%%%%%%%%%%%%%%%%%%%%%%%%%%%%%%%%%%%%%%%%%%%
%%%%%%%%%%%%%%%%%%%%%%% AQUI COMEÇA O CONTEÚDO DE FATO %%%%%%%%%%%%%%%%%%%%%%%%%
%%%%%%%%%%%%%%%%%%%%%%%%%%%%%%%%%%%%%%%%%%%%%%%%%%%%%%%%%%%%%%%%%%%%%%%%%%%%%%%%

\begin{document}

%%%%%%%%%%%%%%%%%%%%%%%%%%% CAPA E PÁGINAS INICIAIS %%%%%%%%%%%%%%%%%%%%%%%%%%%%

% Aqui começa o conteúdo inicial que aparece antes do capítulo 1, ou seja,
% página de rosto, resumo, sumário etc. O comando frontmatter faz números
% de página aparecem em algarismos romanos ao invés de arábicos e
% desabilita a contagem de capítulos.
\frontmatter

\pagestyle{plain}

\onehalfspacing % Espaçamento 1,5 na capa e páginas iniciais

\maketitle % capa e folha de rosto

%%%%%%%%%%%%%%%% DEDICATÓRIA, AGRADECIMENTOS, RESUMO/ABSTRACT %%%%%%%%%%%%%%%%%%

\begin{dedicatoria}
Esta seção é opcional e fica numa página separada; ela pode ser usada para
uma dedicatória ou epígrafe.
\end{dedicatoria}

% Reinicia o contador de páginas (a próxima página recebe o número "i") para
% que a página da dedicatória não seja contada.
\pagenumbering{roman}

% Agradecimentos:
% Se o candidato não quer fazer agradecimentos, deve simplesmente eliminar
% esta página. A epígrafe, obviamente, é opcional; é possível colocar
% epígrafes em todos os capítulos. O comando "\chapter*" faz esta seção
% não ser incluída no sumário.
\chapter*{Agradecimentos}
\epigrafe{Do. Or do not. There is no try.}{Mestre Yoda}

Texto texto texto texto texto texto texto texto texto texto texto texto texto
texto texto texto texto texto texto texto texto texto texto texto texto texto
texto texto texto texto texto texto texto texto texto texto texto texto texto
texto texto texto texto. Texto opcional.

%!TeX root=../tese.tex
%("dica" para o editor de texto: este arquivo é parte de um documento maior)
% para saber mais: https://tex.stackexchange.com/q/78101

% As palavras-chave são obrigatórias, em português e em inglês, e devem ser
% definidas antes do resumo/abstract. Acrescente quantas forem necessárias.
\palavrachave{Palavra-chave1}
\palavrachave{Palavra-chave2}
\palavrachave{Palavra-chave3}

\keyword{Keyword1}
\keyword{Keyword2}
\keyword{Keyword3}

% O resumo é obrigatório, em português e inglês. Estes comandos também
% geram automaticamente a referência para o próprio documento, conforme
% as normas sugeridas da USP.
\resumo{
Elemento obrigatório, constituído de uma sequência de frases concisas e
objetivas, em forma de texto. Deve apresentar os objetivos, métodos empregados,
resultados e conclusões. O resumo deve ser redigido em parágrafo único, conter
no máximo 500 palavras e ser seguido dos termos representativos do conteúdo do
trabalho (palavras-chave). Deve ser precedido da referência do documento.
Texto texto texto texto texto texto texto texto texto texto texto texto texto
texto texto texto texto texto texto texto texto texto texto texto texto texto
texto texto texto texto texto texto texto texto texto texto texto texto texto
texto texto texto texto texto texto texto texto texto texto texto texto texto
texto texto texto texto texto texto texto texto texto texto texto texto texto
texto texto texto texto texto texto texto texto.
Texto texto texto texto texto texto texto texto texto texto texto texto texto
texto texto texto texto texto texto texto texto texto texto texto texto texto
texto texto texto texto texto texto texto texto texto texto texto texto texto
texto texto texto texto texto texto texto texto texto texto texto texto texto
texto texto.
}

\abstract{
Elemento obrigatório, elaborado com as mesmas características do resumo em
língua portuguesa. De acordo com o Regimento da Pós-Graduação da USP (Artigo
99), deve ser redigido em inglês para fins de divulgação. É uma boa ideia usar
o sítio \url{www.grammarly.com} na preparação de textos em inglês.
Text text text text text text text text text text text text text text text text
text text text text text text text text text text text text text text text text
text text text text text text text text text text text text text text text text
text text text text text text text text text text text text.
Text text text text text text text text text text text text text text text text
text text text text text text text text text text text text text text text text
text text text.
}



%%%%%%%%%%%%%%%%%%%%%%%%%%% LISTAS DE FIGURAS ETC. %%%%%%%%%%%%%%%%%%%%%%%%%%%%%

% Como as listas que se seguem podem não incluir uma quebra de página
% obrigatória, inserimos uma quebra manualmente aqui.
\makeatletter
\if@openright\cleardoublepage\else\clearpage\fi
\makeatother

% Todas as listas são opcionais; Usando "\chapter*" elas não são incluídas
% no sumário. As listas geradas automaticamente também não são incluídas por
% conta das opções "notlot" e "notlof" que usamos para a package tocbibind.

% Normalmente, "\chapter*" faz o novo capítulo iniciar em uma nova página, e as
% listas geradas automaticamente também por padrão ficam em páginas separadas.
% Como cada uma destas listas é muito curta, não faz muito sentido fazer isso
% aqui, então usamos este comando para desabilitar essas quebras de página.
% Se você preferir, comente as linhas com esse comando e des-comente as linhas
% sem ele para criar as listas em páginas separadas. Observe que você também
% pode inserir quebras de página manualmente (com \clearpage, veja o exemplo
% mais abaixo).
\newcommand\disablenewpage[1]{{\let\clearpage\par\let\cleardoublepage\par #1}}

% Nestas listas, é melhor usar "raggedbottom" (veja basics.tex). Colocamos
% a opção correspondente e as listas dentro de um grupo para ativar
% raggedbottom apenas temporariamente.
\bgroup
\raggedbottom

%%%%% Listas criadas manualmente

%\chapter*{Lista de Abreviaturas}
\disablenewpage{\chapter*{Lista de Abreviaturas}}

\begin{tabular}{rl}
   CFT & Transformada contínua de Fourier (\emph{Continuous Fourier Transform})\\
   DFT & Transformada discreta de Fourier (\emph{Discrete Fourier Transform})\\
  EIIP & Potencial de interação elétron-íon (\emph{Electron-Ion Interaction Potentials})\\
  STFT & Transformada de Fourier de tempo reduzido (\emph{Short-Time Fourier Transform})\\
  ABNT & Associação Brasileira de Normas Técnicas\\
   URL & Localizador Uniforme de Recursos (\emph{Uniform Resource Locator})\\
   IME & Instituto de Matemática e Estatística\\
   USP & Universidade de São Paulo
\end{tabular}

%\chapter*{Lista de Símbolos}
\disablenewpage{\chapter*{Lista de Símbolos}}

\begin{tabular}{rl}
  $\omega$ & Frequência angular\\
    $\psi$ & Função de análise \emph{wavelet}\\
    $\Psi$ & Transformada de Fourier de $\psi$\\
\end{tabular}

% Quebra de página manual
\clearpage

%%%%% Listas criadas automaticamente

% Você pode escolher se quer ou não permitir a quebra de página
%\listoffigures
\disablenewpage{\listoffigures}

% Você pode escolher se quer ou não permitir a quebra de página
%\listoftables
\disablenewpage{\listoftables}

% Esta lista é criada "automaticamente" pela package float quando
% definimos o novo tipo de float "program" (em utils.tex)
% Você pode escolher se quer ou não permitir a quebra de página
%\listof{program}{\programlistname}
\disablenewpage{\listof{program}{\programlistname}}

% Sumário (obrigatório)
\tableofcontents

\egroup % Final de "raggedbottom"

% Referências indiretas ("x", veja "y") para o índice remissivo (opcionais,
% pois o índice é opcional). É comum colocar esses itens no final do documento,
% junto com o comando \printindex, mas em alguns casos isso torna necessário
% executar texindy (ou makeindex) mais de uma vez, então colocar aqui é melhor.
\index{Inglês|see{Língua estrangeira}}
\index{Figuras|see{Floats}}
\index{Tabelas|see{Floats}}
\index{Código-fonte|see{Floats}}
\index{Subcaptions|see{Subfiguras}}
\index{Sublegendas|see{Subfiguras}}
\index{Equações|see{Modo matemático}}
\index{Fórmulas|see{Modo matemático}}
\index{Rodapé, notas|see{Notas de rodapé}}
\index{Captions|see{Legendas}}
\index{Versão original|see{Tese/Dissertação, versões}}
\index{Versão corrigida|see{Tese/Dissertação, versões}}
\index{Palavras estrangeiras|see{Língua estrangeira}}
\index{Floats!Algoritmo|see{Floats, ordem}}


%%%%%%%%%%%%%%%%%%%%%%%%%%%%%%%% CAPÍTULOS %%%%%%%%%%%%%%%%%%%%%%%%%%%%%%%%%%%%%

% Aqui vai o conteúdo principal do trabalho, ou seja, os capítulos que compõem
% a dissertação/tese. O comando mainmatter reinicia a contagem de páginas,
% modifica a numeração para números arábicos e ativa a contagem de capítulos.
\mainmatter

\pagestyle{mainmatter}

% Espaçamento simples
\singlespacing

%!TeX root=../tese.tex
%("dica" para o editor de texto: este arquivo é parte de um documento maior)
% para saber mais: https://tex.stackexchange.com/q/78101/183146

%% ------------------------------------------------------------------------- %%

% "\chapter" cria um capítulo com número e o coloca no sumário; "\chapter*"
% cria um capítulo sem número e não o coloca no sumário. A introdução não
% deve ser numerada, mas deve aparecer no sumário. Por conta disso, este
% modelo define o comando "\unnumberedchapter".
\unnumberedchapter{Introdução}
\label{cap:introducao}

Escrever bem é uma arte que exige muita técnica e dedicação e,
consequentemente, há vários bons livros sobre como escrever uma boa
dissertação ou tese. Um dos trabalhos pioneiros e mais conhecidos nesse
sentido é o livro de
%Umberto Eco~\cite{eco:09} % usando o estilo alpha
Umberto~\citet{eco:09} % usando o estilo plainnat
intitulado \emph{Como se faz uma tese}; é uma leitura bem interessante mas,
como foi escrito em 1977 e é voltado para trabalhos de graduação na Itália,
não se aplica tanto a nós.

Sobre a escrita acadêmica em geral, John Carlis disponibilizou um texto curto
e interessante~\citep{carlis:09} em que advoga a preparação de um único
rascunho da tese antes da versão final. Mais importante que isso, no
entanto, são os vários \textit{insights} dele sobre a escrita acadêmica.
Dois outros bons livros sobre o tema são \emph{The Craft of Research}~\citep{craftresearch}
e \emph{The Dissertation Journey}~\citep{dissertjourney}. Além disso, a USP
tem uma compilação de normas relativas à produção de documentos
acadêmicos~\citep{usp:guidelines} que pode ser utilizada como referência.

Para a escrita de textos especificamente sobre Ciência da Computação, o
livro de Justin Zobel, \emph{Writing for Computer Science}~\citep{zobel:04}
é uma leitura obrigatória. O livro \emph{Metodologia de Pesquisa para
Ciência da Computação} de
%Raul Sidnei Wazlawick~\cite{waz:09} % usando o estilo alpha
Raul Sidnei~\citet{waz:09} % usando o estilo plainnat
também merece uma boa lida. Já para a área de Matemática, dois livros
recomendados são o de Nicholas Higham, \emph{Handbook of Writing for
Mathematical Sciences}~\citep{Higham:98} e o do criador do \TeX{}, Donald
Knuth, juntamente com Tracy Larrabee e Paul Roberts, \emph{Mathematical
Writing}~\citep{Knuth:96}.

Apresentar os resultados de forma simples, clara e completa é uma tarefa que
requer inspiração. Nesse sentido, o livro de
%Edward Tufte~\cite{tufte01:visualDisplay}, % usando o estilo alpha
Edward~\citet{tufte01:visualDisplay}, % usando o estilo plainnat
\emph{The Visual Display of Quantitative Information}, serve de ajuda na
criação de figuras que permitam entender e interpretar dados/resultados de forma
eficiente.

Além desse material, também vale muito a pena a leitura do trabalho de
%Uri Alon \cite{alon09:how}, % usando o estilo alpha
Uri \citet{alon09:how}, % usando o estilo plainnat
no qual apresenta-se uma reflexão sobre a utilização da Lei de Pareto para
tentar definir/escolher problemas para as diferentes fases da vida acadêmica.
A direção dos novos passos para a continuidade da vida acadêmica deveria ser
discutida com seu orientador.

%% ------------------------------------------------------------------------- %%
\unnumberedsection{Considerações de Estilo}
\label{sec:consideracoes_preliminares}

Normalmente, as citações não devem fazer parte da estrutura sintática da
frase\footnote{E não se deve abusar das notas de rodapé.\index{Notas de rodapé}}.
No entanto, usando referências em algum estilo autor-data (como o estilo
plainnat do \LaTeX{}), é comum que o nome do autor faça parte da frase. Nesses
casos, pode valer a pena mudar o formato da citação para não repetir o nome do
autor; no \LaTeX{}, isso pode ser feito usando os comandos
\textsf{\textbackslash{}citet}, \textsf{\textbackslash{}citep},
\textsf{\textbackslash{}citeyear} etc. documentados no pacote
natbib \citep{natbib}\index{natbib} (esses comandos são compatíveis com biblatex
usando a opção \textsf{natbib=true}, ativada por padrão neste modelo). Em geral,
portanto, as citações devem seguir estes exemplos:

\footnotesize
\begin{verbatim}
Modos de citação:
indesejável: [AF83] introduziu o algoritmo ótimo.
indesejável: (Andrew e Foster, 1983) introduziram o algoritmo ótimo.
certo: Andrew e Foster introduziram o algoritmo ótimo [AF83].
certo: Andrew e Foster introduziram o algoritmo ótimo (Andrew e Foster, 1983).
certo (\citet ou \citeyear): Andrew e Foster (1983) introduziram o algoritmo ótimo.
\end{verbatim}
\normalsize

O uso desnecessário de termos em língua estrangeira deve ser evitado. No entanto,
quando isso for necessário, os termos devem aparecer \textit{em itálico}.
\index{Língua estrangeira}
% index permite acrescentar um item no indice remissivo

Uma prática recomendável na escrita de textos é descrever as
legendas\index{Legendas} das figuras e tabelas em forma auto-contida: as
legendas devem ser razoavelmente completas, de modo que o leitor possa entender
a figura sem ler o texto onde a figura ou tabela é citada.\index{Floats}

Sugerimos que você faça referências bibliográficas de forma similar aos
estilos ``alpha'' (referências alfanuméricas) ou ``plainnat'' (referências
por autor-data) de \LaTeX{}.  Se estiver usando natbib+bibtex\index{natbib}\index{bibtex},
use os arquivos .bst ``alpha-ime.bst'' ou ``plainnat-ime.bst'', que são
versões desses dois formatos traduzidas para o português. Se estiver usando
biblatex\index{biblatex} (recomendado), escolha o estilo ``alphabetic''
(que é um dos estilos padrão do biblatex) ou ``plainnat-ime''. O arquivo de
exemplo inclui todas essas opções; basta des-comentar as linhas
correspondentes e, se necessário, modificar o arquivo Makefile para chamar
o bibtex\index{bibtex} ao invés do biber\index{biber} (este último é usado
em conjunto com o biblatex).

\unnumberedsection{Ferramentas Bibliográficas}

Embora seja possível pesquisar por material acadêmico na Internet usando sistemas
de busca ``comuns'', existem ferramentas dedicadas, como o \textsf{Google Scholar}\index{Google Scholar}
(\url{scholar.google.com}). Você também pode querer usar o \textsf{Web of Science}\index{Web of Science}
(\url{webofscience.com}) e o \textsf{Scopus}\index{Scopus} (\url{scopus.com}), que oferecem
recursos sofisticados e limitam a busca a periódicos com boa reputação acadêmica.
Essas duas plataformas não são gratuitas, mas os alunos da USP têm acesso a elas
através da instituição. Ambas são capazes de exportar os dados para o formato .bib,
usado pelo \LaTeX{}. Algumas editoras, como a ACM e a IEEE, também têm sistemas de
busca bibliográfica.

Apenas uma parte dos artigos acadêmicos de interesse está disponível livremente
na Internet; os demais são restritos a assinantes. A CAPES assina um grande
volume de publicações e disponibiliza o acesso a elas para diversas universidades
brasileiras, entre elas a USP, através do seu portal de periódicos
(\url{periodicos.capes.gov.br}). Existe uma extensão para os navegadores
Chrome e Firefox (\url{www.infis.ufu.br/capes-periodicos}) que facilita o uso
cotidiano do portal.

Para manter um banco de dados organizado sobre artigos e outras fontes bibliográficas
relevantes para sua pesquisa, é altamente recomendável que você use uma ferramenta
como Zotero~(\url{zotero.org})\index{Zotero} ou
Mendeley~(\url{mendeley.com})\index{Mendeley}. Ambas podem exportar seus dados no
formato .bib, compatível com \LaTeX{}. Também existem três plataformas
gratuitas que permitem a busca de referências acadêmicas já no formato .bib:

\begin{itemize}
  \item \emph{CiteULike}\index{CiteULike} (patrocinados por Springer): \url{www.citeulike.org}
  \item Coleção de bibliografia em Ciência da Computação: \url{liinwww.ira.uka.de/bibliography}
  \item Google acadêmico\index{Google Scholar} (habilitar bibtex nas preferências): \url{scholar.google.com}
\end{itemize}

Lamentavelmente, ainda não existe um mecanismo de verificação ou validação das
informações nessas plataformas. Portanto, é fortemente sugerido validar todas
as informações de tal forma que as entradas bib estejam corretas.

De qualquer modo, tome muito cuidado na padronização das referências
bibliográficas: ou considere TODOS os nomes dos autores por extenso, ou TODOS
os nomes dos autores abreviados.  Evite misturas inapropriadas.

%!TeX root=../tese.tex
%("dica" para o editor de texto: este arquivo é parte de um documento maior)
% para saber mais: https://tex.stackexchange.com/q/78101

\chapter{O que o IME espera (normas)}

Fica a critério do aluno definir os aspectos relacionados à aparência da
tese, como o tamanho de fonte, margens, espaçamento, estilo de referências,
cabeçalho, etc., considerando sempre o bom senso.

A CPG, em reunião realizada em junho de 2007, aprovou que as
teses/dissertações deverão seguir o formato padrão por ela definido.
Esse padrão refere-se aos itens que devem estar presentes nas teses/dissertações
(e.g. capa, formato de rosto, sumário, etc.), e não à formatação do documento.
Ele define itens obrigatórios e opcionais, conforme segue:\index{Formatação}
\index{Tese/Dissertação!itens obrigatórios}
\index{Tese/Dissertação!itens opcionais}

\begin{itemize}
  \item \textsc{Capa} (obrigatória)
  \begin{itemize}
    \item O IME usa uma capa padrão de cartolina para todas as
    teses/dissertações. Essa capa tem uma janela recortada por onde se
    vê o título e o autor do trabalho e, portanto, a capa impressa do
    trabalho deve incluir o título e o autor na posição correspondente da
    página. Ela fica centralizada na página, tem 100mm de largura, 60mm de
    altura e começa 47mm abaixo do topo da página.

    \item O título da tese/dissertação deverá começar com letra maiúscula
    e o resto deverá ser em minúsculas, salvo nomes próprios.

    \item O nome do aluno(a) deverá ser completo e sem abreviaturas.

    \item É preciso explicitar se é uma tese ou dissertação (para
    obtenção do título de doutor, tese; para obtenção do título de
    mestre, dissertação).

    \item O nome do programa deve constar da capa (Matemática,
    Matemática Aplicada, Estatística, Ciência da Computação ou
    Mestrado Profissional em Ensino de Matemática).

    \item Também devem constar o nome completo do orientador e do
    co-orientador, se houver.

    \item Se o aluno recebeu bolsa, deve-se indicar a(s) agência(s).

    \item É preciso informar o mês e ano do depósito ou da entrega da
    versão corrigida.
  \end{itemize}

  \newpage % Às vezes, o uso da força é inevitável ;-)

  \item \textsc{Folha de rosto} (obrigatória, tanto para a versão
  depositada quanto para a versão corrigida)
  \begin{itemize}
    \item O título da tese/dissertação deverá seguir o padrão da capa.

    \item Deve informar se se trata da versão original ou da versão
    corrigida (veja mais sobre isso abaixo); no segundo caso, deve
    também incluir os nomes dos membros da banca.
  \end{itemize}

  \item \textsc{Agradecimentos} (opcional)

  \item \textsc{Resumo}, em português (obrigatório)

  \item \textsc{Abstract}, em inglês (obrigatório)

  \item \textsc{Sumário} (obrigatório)

  \item \textsc{Listas} (opcionais)
  \begin{itemize}
    \item Lista de Abreviaturas
    \item Lista de Símbolos
    \item Lista de Figuras
    \item Lista de Tabelas
  \end{itemize}

  \item \textsc{Referências} (obrigatório)

  \item \textsc{Índice Remissivo} (opcional\footnote{O índice remissivo
   pode ser muito útil para a banca; assim, embora seja um item opcional,
   recomendamos que você o crie.})
\end{itemize}

Ao terminar sua tese/dissertação, você deve entregar uma cópia (digital) dela
para a CPG. Após a defesa, você tem 30 dias para revisar o texto e incorporar
as sugestões da banca. Assim, há duas versões oficiais do documento: a versão
original e a versão corrigida, o que deve ser indicado na folha de rosto.
\index{Tese/Dissertação!versões}


%!TeX root=../tese.tex
%("dica" para o editor de texto: este arquivo é parte de um documento maior)
% para saber mais: https://tex.stackexchange.com/q/78101

\chapter{Usando este modelo}

Não é necessário que o texto seja redigido usando \LaTeX{} ou este modelo,
mas seu uso é fortemente recomendado, pois ele facilita diversas etapas do
trabalho e o resultado final é muito bom\footnote{O uso de um sistema de
controle de versões, como mercurial (\url{mercurial-scm.org}) ou git
(\url{git-scm.com}), também é altamente recomendado.}. Este modelo é
distribuído com uma ``colinha'' dos principais comandos \LaTeX{} e inclui
comentários explicativos para auxiliá-lo com ele, sendo composto dos
arquivos:

\begin{itemize}
  \item \texttt{tese.tex}, \texttt{artigo.tex}, \texttt{apresentacao.tex}
        e \texttt{poster.tex} (exemplos de cada um desses tipos de documento);
  \item Capítulos, apêndices, imagens etc. deste texto de exemplo, nos
        diretórios \texttt{conteudo}, \texttt{figuras} e \texttt{logos}
        (procure os comandos \texttt{\textbackslash{}input} e
        \texttt{\textbackslash{}graphicspath} nos arquivos de exemplo
        mencionados acima para modificar o nome desses diretórios);
  \item \texttt{bibliografia.bib} (exemplo de banco de dados bibliográficos;
        procure o comando \texttt{\textbackslash{}addbibresource} nos
        arquivos de exemplo mencionados acima para modificar o nome desse
        arquivo ou acrescentar outros);
  \item Arquivos com as configurações e \textit{packages} usadas, no
        diretório \texttt{extras} (você só precisa mexer neles se quiser
        aprender mais sobre \LaTeX{} ou modificar/acrescentar algo ao
        modelo).
\end{itemize}

Para compilar o documento, basta executar o comando
\textsf{latexmk}\footnote{Você também pode usar \textsf{latexmk poster},
\textsf{latexmk apresentacao} etc.}. Talvez seu editor ofereça uma opção
de menu para compilar o documento; sempre que possível, configure-o para
utilizar o \textsf{latexmk} ao selecioná-la. \LaTeX{} gera diversos arquivos
auxiliares durante a compilação que, em algumas raras situações, podem ficar
inconsistentes (causando erros de compilação ou erros no \textsc{pdf} gerado,
como referências faltando ou numeração de páginas incorreta no sumário).
Nesse caso, é só usar o comando \textsf{latexmk -C}, que apaga todos esses
arquivos auxiliares gerados, e em seguida rodar \textsf{latexmk} novamente.

Você pode mudar a língua do documento para o inglês no início de cada
arquivo .tex de exemplo, na linha \textsf{\textbackslash{}documentclass}.
No caso do arquivo \textsf{tese.tex}, isso muda todos os textos padrão
da capa e folhas de rosto.

Os arquivos deste modelo, incluindo os do diretório \texttt{extras},
incluem vários comentários com dicas e explicações; se o que você precisa
não está mencionado diretamente, é provável que haja pelo menos a indicação
da \textit{package} relacionada ao que você precisa.

Se você encontrar algum problema com o modelo, ajude a melhorá-lo!
Envie um relatório de erro ou entre em contato em
\url{gitlab.com/ccsl-usp/modelo-latex}.

%!TeX root=../tese.tex
%("dica" para o editor de texto: este arquivo é parte de um documento maior)
% para saber mais: https://tex.stackexchange.com/q/78101

\chapter{Instalação do \LaTeX{}}
\label{chap:install}

\LaTeX{} é, na verdade, um conjunto de programas. Ao invés de procurar e
baixar cada um deles, o mais comum é baixar uma coleção com todos eles juntos.
Há duas coleções desse tipo disponíveis: MiK\TeX{} (\url{miktex.org}) e
\TeX{}Live (\url{www.tug.org/texlive}). Ambos funcionam em Linux, Windows e
macOS. Em Linux, \TeX{}Live costuma estar disponível para instalação junto
com os demais opcionais do sistema. Em macOS, o mais popular é o Mac\TeX{}
(\url{www.tug.org/mactex/}), a versão do \TeX{}Live para macOS. Em Windows,
o mais comumente usado é o MiK\TeX{}.

Por padrão, eles não instalam tudo que está disponível, mas sim apenas os
componentes mais usados, e oferecem um gestor de pacotes que permite adicionar
outros. Embora uma instalação completa do \LaTeX{} seja relativamente grande
(perto de 5GB), em geral vale a pena instalar a maior parte dos componentes.
Se você preferir uma instalação mais ``enxuta'', não deixe de incluir tudo
que é necessário para este modelo, como indicado no arquivo README.md.

Também é muito importante ter o \textsf{latexmk}. No Linux, a instalação
é similar à de outros programas. No macOS e no Windows, \textsf{latexmk}
pode ser instalado pelo gestor de pacotes do MiK\TeX{} ou \TeX{}Live.
Observe que ele depende da linguagem \textsf{perl}. No macOS, \textsf{perl}
já faz parte do sistema; no Windows, \TeX{}Live inclui uma versão básica
de perl, mas se você estiver usando MiK\TeX{} será preciso instalar
\textsf{perl} manualmente (\url{www.perl.org/get.html}).

\enlargethispage{.8\baselineskip}

\section{Documentação sobre \LaTeX}
\label{sec:docs}

Há muito material sobre \LaTeX{} na Internet, mas também há muita informação
obsoleta (incluindo trechos da própria documentação oficial!). Em particular,
você pode ignorar explicações sobre como converter arquivos no formato
\textsc{dvi} gerados por \LaTeX{} em \textsc{pdf}: as versões atualmente
recomendadas de \LaTeX{} (cf. Seção~\ref{sec:versions}) geram arquivos
\textsc{pdf} diretamente. Quanto a imagens, os formatos de arquivo
\textsc{ps/eps} (PostScript e Encapsulated PostScript) não são adequados
para essas novas versões de \LaTeX{}; elas trabalham com arquivos de imagem
nos formatos \textsc{pdf}, \textsc{png} e \textsc{jpeg}. Finalmente,
recursos gráficos normalmente não usam mais \textit{packages} como
\textsf{pstricks}, \textsf{eepic} ou outras tradicionalmente citadas;
ao invés disso, \textsf{PGF/TikZ} é a ferramenta mais comum. Finalmente,
\LaTeX{} usa \textsf{utf8} por padrão desde 2019, tornando a \textit{package}
\textsf{inputenc} necessária apenas para documentos legados.

Um possível caminho para o aprendizado é começar com o
Capítulo~\ref{chap:tutorial} deste modelo e o conteúdo em
\url{overleaf.com/learn}, que tem escopo similar mas também inclui
várias páginas sobre como utilizar recursos específicos, ou ainda
o sítio \url{https://www.learnlatex.org/pt/}, em português. Após esse contato
inicial, o tutorial em \url{tug.org/twg/mactex/tutorials/ltxprimer-1.0.pdf}
é bastante abrangente e detalhado. Não deixe de ver também o
Capítulo~\ref{chap:exemplos} deste modelo (e seu código-fonte), que
inclui várias dicas úteis. Para os principais comandos do modo matemático,
veja \textsf{texdoc undergradmath} e, para aprender a criar apresentações,
veja \textsf{texdoc beamer}.

Depois que você estiver razoavelmente familiarizado com a linguagem,
utilize o manual de referência que pode ser acessado em \url{latexref.xyz}
ou com \textsf{texdoc latex2e} (disponível também em francês, com
\textsf{texdoc latex2e-fr}, e em espanhol, com \textsf{texdoc latex2e-es}).

A documentação de referência mais importante sobre os recursos matemáticos
é acessível com \textsf{texdoc amsmath}, \textsf{texdoc amsthm} e
\textsf{texdoc mathtools}; \textsf{texdoc maths-symbols} agrega os símbolos
matemáticos disponíveis. Para uma lista completa de todos os símbolos
disponíveis com \LaTeX{}, use \textsf{texdoc symbols-a4} (esse documento
tem mais de 300 páginas!).

Como dito anteriormente, \LaTeX{} é, na verdade, um conjunto de programas
e, em geral, instalamos coleções pré-prontas com todos eles. Essas coleções
(\TeX{}Live e MiK\TeX{}) contêm também a documentação das \textit{packages}
incluídas: Basta digitar \textsf{texdoc nome-da-package} (\TeX{}Live) ou
\textsf{mthelp nome-da-package} (MiK\TeX{}) para ter acesso à documentação
correspondente\footnote{O sítio \url{texdoc.org} também oferece acesso a
esse conteúdo.}. \textsf{texdoc/mthelp} incluem também alguns tutoriais e
textos introdutórios.

Para dúvidas pontuais, o sítio \url{tex.stackexchange.com} é um fórum
de perguntas e respostas sobre \LaTeX{} muito útil, pois os principais
desenvolvedores do sistema participam das discussões, e o sítio
\url{texfaq.org} é bastante abrangente e atualizado.

Existem também diversos bons livros sobre \LaTeX{} (embora em geral um
tanto antigos), dos quais destacamos dois:

\begin{enumerate}

  % https://www.math.ucdavis.edu/~tracy/courses/math129/Guide_To_LaTeX.pdf
  % https://archive.org/details/a-guide-to-la-te-x-and-electronic-publishing/page/n7/mode/2up
  \item A quarta edição de ``A Guide to \LaTeX'', de Helmut Kopka
        e Patrick W. Daly (publicada em 2003), além de uma ótima
        introdução, aborda vários tópicos relativamente avançados
        e úteis\footnote{Uma versão não-final está disponível em
        \url{www2.mps.mpg.de/homes/daly/GTL/gtl_20030512.pdf}.}.
  \item A segunda edição de ``The \LaTeX{} Companion'' (publicada em
        2004) é um livro quase obrigatório, pois discute em detalhes
        praticamente todos os recursos e \textit{packages} importantes
        de \LaTeX{}, servindo tanto para o aprendizado quanto como
        material de referência. A terceira edição, amplamente atualizada,
        foi lançada em 2023.

\end{enumerate}

\froufrou

Existem inúmeras alternativas aos materiais citados acima; outros exemplos de
textos introdutórios são \url{www.maths.tcd.ie/~dwilkins/LaTeXPrimer/GSWLaTeX.pdf}
e \url{www.andy-roberts.net/writing/latex}. Em português, você pode
consultar \url{polignu.org/sites/polignu.org/files/latex/latex-fflch.pdf}
e \url{git.febrace.org.br/material-latex/material-latex} (este precisa ser
baixado e compilado). O canal \url{youtube.com/c/anteroneves} tem vários
vídeos instrutivos em português. \textsf{texdoc/mthelp} incluem ainda opções
como ``The Not So Short Introduction to \LaTeXe{}'' (\textsf{texdoc
lshort-eng}; há uma versão em português, mas não está em dia com o original)
e ``A Simplified Introduction to \LaTeX{}'' (\textsf{texdoc simplified-intro}).
Versões recentes do \LaTeX{} incluem também o ``\LaTeXe{} via exemplos''
(\textsf{texdoc latex-via-exemplos}), em português.

\subsection{Outros recursos (avançados)}

O sítio \url{ctan.org} é o repositório semi-oficial das
\textit{packages} \LaTeX{} e sua documentação; \TeX{}Live e MiK\TeX{} são
construídas a partir do que está nesse site, então a última versão estável de
qualquer \textit{package} (e da documentação acessível com \textsf{texdoc/mthelp})
em geral está ali.

\textsf{texdoc fntguide} explica como funciona a gestão de fontes de
\LaTeX{}, e você pode ver exemplos de fontes disponíveis para \LaTeX{}
em \url{tug.org/FontCatalogue}. Lua\LaTeX{} e \XeLaTeX{} funcionam de
outra maneira, permitindo também o uso das fontes comuns instaladas no
seu sistema operacional (veja \textsf{texdoc fontspec}).

Minúcias sobre o funcionamento interno do sistema estão descritas em
\textsf{texdoc source2e} e, sobre as classes padrão (\textsf{article, book}
etc.), em \textsf{texdoc classes}. Você normalmente não vai usar esses
documentos, mas eles podem servir para esclarecer algum detalhe.
\textsf{texdoc macros2e}, \textsf{texdoc xparse} e \textsf{texdoc
interface3} apresentam a linguagem de programação usada por \LaTeX{},
enquanto \textsf{texdoc clsguide} é um guia para a criação de novas
classes e \textit{packages}.

Quando você se tornar um usuário avançado, pode se interessar em conhecer
melhor a linguagem \TeX{}, que está na base do \LaTeX{}. ``The \TeX{} book'',
de Donald Knuth (o criador do \TeX), é amplamente recomendado, mas há três
livros completos a respeito que são instalados com \LaTeX{}: ``A gentle
introduction to \TeX{}'' (\textsf{texdoc gentle}), ``\TeX{} for the
impatient'' (\textsf{texdoc impatient}) e ``\TeX{} by topic'' (\textsf{texdoc
texbytopic}).

%!TeX root=../tese.tex
%("dica" para o editor de texto: este arquivo é parte de um documento maior)
% para saber mais: https://tex.stackexchange.com/q/78101

% Vamos definir alguns comandos auxiliares para facilitar.

% "textbackslash" é muito comprido.
\newcommand{\sla}{\textbackslash}

% Vamos escrever comandos (como "make" ou "itemize") com formatação especial.
\newcommand{\cmd}[1]{\textsf{#1}}

% Idem para packages; aqui estamos usando a mesma formatação de \cmd,
% mas poderíamos escolher outra.
\newcommand{\pkg}[1]{\textsf{#1}}

% A maioria dos comandos LaTeX começa com "\"; vamos criar um
% comando que já coloca essa barra e formata com "\cmd".
\newcommand{\ltxcmd}[1]{\cmd{\sla{}#1}}

\chapter{Do zero ao mínimo com \LaTeX{}}
\label{chap:tutorial}

Neste capítulo, apresentamos uma visão geral sobre \LaTeX{} para quem
nunca trabalhou com ele antes. Se você já tem conhecimento básico ou
intermediário sobre o sistema, sinta-se à vontade para ir diretamente ao
Capítulo~\ref{chap:exemplos}, que inclui diversos exemplos e dicas úteis.
A intenção deste capítulo não é propriamente ensinar a usar \LaTeX{},
mas sim expor seus princípios de funcionamento e principais recursos,
de maneira que o leitor esteja melhor capacitado a compreender outros
documentos e exemplos.

\enlargethispage{-.5\baselineskip}

\section{Por que \LaTeX{}?}

Preparar um texto para impressão envolve duas coisas:

\begin{description}
\item[Escrever:] digitar, recortar/colar trechos, revisar etc.
\item[Formatar:] definir o tamanho da fonte, o
espaçamento entre parágrafos etc.
\end{description}

Hoje é comum fazer essas duas coisas ao mesmo tempo, graças à visualização
imediata que o computador oferece. No entanto, imagine como era o processo de
produção de um livro nos anos 1970: o autor escrevia seu texto em uma máquina
de escrever e enviava esse material para o editor, que era responsável pela
tarefa de formatá-lo para impressão. O autor muitas vezes inseria anotações
para o editor explicando coisas como ``este parágrafo é uma citação'', e o
editor criava algum mecanismo visual para representar isso.

Não é de se surpreender que, com o surgimento do microcomputador, os primeiros
programas para criação de textos seguissem um funcionamento similar: o autor
digitava e editava seu texto sem formatá-lo visualmente, apenas inserindo
alguns comandos correspondentes a aspectos da formatação que ele depois
revisava na versão impressa. \LaTeX{} é uma ferramenta baseada nesse processo:
você prepara seu texto no editor de sua preferência, insere comandos no texto
que indicam a estrutura do documento e o processa com o \LaTeX{}, que gera um
arquivo \textsc{pdf} formatado. Embora seja um estilo ``antigo'' de trabalhar,
ele é muito eficiente em vários casos. Ou seja, dependendo da situação, pode
ser mais adequado trabalhar fazendo tudo ao mesmo tempo ou dividindo o trabalho
nessas duas fases. De maneira geral:

\begin{itemize}
\item Se você precisa criar páginas diferentes entre si com \emph{layout}
definido manualmente, é melhor usar uma ferramenta que permita trabalhar
visualmente, como LibreOffice Writer, MS-Word, Google Docs etc.;

\item Se você precisa fazer um documento relativamente longo com estrutura
regular (capítulos, seções etc.), é melhor usar ferramentas que formalizam
essa estrutura (como \LaTeX{}) ao invés de ferramentas visuais;

\item Se você precisa fazer um documento envolvendo referências cruzadas,
bibliografia relativamente extensa ou fórmulas matemáticas, é difícil
encontrar outra ferramenta tão eficiente quanto \LaTeX{};

\item Se você precisa criar um documento simples, ambas as abordagens
funcionam bem; cada um escolhe esta ou aquela em função da familiaridade
com as ferramentas;

\item Se você quer que a qualidade tipográfica do resultado seja realmente
excelente, é necessário usar uma ferramenta profissional, como \LaTeX{},
Scribus, Adobe InDesign ou outras; processadores de texto convencionais não
oferecem o mesmo nível de qualidade dessas ferramentas\footnote{A maior
diferença (mas não a única) é o algoritmo que divide cada parágrafo em uma
série de linhas: \TeX{} (desde 1982) e Adobe InDesign (desde 1999) analisam
cada parágrafo como um todo, ao invés de uma linha por vez, para obter
espaçamentos mais homogêneos e menos palavras hifenizadas.}.
\end{itemize}

\section{Visão geral}

\enlargethispage{.5\baselineskip}

Com \LaTeX{}, você prepara o texto (incluindo as indicações de estrutura) em
um editor de textos qualquer, salva como arquivo de texto puro (``.txt'',
mas é comum usar a extensão ``.tex'' ao invés de ``.txt'') e processa esse
arquivo com o comando ``latexmk'' (``compila'' o documento) para obter o
\textsc{pdf} correspondente. Qualquer editor capaz de salvar arquivos em formato
texto puro, como o bloco de notas do windows, vim, emacs etc. pode ser usado.
Programas como LibreOffice Writer, MS-Word etc. também funcionam, mas
possivelmente vão gerar dores de cabeça porque vão tentar formatar algumas
coisas automaticamente (e de maneira incompatível com \LaTeX{}).

Em geral, é recomendável usar editores projetados especificamente para
trabalhar com \LaTeX{}; eles utilizam cores para distinguir o texto dos
comandos de formatação, automatizam o processo de compilação do documento
(veja a Seção~\ref{sec:make})
e oferecem outras comodidades. O mais usado atualmente é o \TeX{}studio,
que é software livre e funciona em Windows, macOS e Linux. O editor Visual
Studio Code (\url{code.visualstudio.com}) é voltado para programadores e
tem uma interface às vezes peculiar para outros usuários, mas em conjunto
com a \emph{package} \pkg{LaTeX Workshop} (do editor, não do \LaTeX), é uma
boa opção. O mesmo vale para o editor emacs (\url{www.gnu.org/software/emacs})
e sua package \pkg{AUC\TeX{}}. Ainda outra possibilidade são os editores
\emph{online}; dentre eles, o overleaf (\url{www.overleaf.com}) é o mais usado.
\looseness=-1

Um documento \LaTeX{} é dividido em duas partes: o \emph{preâmbulo}, onde
você coloca comandos de configuração para o documento, e o \emph{corpo}
do documento em si, que contém o texto propriamente dito. O preâmbulo é
onde você define as características do resultado tipográfico esperado
para o documento como um todo: tipo e tamanho da fonte a usar, posição
dos títulos e subtítulos na página etc. O corpo, por sua vez, consiste no
texto e em alguns comandos indicativos da estrutura.

Dado que configurar o preâmbulo é um tanto complexo e que mesmo no corpo
do texto às vezes há comandos especiais (para a geração da bibliografia
ou tabelas, por exemplo),
usar algum documento existente como base para criar seu texto em geral é
uma boa ideia. O IME/USP oferece um conjunto de modelos adequados para
teses/dissertações, artigos, apresentações e pôsteres (\url{gitlab.com/ccsl-usp/modelo-latex})
que pode ser adaptado para outros usos e outras instituições. Há também uma
família de modelos (\url{www.abntex.net.br}) que procura seguir as normas
da ABNT para diversos tipos de documentos científicos, e algumas publicações
científicas fornecem modelos de acordo com suas diretrizes.

\section{Estrutura de um documento \LaTeX{}}
\label{sec:basico}

\enlargethispage{.5\baselineskip}

O preâmbulo \LaTeX{} começa com a definição da \emph{classe} a ser utilizada,
que determina boa parte da configuração do documento. As principais classes
são \pkg{book}, \pkg{article} e \pkg{beamer} (para apresentações); você pode
saber mais sobre elas (e outras) em qualquer texto introdutório sobre \LaTeX{}
na Internet (veja a Seção~\ref{sec:docs})\footnote{Algumas revistas acadêmicas
têm suas próprias classes; por exemplo, a AMS (American Mathematical Society)
disponibiliza as classes \pkg{amsart}, \pkg{amsbook} e \pkg{amsproc}; você
pode usá-las para seus trabalhos mesmo que não pretenda publicar com a AMS,
veja \cmd{texdoc Author\_Handbook\_Journals}.}. A seguir, são carregadas
várias \emph{packages} (``\emph{plugins}'') que acrescentam funcionalidades ou
modificam as classes padrão; qualquer documento \LaTeX{} utiliza várias delas.
A classe é definida com o comando \ltxcmd{documentclass\{nome-da-classe\}};
packages são carregadas com o comando \ltxcmd{usepackage\{nome-da-package\}}.
Classes e packages podem receber opções adicionais entre colchetes
(\ltxcmd{usepackage[opção1,opção2...]\{nome-da-package\}}); a documentação
de cada package e classe (veja a Seção~\ref{sec:docs}) detalha as opções
disponíveis.

\LaTeX{} ignora quebras de linha e trata sequências de vários espaços como
se fossem apenas um. Isso significa que você pode usar quebras de linha e
espaços no texto que está digitando como ``dicas visuais'' da estrutura do
texto durante a edição. É muito comum fazer isso com listas de itens, por
exemplo (veja a Seção~\ref{sec:estrutura}). Uma ou mais linhas em branco
sinalizam o fim de um parágrafo e o início de outro. O caractere ``\%''
indica que o restante da linha é um comentário, ou seja, um trecho de texto
que não tem nenhum efeito sobre o resultado final do documento. Comentários
podem ser usados como lembretes sobre alguma decisão, para indicar um
parágrafo que ainda precisa de revisão etc. Por conta desse significado
especial, para inserir um caractere \% ``normal'' no texto é preciso digitar
``\ltxcmd{\%}''.\looseness=-1

Como mencionado anteriormente, \LaTeX{} divide o trabalho de produção
de um texto entre a preparação do conteúdo e a definição da forma de
apresentação. Assim, os comandos usados durante a produção do conteúdo
procuram expressar o \emph{significado} de cada elemento, e não sua
aparência. Por exemplo, para realçar uma palavra é comum usar texto
\textit{em itálico}; embora exista um comando especificamente para gerar
textos em itálico em \LaTeX{}, o recomendado é que se utilize o comando
\ltxcmd{emph} (``enfatizado''), pois em alguns casos pode ser melhor
utilizar \textbf{negrito}, \textsc{Versalete} ou outro mecanismo para
dar ênfase a uma palavra. Essa é uma orientação geral para a escrita de
textos com \LaTeX{}: procure definir a estrutura, não a aparência.

Um exemplo de documento \LaTeX{} simples (lembre-se, ``\%'' indica um
comentário):

\begin{verbatim}
        % O documento começa com o preâmbulo
        % Vamos usar a classe "book" com fonte no tamanho 11pt
        \documentclass[11pt]{book}
        % Vamos escrever em português do Brasil
        \usepackage[brazilian]{babel}
        % Finaliza o preâmbulo e inicia o conteúdo:
        \begin{document}
        % Estas linhas não imprimem nada, apenas definem as
        % informações que serão usadas por "\maketitle" a seguir
        \author{Fulano de Tal}
        \title{Começando a usar o \LaTeX{}}
        % Cria um bloco ou página de título com os dados acima
        \maketitle
        % Capítulos, seções etc. são numerados automaticamente
        \chapter{Cheguei!}
        Oi, Galera!
        % É preciso sinalizar o final do documento
        \end{document}
\end{verbatim}

Esse exemplo mostra como definir o nome de um capítulo. Existem também os
comandos \ltxcmd{section}, \ltxcmd{subsection}, \ltxcmd{subsubsection} e
\ltxcmd{paragraph} (a classe \pkg{book} inclui também \ltxcmd{part}, um nível
acima de \ltxcmd{chapter}). Usar o nome do comando seguido de um asterisco
(\ltxcmd{chapter*} etc.) faz o capítulo/seção não ser numerado (nem
considerado na contagem de capítulos, seções etc.) nem incluído no
sumário. Este modelo ainda define \ltxcmd{unnumberedchapter},
\ltxcmd{unnumberedsection} e \ltxcmd{unnumberedsubsection}, que eliminam
a numeração mas incluem o capítulo ou seção no sumário (úteis para a
introdução, por exemplo).

\section{Executando \LaTeX{} e comandos auxiliares}
\label{sec:make}

\enlargethispage{.5\baselineskip}

Depois de escrever o arquivo \cmd{.tex}, é preciso \emph{compilá-lo}, ou
seja, processá-lo para gerar o \textsc{pdf} desejado. Isso envolve executar,
além do próprio \LaTeX{} (veja a Seção~\ref{sec:versions}), alguns
programas auxiliares (em geral, \cmd{biber} ou \cmd{bibtex} e
\cmd{makeindex}). Nesse processo, \LaTeX{} quase sempre precisa ser
executado três ou mais vezes antes de gerar o \textsc{pdf} final\footnote{A
cada vez, ele gera uma nova versão intermediária do arquivo \textsc{pdf},
mas essas versões têm defeitos, como citações e referências cruzadas
incorretas ou sumário inexistente.}. Por conta dessa complexidade, é comum
utilizar alguma ferramenta para automatizar o processamento. Existem diversas
opções, mas a mais comum é o \cmd{latexmk}, que é capaz de identificar
automaticamente os passos necessários para a geração do documento,
executando os programas na ordem correta quantas vezes forem
necessárias\footnote{É possível personalizar o comportamento de \cmd{latexmk}
com o arquivo de configuração \cmd{latexmkrc}.}. Assim, embora seja possível
gerar o \textsc{pdf} executando apenas \cmd{pdflatex nome-do-arquivo.tex},
acostume-se a compilar o documento sempre com \cmd{latexmk -pdf nome-do-arquivo.tex}.
Note que editores especializados em \LaTeX{} costumam ter uma opção de menu
para a compilação do documento; dentre as configurações possíveis do editor,
prefira sempre a que simplesmente aciona \cmd{latexmk}.

\section{Mais sobre estrutura}
\label{sec:estrutura}

Para criar listas de itens, você pode fazer\footnote{Observe o uso de
espaços no início das linhas com \ltxcmd{item} para deixar a
estrutura visualmente mais clara durante a edição.}:

\begin{verbatim}
        \begin{itemize}
            \item Pedra
            \item Papel
            \item Tesoura
        \end{itemize}
\end{verbatim}

Além de ``itemize'', há também ``enumerate'' (auto-explicativo) e ``description'':

\begin{verbatim}
        \begin{description}
            \item[Pedra:] perde para papel;
            \item[Papel:] perde para tesoura;
            \item[Tesoura:] perde para pedra.
        \end{description}
\end{verbatim}

Citações curtas normalmente são incluídas no fluxo normal do texto e colocadas
entre aspas; para citações mais longas, use \ltxcmd{begin\{quote\}} ou
\ltxcmd{begin\{quotation\}} (este último é mais adequado para citações com
vários parágrafos). A package \pkg{csquotes} acrescenta recursos sofisticados
para citações.

Para poesia, use \ltxcmd{begin\{verse\}} (a package \pkg{verse} acrescenta
vários recursos ao comando \cmd{verse}). Estrofes são separadas por uma linha
em branco e versos são separados por \cmd{\sla\sla{}*}. O asterisco é opcional;
ele instrui \LaTeX{} a manter as linhas na mesma página.

Para inserir uma nota de rodapé, use o comando \ltxcmd{footnote\{texto da
nota\}}\index{Notas de rodapé}.

\section{Figuras e tabelas (\emph{floats})}
\label{sec:floats}

\enlargethispage{.5\baselineskip}

É possível utilizar \ltxcmd{includegraphics} para acrescentar figuras
ao texto (nos formatos \textsc{pdf}, \textsc{png} e \textsc{jpeg}),
mas normalmente elas não são inseridas diretamente. A razão é que, se
você simplesmente inserir uma figura em qualquer lugar, ela pode
ser grande demais para o espaço disponível na página, o que forçará
\LaTeX{} a deixar um espaço em branco e colocá-la na página seguinte.
O mesmo vale para tabelas (criadas com \ltxcmd{begin\{tabular\}}).
Para contornar esse problema, \LaTeX{} possui \emph{floats}, que
são blocos com algum conteúdo cuja localização é flexível: \LaTeX{}
procura colocar um \emph{float} ``perto'' de onde ele foi definido,
mas não necessariamente no lugar exato.

Ao invés de um único comando como ``\ltxcmd{begin\{float\}}'' a
ser usado tanto para figuras quanto para tabelas, \LaTeX{} define
\ltxcmd{begin\{figure\}} e \ltxcmd{begin\{table\}}. Ele faz isso
porque, assim como com capítulos e seções, \LaTeX{} também numera
figuras e tabelas --- mas, para isso, ele precisa saber qual é o tipo de
cada \emph{float}\footnote{É possível criar outros tipos de \emph{float}
também: como pode ser visto no Captítulo~\ref{chap:exemplos}, este
modelo define o tipo \cmd{program}.}. À parte isso, o conteúdo de
um \emph{float} pode ser qualquer coisa mas, em geral, é
\ltxcmd{includegraphics} ou \ltxcmd{begin\{tabular\}} respectivamente.

Uma consequência importante (e proposital) dos tipos diferentes de
\emph{floats} é que \LaTeX{}\index{Floats!ordem} garante que a sequência
das figuras e a sequência das tabelas sejam respeitadas (a Figura~6 nunca
aparece depois da Figura~7). No entanto, isso \emph{não} se aplica a
\emph{floats} de tipos diferentes, ou seja, se você definiu a Figura~5,
a Tabela~3 e a Figura~6, elas podem aparecer no documento na ordem
``Figura~5, Tabela~3, Figura~6'', ``Figura~5, Figura~6, Tabela~3'' ou
``Tabela~3, Figura~5, Figura~6''.

\section{Referências cruzadas}
\label{sec:refs}

É comum que um trecho do texto faça referência a outro trecho (``como
discutimos no Capítulo~X\ldots''). Isso pode ser feito diretamente, mas
se você reorganizar o documento ou acrescentar seções, a numeração pode
mudar. Para evitar esse problema, você pode gerar essas referências
automaticamente com o par de comandos \ltxcmd{label\{nome-sugestivo\}} e
\ltxcmd{ref\{nome-sugestivo\}} (para o número da seção/capítulo) ou
\ltxcmd{pageref\{nome-sugestivo\}} (para o número da página).

Esse mecanismo também é muito útil para figuras e tabelas. Dentro do
\emph{float}, além da figura em si, em geral é uma boa ideia acrescentar
uma legenda com \ltxcmd{caption}\index{Legendas}. Além disso, é possível
inserir um \ltxcmd{label} dentro da legenda para que se possa fazer
referência à figura/tabela no texto (com os comandos \ltxcmd{ref} e
\ltxcmd{pageref}).

\section{Referências bibliográficas e bibliografia}

A geração de bibliografias no \LaTeX{} é feita através da package
\pkg{biblatex}\index{biblatex} e do programa auxiliar
\cmd{biber}\index{biber}\footnote{Antigamente, usava-se a package
\pkg{natbib}\index{natbib} e o comando auxiliar \cmd{bibtex}\index{bibtex}.
O funcionamento geral dos dois mecanismos é similar e o formato do banco
de dados de ambos é o mesmo.} e envolve três passos:

\begin{enumerate}
\item A criação de um banco de dados, no formato ``.bib'', das obras de
interesse. Esse banco de dados pode incluir obras que não vão ser de fato
referenciadas no documento final. Isso significa que você pode criar um
único banco de dados e utilizá-lo em todos seus documentos\footnote{É
comum criar bancos de dados desse tipo separados por assunto, mas isso
não é necessário.}.

\item A inserção de referências às obras ao longo do texto, usando
diferentes comandos dependendo do caso: \ltxcmd{cite}, \ltxcmd{citet},
\ltxcmd{citep} etc. Como já mencionado, esses comandos estão descritos
na documentação da package \pkg{natbib}\index{natbib} \citep{natbib}.

\item A escolha do estilo bibliográfico (usando as opções da package
\pkg{biblatex}) que formata as citações ao longo do texto e gera a bibliografia
automaticamente através do comando \ltxcmd{printbibliography}. Normalmente,
apenas as obras efetivamente citadas são incluídas na lista de referências,
mas é possível forçar a inclusão de uma obra sem citá-la explicitamente com
o comando \ltxcmd{nocite}.
\end{enumerate}

O banco de dados é um arquivo de texto contendo uma \emph{entrada} para cada
item da bibliografia e, em cada entrada, uma série de \emph{campos} com os
dados (título, autor etc.). A entrada inclui também uma \emph{chave}, que é
usada para inserir as citações no texto. Há vários tipos de entrada (para
artigos, livros, sítios web etc.) e, para cada tipo, uma lista de campos
possíveis (considere que periódicos normalmente incluem o número do volume,
mas teses não). O exemplo abaixo é um livro cuja chave é ``dissertjourney'';
ele pode ser citado com o comando \ltxcmd{cite\{dissertjourney\}}:

\begin{verbatim}
        @book{dissertjourney,
            author    = {Carol M. Roberts},
            title     = {The Dissertation Journey},
            publisher = {Corwin},
            year      = 2010,
            edition   = 2,
            location  = {Thousand Oaks, CA},
        }

\end{verbatim}

Em alguns casos, \LaTeX{} troca as letras maiúsculas definidas em
\ltxcmd{title} para minúsculas. Para evitar que isso afete siglas
ou nomes próprios, basta colocá-los entre chaves (``Automated
Application-Level Checkpointing of \{MPI\} Programs'').

% Esta informação é muito pouco relevante...
%Observe que existem dois formatos comumente usados para escrever títulos
%de artigos, livros etc:
%
%\begin{description}
%  \item[Title case:] Substantivos, adjetivos e verbos (além de nomes
%  próprios e siglas) são escritos com a primeira letra maiúscula (``Um
%  Exemplo de Título no Estilo Title Case''). Em geral, a regra não se
%  aplica ao título de artigos ou capítulos de livro, apenas aos livros
%  dos quais eles fazem parte;
%
%  \item[Sentence case:] O título é escrito como qualquer outra frase
%  (``Um título só tem maiúsculas em abreviaturas, como ABNT, ou nomes
%  próprios'').
%\end{description}
%
%Cada estilo de bibliografia utiliza um desses formatos e, portanto, é
%desejável que o banco de dados funcione corretamente com ambos. No
%entanto, nem sempre é claro quais palavras devem ser iniciadas com letra
%maiúscula ao usar \emph{title case} e, por conta disso, não há um sistema
%automático em \LaTeX{} para adaptar títulos a ele. Sendo assim, como fazer
%um banco de dados bibliográfico capaz de funcionar com os dois formatos?
%A solução é sempre inserir os títulos dos itens no banco de dados seguindo
%o formato \emph{title case}. Se o estilo utiliza esse formato, o título
%é reproduzido na bibliografia como digitado no banco de dados. Se o estilo
%usa \emph{sentence case}, o texto (exceto a primeira letra) é convertido
%para letras minúsculas. Para evitar que isso afete siglas e nomes próprios,
%basta colocá-los entre chaves (``Automated Application-Level Checkpointing
%of \{MPI\} Programs'').

Os campos \textsf{author} e \textsf{publisher} podem incluir uma lista
de nomes separados por \textsf{and}; biblatex reconhece que cada nome é
composto por nome e sobrenome, às vezes com partículas como ``de'', ``dos''
ou ``von'' e, dependendo do estilo bibliográfico, pode abreviar nomes, mudar
sobrenomes para caixa alta etc. Isso evidentemente não funciona quando o autor
é, na verdade, uma instituição; nesses casos, basta colocar o nome inteiro da
instituição entre chaves (``\{Universidade de São Paulo --- Sistema Integrado
de Bibliotecas\}'') para que biblatex não faça alterações desse tipo. Se o
nome é longo, pode ser interessante definir o campo \textsf{shortauthor}.

A fonte mais detalhada de informações sobre o banco de dados é a documentação
da package \pkg{biblatex} \citep[em especial as seções 2.1.1 e 2.2.2]{biblatex},
mas o material ali é um tanto denso.
Há muito material introdutório ao formato ``.bib'' e ao bibtex disponível
\emph{online}, e você pode se inspirar em exemplos para criar seu banco de
dados bibliográfico. Além disso, ferramentas como Zotero\index{Zotero} ou
Mendeley\index{Mendeley} (o uso de uma delas é altamente recomendado!)
podem exportar para o formato .bib. Observe que \pkg{biblatex}
\index{biblatex} oferece recursos bastante sofisticados para o tratamento de
referências e bibliografias. Se você precisar de alguma funcionalidade
especial, consulte a documentação do pacote ou a Internet; é quase certeza
que \pkg{biblatex} oferece uma solução.


\section{Fórmulas matemáticas}

A diagramação de fórmulas matemáticas tem regras específicas: letras são
interpretadas como variáveis e espaços em branco são ignorados (\LaTeX{}
usa o contexto da fórmula para definir o espaçamento). Assim, para criar
fórmulas em \LaTeX{}, é preciso usar um comando para iniciar o modo
matemático. Isso pode ser feito de duas formas:

\begin{itemize}
  \item Pequenas fórmulas no meio do texto ($e^{i\pi}+1=0$) são inseridas
  com \cmd{\$\emph{fórmula}\$} (e, portanto, para inserir um caractere \$
  normal no texto, é preciso usar \cmd{\sla{}\$}).

  \item Fórmulas mais longas ou que devem aparecer em um parágrafo
  separado são inseridas com \cmd{\sla{}[\emph{fórmula}\sla{}]} (ou
  \ltxcmd{begin\{displaymath\}}).
\end{itemize}

\LaTeX{} é capaz de oferecer uma boa solução para praticamente qualquer
problema de diagramação para matemática; basta ler a documentação.

\section{Formatação manual}

Às vezes é preciso inserir formatação de forma manual; os comandos mais
importantes são:
\ltxcmd{emph} (texto \emph{enfatizado}, em geral itálico),
\ltxcmd{texttt} (texto \texttt{teletype}, imitando um
terminal de texto ou uma impressora),
\ltxcmd{textit} (\textit{itálico}),
\ltxcmd{textbf} (\textbf{negrito}),
\ltxcmd{textsf} (fonte \textsf{sem serifa}),
\ltxcmd{textsc} (texto \textsc{Versalete} --- nem todas
as fontes oferecem essa possibilidade),
\ltxcmd{normalsize} (tamanho normal),
\ltxcmd{small} (tamanho reduzido),
\ltxcmd{footnotesize} (ainda menor),
\ltxcmd{scriptsize} (ainda menor),
\ltxcmd{tiny} (ainda menor),
\ltxcmd{large} (tamanho aumentado),
\ltxcmd{Large} (ainda maior),
\ltxcmd{LARGE} (ainda maior),
\ltxcmd{Huge} (ainda maior),
\ltxcmd{vspace\{\sla{}baselineskip\}} (deixa uma linha em branco),
\ltxcmd{begin\{center\}} (centraliza parágrafos),
\ltxcmd{begin\{flushleft\}} (alinha parágrafos à esquerda),
\ltxcmd{begin\{flushright\}} (alinha parágrafos à direita)\footnote{É
altamente recomendável carregar a package \pkg{ragged2e} (já incluída
neste modelo) e utilizar \cmd{Center}, \cmd{FlushLeft} e
\cmd{FlushRight} ao invés de \cmd{center}, \cmd{flushleft}
e \cmd{flushright}.},
\ltxcmd{babelhyphenation} (permite ``ensinar'' \LaTeX{} como hifenizar
uma lista de palavras, veja \cmd{texdoc babel}; note que, em geral, a
hifenização automática de \LaTeX{} é excelente),
\ltxcmd{-} (sugere uma possível hifenização localizada),
\ltxcmd{linebreak}[0--4] (sugere uma quebra de linha; o número indica
quão forte é a sugestão, ou seja, 4 faz a quebra obrigatória; se o
parágrafo é justificado, a linha quebrada também é justificada),
\ltxcmd{newline} ou \cmd{\sla\sla} (força uma quebra de linha; a
linha \emph{não} é justificada nesse caso),
\ltxcmd{pagebreak}[0--4] (sugere uma quebra de página; como
\ltxcmd{linebreak}, o número indica quão forte é a sugestão; o texto
da página é espalhado verticalmente de maneira a fazer a última linha
alinhada com o final das demais páginas) e
\ltxcmd{newpage} (força uma quebra de página; o final da página
\emph{não} é alinhado com o final das demais páginas nesse~caso).

Mas, como discutido na Seção~\ref{sec:basico}, não é recomendável
usar esses comandos ao longo do texto: o ideal em \LaTeX{} é expressar
o significado de cada elemento, não a sua forma de apresentação,
pois isso permite que você faça alterações na formatação com mais
facilidade. Assim, quando os recursos pré-definidos do \LaTeX{}
(\ltxcmd{itemize}, \ltxcmd{chapter} etc.) não forem suficientes,
o mais adequado é definir comandos novos, em geral usando os comandos
de formatação mencionados acima. Esse é um tópico avançado, mas você
pode consultar o início do arquivo \LaTeX{} deste capítulo para alguns
exemplos simples.

% Esta informação é muito pouco relevante...
%\section{Detalhes da linguagem}
%
%Há quatro estilos típicos de comandos \LaTeX{}:
%
%\begin{itemize}
%\item Comandos que se referem a um parâmetro; por exemplo,
%\ltxcmd{emph\{um texto\}} significa ``escreva a frase
%`um texto' com ênfase'' (em geral, itálico). As chaves delimitam o início
%e o final do escopo sobre o qual o comando tem efeito. Aqui entram também
%comandos como \ltxcmd{title} e \ltxcmd{author},
%que não escrevem nada diretamente mas definem o título e autoria do documento
%(essa informação é usada, por exemplo, por \ltxcmd{maketitle}).
%
%\item Comandos que se referem a um parâmetro que é um bloco grande de
%texto, possivelmente vários parágrafos; por exemplo, \ltxcmd{begin\{center\}}
%um texto \ltxcmd{end\{center\}} faz ``um texto'' (que podem ser vários
%parágrafos) ser centralizado.
%
%\item Comandos que ativam alguma opção; por exemplo, \ltxcmd{itshape}
%significa ``ative o modo itálico''. Nesse caso, o texto vai ser impresso
%em itálico até outro comando selecionar outro estilo de fonte. Se o comando
%for inserido dentro de um bloco delimitado por chaves, ele ``perde o
%efeito'' após o caractere de fecha-chaves (exemplo: ``\{\ltxcmd{itshape\{\}}
%Fulano de Tal\} é meu nome'' será impresso como ``\textit{Fulano de Tal} é
%meu nome''). Você normalmente não vai utilizar esse estilo de comando, mas
%ele é útil em alguns casos.
%
%\item Comandos que fazem o programa escrever algo específico; por exemplo,
%em várias classes padrão o comando \ltxcmd{maketitle} gera
%uma página de título com o nome do trabalho, autor etc.
%\end{itemize}
%
%Nos dois últimos, não é preciso usar chaves após o comando. Ainda assim, as
%chaves podem ser colocadas e muitas vezes isso é bom: sem elas, \LaTeX{}
%entende que o caractere espaço que se segue a esses comandos serve apenas
%como separador em relação ao que vem a seguir. Por conta disso, ele ignora
%esse espaço. Quando isso não é o que se deseja, a solução é usar as chaves:
%\ltxcmd{itshape\{\}}.
%Vale observar que alguns comandos aceitam mais de um parâmetro, às vezes
%entre chaves, às vezes entre colchetes. Você pode descobrir a sintaxe
%correta para cada caso lendo a documentação de cada comando.

\section{Versões do \LaTeX{}}
\label{sec:versions}

Assim como há packages para o \LaTeX{}, o próprio \LaTeX{} é, na verdade, um
conjunto de extensões para o programa \TeX{}. Assim, se você encontrar
referências a ``\TeX{}'' ou a ``plain \TeX{}'', basta saber que esse é o
sistema que funciona ``por baixo'' do \LaTeX{}.

\LaTeX{} é um sistema em evolução (desde os anos 80!). Uma das consequências
disso é que há, na verdade, quatro versões diferentes dele:

\begin{enumerate}
\item \LaTeX{} ``tradicional'', que gera arquivos em formato \textsc{dvi}
que, por sua vez, precisam ser convertidos para o formato \textsc{pdf}.
Essa versão não é capaz de usar as fontes instaladas no sistema; ela só
pode usar fontes adaptadas para uso com o \LaTeX{}. Hoje em dia não há
boas razões para usar essa versão.

\item pdf\LaTeX{}, que gera arquivos \textsc{pdf} e dá suporte a alguns
recursos avançados de tipografia adicionais. É a versão mais usada hoje
em dia, embora também só possa usar as fontes adaptadas para uso com o
\LaTeX{}.

\item \XeLaTeX{} que, além dos recursos do pdf\LaTeX{}, opera internamente
em UTF-8 (ou seja, funciona melhor com múltiplas línguas) e pode funcionar
não só com as fontes adaptadas para o \LaTeX{} como também com as fontes
instaladas no sistema. \XeLaTeX{} foi muito importante ao ser lançado,
mas atualmente a comunidade está mais empenhada em evoluir o sistema com
\LuaLaTeX{}.

\item \LuaLaTeX{}, que oferece os mesmos recursos que o \XeLaTeX{} e
também pode ser estendido internamente com mais facilidade (através da
linguagem de programação Lua).
\end{enumerate}

Todas essas versões são instaladas quando você instala \LaTeX{} na
sua máquina. Em geral, se você pretende escrever apenas com línguas no
alfabeto latino e não pretende usar fontes diferentes das disponíveis
por padrão, qualquer das três versões modernas (pdf\LaTeX{},
\XeLaTeX{} e \LuaLaTeX{}) é adequada; pdf\LaTeX{} é um pouco mais
rápido, mas \LuaLaTeX{} gera arquivos \textsc{pdf} um pouco menores.
Se você pretende usar outros alfabetos, gostaria de escolher
fontes diferentes ou precisa de recursos tipográficos específicos
(\cmd{texdoc fontspec}, \cmd{texdoc unicode-math}), use \LuaLaTeX{}.

\section{Limitações do \LaTeX{}}
\label{sec:limitations}

Como qualquer ferramenta, \LaTeX{} tem limitações e características
indesejáveis:

\begin{itemize}
    % \linebreak[0]{} -> sugestão (não-obrigatória) de quebra de linha
    \item A linguagem é muito prolixa: é bastante tedioso escrever
    coisas como ``\ltxcmd{begin\linebreak[0]{}\{itemize\}}'' etc.
    Linguagens como asciidoc (\url{asciidoctor.org}), markdown
    (\url{commonmark.org}), bookdown (\url{bookdown.org}) e
    reStructuredText (\url{sphinx-doc.org}) operam de maneira similar
    a \LaTeX{}, mas sua sintaxe é bem mais enxuta. Elas funcionam
    muito bem para a geração de páginas web, mas \LaTeX{} oferece
    mais recursos e geralmente produz resultados impressos melhores.

    \item \LaTeX{} gera muitas mensagens pouco importantes durante
    o processamento do documento, o que dificulta a identificação
    de problemas (o programa auxiliar \cmd{texlogsieve}, incluído
    com versões recentes de \LaTeX{}, pode minimizar esse incômodo).
    Além disso, quando ocorrem erros durante esse processamento, as
    mensagens explicativas muitas vezes são confusas ou, pior, não
    indicam o problema real que causou a falha.

    \item \LaTeX{} procura ser uma linguagem \emph{declarativa}, ou seja,
    os comandos buscam expressar o que se deseja e não como fazer algo
    (``este texto é um título'' e não ``pule duas linhas, selecione uma
    fonte maior, escreva este texto, pule mais duas linhas e selecione a
    fonte de tamanho padrão''). No entanto, ela é insuficiente em algumas
    situações, obrigando o usuário a utilizar vários comandos, às vezes
    obscuros, para obter resultados relativamente simples.

    \item Há diversas packages para personalizar os aspectos básicos
    da formatação final do documento, como o tipo de fonte, tamanho dos
    títulos das seções, espaçamento etc. No entanto, quando se quer
    fazer modificações maiores, é preciso lidar com partes complexas da
    linguagem e diversos comportamentos surpreendentes.

    \item Às vezes há incompatibilidades entre packages; em alguns casos,
    isso pode ser contornado mudando a ordem em que elas são carregadas,
    mas em outros pode simplesmente não ser possível combiná-las.

    \item A colocação automática dos \emph{floats} e o algoritmo
    que encontra as quebras de página em geral funcionam bem, mas às
    vezes é possível obter resultados melhores manualmente (veja as
    Seções~\ref{sec:exemplos-graficos} e \ref{sec:quebras}).

    % LaTeX atualmente usa utf-8 por padrão; esta informação já
    % era pouco útil antes, agora é quase totalmente irrelevante
    %\item Como muitos outros sistemas de texto, \LaTeX{} pode usar mais de
    %um padrão para a codificação de caracteres acentuados (através da
    %configuração da package \pkg{inputenc}). Alguns anos atrás,
    %o mais comum era o ISO-8859-1, também conhecido como latin1 (esse é o
    %nome usado no \LaTeX) ou Windows-1252; atualmente, o mais comum é o
    %UTF-8 (default em versões recentes de \LaTeX). De maneira geral, é
    %simples reconhecer e resolver os problemas
    %causados por inconsistências na codificação (seja trocando a opção
    %de \pkg{inputenc}, seja recodificando o arquivo), mas arquivos ``.bib''
    %são um caso especial: biblatex (usado neste modelo) funciona normalmente
    %com caracteres acentuados, mas bibtex oficialmente não tem suporte a eles
    %(embora em geral funcione corretamente). Além disso, é bastante comum que
    %arquivos desse tipo sejam compartilhados por várias pessoas, com diferentes
    %configurações. Para evitar problemas com os acentos
    %nesse caso, uma possibilidade é representar os caracteres acentuados
    %usando comandos \LaTeX{}: \cmd{\sla\textquotesingle{}a} para á,
    %\cmd{\sla{}c\{c\}} para cedilha etc., independentemente da
    %codificação usada no texto\footnote{Você pode consultar os comandos
    %desse tipo mais comuns em \url{en.wikibooks.org/wiki/LaTeX/Special_Characters}.
    %Observe que a dica sobre o pingo do i \emph{não} é mais
    %válida atualmente; basta usar \cmd{\sla\textquotesingle{}i}.}.

    \item As classes padrão (\pkg{book}, \pkg{article} etc.) não foram criadas
    para serem facilmente modificadas, o que deu origem a inúmeras packages
    voltadas para possibilitar a personalização de diversos aspectos da
    apresentação final do documento. Esse mecanismo não é ideal, por diversas
    razões. Por conta disso, existe um conjunto de versões alternativas dessas
    classes (\pkg{scrbook} no lugar de \pkg{book}, \pkg{scrartcl} no lugar de
    \pkg{article} etc.) chamado \pkg{KOMA-Script}, com mais recursos e mais
    possibilidades de customização. A classe \pkg{memoir} tem o mesmo objetivo,
    mas procura dar suporte a livros e artigos com uma única classe. Ambas
    abordagens são muito boas, mas a maioria dos modelos usados por revistas e
    outras publicações é baseada nas classes padrão. A versão 3 de \LaTeX{}
    está em desenvolvimento com vistas a resolver boa parte dos problemas
    atuais do sistema, mas ainda deve demorar muitos anos para ficar pronta.
    \ConTeXt{} é um ``irmão mais novo'' de \LaTeX{} com diversas
    vantagens, mas com sintaxe diferente e que ainda não é tão popular.
\end{itemize}

%!TeX root=../tese.tex
%("dica" para o editor de texto: este arquivo é parte de um documento maior)
% para saber mais: https://tex.stackexchange.com/q/78101

\chapter{Exemplos e dicas de uso do \LaTeX{}}
\label{chap:exemplos}

Neste capítulo, apresentamos exemplos comuns com alguma complexidade e,
principalmente, pequenas dicas para evitar surpresas indesejáveis. Mesmo
que você já conheça \LaTeX{}, vale a pena analisar este material, incluindo
o código-fonte do capítulo. Se você ainda não conhece nada sobre \LaTeX{},
o Capítulo~\ref{chap:tutorial} e outros materiais citados na
Seção~\ref{sec:docs} apresentam os conceitos básicos.

\section{Bibliografia e referências}

A documentação do pacote biblatex\index{biblatex}~\citep{biblatex} é
bastante extensa e explica (nas Seções~2.1.1 e 2.2.2) os diversos
tipos de documento suportados, bem como o significado de cada campo.
Na prática, às vezes é preciso fazer escolhas sobre
o que incluir na descrição de um item bibliográfico e muitas vezes
é mais fácil aprender copiando exemplos já existentes, como estes (consulte o
arquivo \texttt{bibliografia.bib} para ver como foi criado o banco de dados e a
bibliografia na página \pageref{bibliografia} para ver o resultado impresso):

\begin{multicols}{2}
  \begin{itemize}
    \item @Book: \cite{Knuth:96}.

    \item @Article (em periódico): \cite{floats2014}.

    \item @InProceedings (ou @Conference): \cite{alves03:simi}.

    \item @InCollection (capítulo de livro ou coletânea): \cite{bobaoglu93:concepts}.

    \item @PhdThesis: \cite{garcia01:PhD}.

    \item @MastersThesis: \cite{schmidt03:MSc}.

    \item @Techreport: \cite{alvisi99:analysisCIC}.

    \item @Manual: \cite{biblatex}.

    \item @Misc: \cite{gridftp}.

    \item @Online (para referência a artigo \emph{online}): \cite{fowler04:designDead}.

    \item @Online (para referência a página web): \cite{FSF:GNU-GPL}.
  \end{itemize}
\end{multicols}

\textsf{biblatex} prefere o uso do campo ``date'' para definir ano, mês
etc. No entanto, se você quiser garantir compatibilidade tanto com
\textsf{biblatex} quanto com \textsf{bibtex}, use os campos ``year'' e
``month''. Ambos reconhecem diversos formatos para o campo ``month'',
mas apenas um funciona corretamente com os dois: o nome do mês em inglês,
abreviado com três letras minúsculas e sem chaves, ou seja:

\begin{verbatim}
    ...
    author = {Fulano de Tal},
    year = {2011},
    month = oct,
    title = {Um título grandioso},
    ...
\end{verbatim}

\section{Caracteres especiais}

Um espaço não-separável é indicado pelo caractere til
(``\cmd{\textasciitilde{}}'') e é possível forçar uma quebra de linha com
``\cmd{\sla\sla{}}''. Aspas tipográficas (``\;'' e `\;') são inseridas
com \`\space\,\`\space\space\,\textquotesingle\,\textquotesingle{} e
\`\space\,\,\textquotesingle. Os principais símbolos matemáticos estão
listados em \textsf{texdoc undergradmath} e você pode consultar a lista
completa de símbolos disponíveis com \textsf{texdoc symbols-a4} ou em
\url{www.ctan.org/tex-archive/info/symbols/comprehensive/symbols-a4.pdf}.
Uma outra maneira de encontrar símbolos é usar este sítio:
\url{detexify.kirelabs.org/classify.html}.

\section{Modo matemático}\index{Modo matemático}

O modo matemático do \LaTeX{} tem sintaxe própria, mas ela não é complicada e
há bastante documentação \emph{online} a respeito. Por exemplo, ``massa e
energia são grandezas relacionadas pela Equação $E=mc^2$, definida inicialmente
por Einstein'', ou ainda ``equações de segundo grau (Equação \ref{eq:2grau})
são estudadas no ensino médio. As raízes de uma equação de segundo grau podem
ser encontradas por~\eqref{eq:bhaskara} --- a fórmula de Bháskara.
O valor do discriminante $\Delta$ (Equação \ref{eq:delta}) determina se a
equação tem zero, uma ou duas raízes reais distintas''.

% Equação simples, com numeração à direita
\begin{equation}
  \label{eq:2grau}
  ax^2+bx+c=y \quad \forall x \in \mathbb{R}
\end{equation}

% Conjunto de equações agrupadas mas sem alinhamento, com numeração à direita
\begin{gather}
  \label{eq:bhaskara}
    y=0 \Leftrightarrow x=\frac{-b \pm \sqrt{\Delta}}{2a}
    \Leftrightarrow x \text{ é raiz da equação}\\
  \label{eq:delta}
    \Delta\enspace(\mathit{delta}) = b^2-4ac
\end{gather}

Para inserir um espaço explicitamente no modo matemático, use
\ltxcmd{quad} ou \ltxcmd{enspace}. Para inserir texto ``normal'' em
uma fórmula matemática, use \ltxcmd{text\{texto\}} (para texto de fato)
ou \ltxcmd{mathit\{texto\}} (para nomes de variáveis ou funções com
mais de uma letra). Pode ser necessário deixar um espaço no início do
texto para evitar que ele fique colado com o caractere matemático que
o antecede.

Para recursos mais sofisticados, incluindo frações com múltiplas linhas,
matrizes, sistemas de equações alinhadas, setas, acentos etc., procure
a documentação das packages \textsf{amsmath} e \textsf{mathtools}. Para
teoremas, lemas, conjecturas etc., leia a documentação da package
\textsf{amsthm} e decida de quais tipos de estrutura você vai precisar
no seu documento. Aqui criamos três: ``Pegadinha'', ``Teorema'' e
``Conjectura'' (observe as numerações):

% A aparência dos teoremas/lemas/etc. é definida por estilos. Há três
% estilos padrão (plain, definition e remark), mas é possível criar outros:
\newtheoremstyle{smile} % nome do estilo
{3pt} % espaço antes
{3pt} % espaço depois
{\itshape} % fonte do corpo
{} % Indentação
{\bfseries} % fonte do título
{} % pontuação após o título
{.5em} % espaço após o título
{\thmname{#1}\thmnumber{ #2}\thmnote{ (#3)} :-)\space} % formato do título
% vazio significa {\thmname{#1}\thmnumber{ #2}\thmnote{ (#3)}}

\newtheoremstyle{maybe} % nome do estilo
{3pt} % espaço antes
{3pt} % espaço depois
{\itshape} % fonte do corpo
{} % Indentação
{\bfseries} % fonte do título
{:} % pontuação após o título
{.5em} % espaço após o título
{\thmname{#1}\space(?)\thmnumber{ #2}\thmnote{ (#3)}} % formato do título
% vazio significa {\thmname{#1}\thmnumber{ #2}\thmnote{ (#3)}}

\theoremstyle{smile} % As próximas definições usam este estilo
\newtheorem{absurd}{Pegadinha}

\begin{absurd}\label{thm:doisum}
  $1=0$
\end{absurd}

% "proof" é pré-definido por amsthm
\begin{proof}
Tomemos dois números, $a$ e $b$, tais que $a=b+1$.

% Múltiplas linhas alinhadas em "&", sem numeração
\begin{displaymath}
  \begin{split}
    a &= b+1 \\
    (a-b)a &= (a-b)(b+1) \\
    a^2 - ab &= ab + a -b^2 -b \\
    a^2 -ab -a &= ab -b^2 -b \\
    a(a-b-1) &= b(a-b-1) \\
    a \cancel{(a-b-1)} &= b \cancel{(a-b-1)} \\
    a &= b \\
    b+1 &= b \\
    1 &= b-b \\
    1 &= 0
  \end{split}
\end{displaymath}
\end{proof}

\theoremstyle{plain} % As próximas definições usam este estilo
\newtheorem{theorem}{Teorema}

\begin{theorem}
  \label{thm:graphcolor}
  É sempre possível colorir os vértices de um grafo sem que dois vértices
  adjacentes tenham a mesma cor usando no máximo quatro cores diferentes.
\end{theorem}

\begin{proof}
  A demonstração do Teorema~\ref{thm:graphcolor} é um exercício
  a cargo do leitor.
  % Como isto não é de fato uma prova, queremos eliminar o símbolo
  % "QED" que o ambiente proof acrescenta automaticamente.
  \renewcommand\qedsymbol{} % tem efeito apenas nesta prova!
\end{proof}

\theoremstyle{maybe} % As próximas definições usam este estilo
\newtheorem{conjecture}{Conjectura}

\begin{conjecture}
  \label{conj:fermat}
  Dado qualquer inteiro $n > 2$, não existem inteiros positivos
  $a$, $b$ e $c$ tais que $a^n + b^n = c^n$.
\end{conjecture}

\begin{proof}
  Este espaço é muito pequeno para apresentá-la.
  \renewcommand\qedsymbol{}
\end{proof}

\begin{theorem}
  \label{thm:pnp}
  \textup{\textsf{\textbf{P$\ne$NP}}}
\end{theorem}

\begin{proof}
  \textsf{\textbf{P}} tem apenas uma letra, enquanto
  \textsf{\textbf{NP}} tem duas letras.
\end{proof}

\section{Figuras, gráficos e outros \emph{floats}}\index{Floats}

Evidentemente, \LaTeX{} permite inserir figuras no texto; além disso, ele
também permite girá-las e criar subfiguras (com sublegendas\index{Legendas}),
como no exemplo da Figura~\ref{fig:subfigures}\index{Subfiguras}, que inclui
as subfiguras~\ref{fig:subfigures:a} e~\ref{fig:subfigures:b}.

% As packages relevantes para lidar com figuras são graphicx,
% float, caption, rotating e subcaption. Observe que "subfigure"
% e "subtable" são definidos na package subcaption, *não* na
% package subfigure! A package subfigure é obsoleta.

%%%%%%%%% Figuras lado-a-lado %%%%%%%%%
\begin{figure}
  \centering

  \begin{subfigure}{0.4\textwidth}
    \centering
    \includegraphics[width=.8\textwidth]{exemplo-grafo}
    \caption{Uma figura simples.\label{fig:subfigures:a}}
  \end{subfigure}
  % ATENÇÃO: Se você deixar uma linha em branco entre as subfiguras,
  % LaTeX vai considerar que cada uma delas pertence a um "parágrafo"
  % diferente e, portanto, vai colocá-las em linhas separadas ao invés
  % de lado a lado.
  \begin{subfigure}{0.4\textwidth}
    \centering
    \begin{turn}{90} % package rotating
      \includegraphics[width=.8\textwidth]{exemplo-grafo}
    \end{turn}
    \caption{O mesmo exemplo, girado.\label{fig:subfigures:b}}
  \end{subfigure}

  \caption{Exemplo de subfiguras.\label{fig:subfigures}}
\end{figure}

Uma ``figura'', na verdade, pode ser qualquer tipo de conteúdo ilustrativo
(um exemplo interessante é o cronograma mostrado na Figura~\ref{fig:gantt})
mas, com a \textit{package} \textsf{float}, também é possível definir ambientes
específicos para cada tipo de conteúdo adicional (cada um com numeração
independente), como é o caso do Programa~\ref{prog:java}\index{Floats}. Há
mais informações e dicas sobre recursos específicos para inclusão de
código-fonte e pseudocódigo no Apêndice \ref{ap:pseudocode}\footnote{
Observe que o nome do Apêndice (``\ref{ap:pseudocode}'') foi impresso em
uma linha separada, o que não é muito bom visualmente. Para evitar que isso
aconteça (não só no final do parágrafo, mas em qualquer quebra de linha),
utilize um espaço não-separável para fazer referências a figuras, tabelas,
seções etc. ou antes de símbolos: ``\textsf{\dots no
Apêndice\textasciitilde\textbackslash{}ref\{ap:pseudocode\}}'',
``\textsf{O discriminante é denotado
por\textasciitilde{}\$\textbackslash{}Delta\$}''.}.

%%%%%%% Cronograma %%%%%%%

\begin{figure}
  \centering

  % Package pgfgantt
  \begin{ganttchart}[
                     time slot format=isodate-yearmonth,
                     time slot unit=month,
                    ]{2017-11}{2018-5}

    \gantttitlecalendar{year,month=shortname} \ganttnewline

    \ganttgroup[progress=45]{Experimento}{2017-11}{2018-2} \ganttnewline
    \ganttbar[progress=100]{
      Preparação\ganttalignnewline
      (compra de insumos)
      }{2017-11}{2017-12} \ganttnewline
    \ganttbar[progress=30]{Execução}{2017-12}{2018-1} \ganttnewline
    \ganttbar[progress=0]{Análise}{2017-12}{2018-2} \ganttnewline

    \ganttgroup[progress=0]{Artigo}{2018-1}{2018-4} \ganttnewline
    \ganttbar[progress=0]{Escrita}{2018-1}{2018-3} \ganttnewline
    \ganttbar[progress=0]{Revisão}{2018-3}{2018-4} \ganttnewline

    \ganttmilestone{Submissão}{2018-4}
  \end{ganttchart}

  \caption{Exemplo de cronograma.\label{fig:gantt}}
\end{figure}

%%%%%%%% Código fonte %%%%%%%%

% Foi utilizado o pacote listings para formatar o código fonte.
% Veja os parâmetros de configuração no arquivo source-code.tex.
\begin{program}
  \index{Java}
  \centering

\begin{lstlisting}[language=Java, style=wider]
  for (i = 0; i < 20; i++)
  {
      // Comentário
      System.out.println("Mensagem...");
  }
\end{lstlisting}

  \caption{Exemplo de laço em Java.\label{prog:java}}
\end{program}

%%%%%

\LaTeX{} também é capaz de gerar ilustrações e diagramas diretamente, mas
usar esses recursos em geral não é trivial. Em particular, a package
\textsf{tikz} oferece bons mecanismos para a criação de figuras (incluindo
funções pré-prontas para formas geométricas, grafos, matrizes etc.) e é
fácil usá-la para traçar linhas ou curvas simples.

Gráficos de dados ou funções matemáticas de excelente qualidade podem ser
gerados com a \textit{package} \pkg{pgfplots} (há um exemplo comentado neste
arquivo; experimente des-comentar para ver o resultado). Também é possível
importar gráficos gerados por \texttt{matplotlib} e por \texttt{gnuplot}
como qualquer outra imagem, mas nesse caso a fonte usada nesses gráficos
provavelmente será diferente do corpo do texto. Felizmente, isso pode ser
solucionado: \cmd{Gnuplot}, com o \emph{driver} \cmd{lua tikz}\footnote{
\url{www.gnuplot.info/docs\_5.2/Gnuplot\_5.2.pdf\#section*.516}}, e
\cmd{matplotlib}, com o \emph{backend} \textsc{pgf}\footnote{
\url{matplotlib.org/users/pgf.html}}, são capazes de exportar gráficos de
dados na forma de comandos para \pkg{tikz}\footnote{Você pode se interessar
também pela \textit{package} \texttt{gnuplottex}.}: o resultado pode ser
visto na Figura~\ref{fig:graficos}.

\begin{figure}
  \centering
  \begin{subfigure}[b]{.45\textwidth}
    \input{figuras/gnuplot.tkz} % Exemplo com gnuplot
    %\begin{tikzpicture}
  \sffamily\scriptsize
  \begin{axis}[
    width=2.9in, height=1.9in,
    %title=Algum título,
    ylabel=$f(x)$,
    xlabel=$x$,
    xmin = 1,
    ymin = 0,
    xmax = 12.5,
    ymax = 145,
    ytick distance = 25,
    legend pos = north west,
    legend cell align = left,
  ]

  \addlegendentry{$10x-10$}
  \addplot+ [no marks] table
  {
     1.0   0
     2.0  10
     5.0  40
     7.0  60
    10.0  90
    11.9 109
  };

  \addlegendentry{$x^2-1$}
  \addplot+ [mark = +] table
  {
     1.0   0.0
     2.0   3.0
     3.0   8.0
     4.0  15.0
     6.0  35.0
     8.0  63.0
    11.9 140.6
  };
  \addlegendentry{$25\log_{2}(x)$}
  \addplot+ [error bars/y dir=both, error bars/y explicit,]
      table [y error minus index=2, y error plus index=3,]
  {
     1.0  0.00 0.0 0.0
     1.5 14.62 3.1 6.8
     2.0 25.00 5.3 3.8
     3.0 39.62 3.7 6.2
     5.0 58.05 2.2 5.5
     7.0 70.18 3.9 6.6
     8.0 75.00 7.9 2.3
    11.9 89.32 7.9 3.8
  };
  \end{axis}
\end{tikzpicture}
 % Exemplo com pgfplots
    \caption{\texttt{gnuplot}.\label{fig:gnuplot}}
  \end{subfigure}
  \begin{subfigure}[b]{.5\textwidth}
    %% Creator: Matplotlib, PGF backend
%%
%% To include the figure in your LaTeX document, write
%%   \input{<filename>.pgf}
%%
%% Make sure the required packages are loaded in your preamble
%%   \usepackage{pgf}
%%
%% and, on pdftex
%%   \usepackage[utf8]{inputenc}\DeclareUnicodeCharacter{2212}{-}
%%
%% or, on luatex and xetex
%%   \usepackage{unicode-math}
%%
%% Figures using additional raster images can only be included by \input if
%% they are in the same directory as the main LaTeX file. For loading figures
%% from other directories you can use the `import` package
%%   \usepackage{import}
%%
%% and then include the figures with
%%   \import{<path to file>}{<filename>.pgf}
%%
%% Matplotlib used the following preamble
%%   \usepackage{fontspec}
%%   \setmainfont{DejaVuSerif.ttf}[Path=/usr/share/matplotlib/mpl-data/fonts/ttf/]
%%   \setsansfont{DejaVuSans.ttf}[Path=/usr/share/matplotlib/mpl-data/fonts/ttf/]
%%   \setmonofont{DejaVuSansMono.ttf}[Path=/usr/share/matplotlib/mpl-data/fonts/ttf/]
%%
\begingroup%
\makeatletter%
\begin{pgfpicture}%
\pgfpathrectangle{\pgfpointorigin}{\pgfqpoint{2.900000in}{1.800000in}}%
\pgfusepath{use as bounding box, clip}%
\begin{pgfscope}%
\pgfsetbuttcap%
\pgfsetmiterjoin%
\definecolor{currentfill}{rgb}{1.000000,1.000000,1.000000}%
\pgfsetfillcolor{currentfill}%
\pgfsetlinewidth{0.000000pt}%
\definecolor{currentstroke}{rgb}{1.000000,1.000000,1.000000}%
\pgfsetstrokecolor{currentstroke}%
\pgfsetdash{}{0pt}%
\pgfpathmoveto{\pgfqpoint{0.000000in}{0.000000in}}%
\pgfpathlineto{\pgfqpoint{2.900000in}{0.000000in}}%
\pgfpathlineto{\pgfqpoint{2.900000in}{1.800000in}}%
\pgfpathlineto{\pgfqpoint{0.000000in}{1.800000in}}%
\pgfpathclose%
\pgfusepath{fill}%
\end{pgfscope}%
\begin{pgfscope}%
\pgfsetbuttcap%
\pgfsetmiterjoin%
\definecolor{currentfill}{rgb}{1.000000,1.000000,1.000000}%
\pgfsetfillcolor{currentfill}%
\pgfsetlinewidth{0.000000pt}%
\definecolor{currentstroke}{rgb}{0.000000,0.000000,0.000000}%
\pgfsetstrokecolor{currentstroke}%
\pgfsetstrokeopacity{0.000000}%
\pgfsetdash{}{0pt}%
\pgfpathmoveto{\pgfqpoint{0.616528in}{0.472778in}}%
\pgfpathlineto{\pgfqpoint{2.780000in}{0.472778in}}%
\pgfpathlineto{\pgfqpoint{2.780000in}{1.680000in}}%
\pgfpathlineto{\pgfqpoint{0.616528in}{1.680000in}}%
\pgfpathclose%
\pgfusepath{fill}%
\end{pgfscope}%
\begin{pgfscope}%
\pgfsetbuttcap%
\pgfsetroundjoin%
\definecolor{currentfill}{rgb}{0.000000,0.000000,0.000000}%
\pgfsetfillcolor{currentfill}%
\pgfsetlinewidth{0.803000pt}%
\definecolor{currentstroke}{rgb}{0.000000,0.000000,0.000000}%
\pgfsetstrokecolor{currentstroke}%
\pgfsetdash{}{0pt}%
\pgfsys@defobject{currentmarker}{\pgfqpoint{0.000000in}{-0.048611in}}{\pgfqpoint{0.000000in}{0.000000in}}{%
\pgfpathmoveto{\pgfqpoint{0.000000in}{0.000000in}}%
\pgfpathlineto{\pgfqpoint{0.000000in}{-0.048611in}}%
\pgfusepath{stroke,fill}%
}%
\begin{pgfscope}%
\pgfsys@transformshift{0.804656in}{0.472778in}%
\pgfsys@useobject{currentmarker}{}%
\end{pgfscope}%
\end{pgfscope}%
\begin{pgfscope}%
\definecolor{textcolor}{rgb}{0.000000,0.000000,0.000000}%
\pgfsetstrokecolor{textcolor}%
\pgfsetfillcolor{textcolor}%
\pgftext[x=0.804656in,y=0.375556in,,top]{\color{textcolor}\sffamily\fontsize{8.000000}{9.600000}\selectfont 2}%
\end{pgfscope}%
\begin{pgfscope}%
\pgfsetbuttcap%
\pgfsetroundjoin%
\definecolor{currentfill}{rgb}{0.000000,0.000000,0.000000}%
\pgfsetfillcolor{currentfill}%
\pgfsetlinewidth{0.803000pt}%
\definecolor{currentstroke}{rgb}{0.000000,0.000000,0.000000}%
\pgfsetstrokecolor{currentstroke}%
\pgfsetdash{}{0pt}%
\pgfsys@defobject{currentmarker}{\pgfqpoint{0.000000in}{-0.048611in}}{\pgfqpoint{0.000000in}{0.000000in}}{%
\pgfpathmoveto{\pgfqpoint{0.000000in}{0.000000in}}%
\pgfpathlineto{\pgfqpoint{0.000000in}{-0.048611in}}%
\pgfusepath{stroke,fill}%
}%
\begin{pgfscope}%
\pgfsys@transformshift{1.180912in}{0.472778in}%
\pgfsys@useobject{currentmarker}{}%
\end{pgfscope}%
\end{pgfscope}%
\begin{pgfscope}%
\definecolor{textcolor}{rgb}{0.000000,0.000000,0.000000}%
\pgfsetstrokecolor{textcolor}%
\pgfsetfillcolor{textcolor}%
\pgftext[x=1.180912in,y=0.375556in,,top]{\color{textcolor}\sffamily\fontsize{8.000000}{9.600000}\selectfont 4}%
\end{pgfscope}%
\begin{pgfscope}%
\pgfsetbuttcap%
\pgfsetroundjoin%
\definecolor{currentfill}{rgb}{0.000000,0.000000,0.000000}%
\pgfsetfillcolor{currentfill}%
\pgfsetlinewidth{0.803000pt}%
\definecolor{currentstroke}{rgb}{0.000000,0.000000,0.000000}%
\pgfsetstrokecolor{currentstroke}%
\pgfsetdash{}{0pt}%
\pgfsys@defobject{currentmarker}{\pgfqpoint{0.000000in}{-0.048611in}}{\pgfqpoint{0.000000in}{0.000000in}}{%
\pgfpathmoveto{\pgfqpoint{0.000000in}{0.000000in}}%
\pgfpathlineto{\pgfqpoint{0.000000in}{-0.048611in}}%
\pgfusepath{stroke,fill}%
}%
\begin{pgfscope}%
\pgfsys@transformshift{1.557168in}{0.472778in}%
\pgfsys@useobject{currentmarker}{}%
\end{pgfscope}%
\end{pgfscope}%
\begin{pgfscope}%
\definecolor{textcolor}{rgb}{0.000000,0.000000,0.000000}%
\pgfsetstrokecolor{textcolor}%
\pgfsetfillcolor{textcolor}%
\pgftext[x=1.557168in,y=0.375556in,,top]{\color{textcolor}\sffamily\fontsize{8.000000}{9.600000}\selectfont 6}%
\end{pgfscope}%
\begin{pgfscope}%
\pgfsetbuttcap%
\pgfsetroundjoin%
\definecolor{currentfill}{rgb}{0.000000,0.000000,0.000000}%
\pgfsetfillcolor{currentfill}%
\pgfsetlinewidth{0.803000pt}%
\definecolor{currentstroke}{rgb}{0.000000,0.000000,0.000000}%
\pgfsetstrokecolor{currentstroke}%
\pgfsetdash{}{0pt}%
\pgfsys@defobject{currentmarker}{\pgfqpoint{0.000000in}{-0.048611in}}{\pgfqpoint{0.000000in}{0.000000in}}{%
\pgfpathmoveto{\pgfqpoint{0.000000in}{0.000000in}}%
\pgfpathlineto{\pgfqpoint{0.000000in}{-0.048611in}}%
\pgfusepath{stroke,fill}%
}%
\begin{pgfscope}%
\pgfsys@transformshift{1.933424in}{0.472778in}%
\pgfsys@useobject{currentmarker}{}%
\end{pgfscope}%
\end{pgfscope}%
\begin{pgfscope}%
\definecolor{textcolor}{rgb}{0.000000,0.000000,0.000000}%
\pgfsetstrokecolor{textcolor}%
\pgfsetfillcolor{textcolor}%
\pgftext[x=1.933424in,y=0.375556in,,top]{\color{textcolor}\sffamily\fontsize{8.000000}{9.600000}\selectfont 8}%
\end{pgfscope}%
\begin{pgfscope}%
\pgfsetbuttcap%
\pgfsetroundjoin%
\definecolor{currentfill}{rgb}{0.000000,0.000000,0.000000}%
\pgfsetfillcolor{currentfill}%
\pgfsetlinewidth{0.803000pt}%
\definecolor{currentstroke}{rgb}{0.000000,0.000000,0.000000}%
\pgfsetstrokecolor{currentstroke}%
\pgfsetdash{}{0pt}%
\pgfsys@defobject{currentmarker}{\pgfqpoint{0.000000in}{-0.048611in}}{\pgfqpoint{0.000000in}{0.000000in}}{%
\pgfpathmoveto{\pgfqpoint{0.000000in}{0.000000in}}%
\pgfpathlineto{\pgfqpoint{0.000000in}{-0.048611in}}%
\pgfusepath{stroke,fill}%
}%
\begin{pgfscope}%
\pgfsys@transformshift{2.309680in}{0.472778in}%
\pgfsys@useobject{currentmarker}{}%
\end{pgfscope}%
\end{pgfscope}%
\begin{pgfscope}%
\definecolor{textcolor}{rgb}{0.000000,0.000000,0.000000}%
\pgfsetstrokecolor{textcolor}%
\pgfsetfillcolor{textcolor}%
\pgftext[x=2.309680in,y=0.375556in,,top]{\color{textcolor}\sffamily\fontsize{8.000000}{9.600000}\selectfont 10}%
\end{pgfscope}%
\begin{pgfscope}%
\pgfsetbuttcap%
\pgfsetroundjoin%
\definecolor{currentfill}{rgb}{0.000000,0.000000,0.000000}%
\pgfsetfillcolor{currentfill}%
\pgfsetlinewidth{0.803000pt}%
\definecolor{currentstroke}{rgb}{0.000000,0.000000,0.000000}%
\pgfsetstrokecolor{currentstroke}%
\pgfsetdash{}{0pt}%
\pgfsys@defobject{currentmarker}{\pgfqpoint{0.000000in}{-0.048611in}}{\pgfqpoint{0.000000in}{0.000000in}}{%
\pgfpathmoveto{\pgfqpoint{0.000000in}{0.000000in}}%
\pgfpathlineto{\pgfqpoint{0.000000in}{-0.048611in}}%
\pgfusepath{stroke,fill}%
}%
\begin{pgfscope}%
\pgfsys@transformshift{2.685936in}{0.472778in}%
\pgfsys@useobject{currentmarker}{}%
\end{pgfscope}%
\end{pgfscope}%
\begin{pgfscope}%
\definecolor{textcolor}{rgb}{0.000000,0.000000,0.000000}%
\pgfsetstrokecolor{textcolor}%
\pgfsetfillcolor{textcolor}%
\pgftext[x=2.685936in,y=0.375556in,,top]{\color{textcolor}\sffamily\fontsize{8.000000}{9.600000}\selectfont 12}%
\end{pgfscope}%
\begin{pgfscope}%
\definecolor{textcolor}{rgb}{0.000000,0.000000,0.000000}%
\pgfsetstrokecolor{textcolor}%
\pgfsetfillcolor{textcolor}%
\pgftext[x=1.698264in,y=0.212470in,,top]{\color{textcolor}\sffamily\fontsize{8.000000}{9.600000}\selectfont \(\displaystyle x\)}%
\end{pgfscope}%
\begin{pgfscope}%
\pgfsetbuttcap%
\pgfsetroundjoin%
\definecolor{currentfill}{rgb}{0.000000,0.000000,0.000000}%
\pgfsetfillcolor{currentfill}%
\pgfsetlinewidth{0.803000pt}%
\definecolor{currentstroke}{rgb}{0.000000,0.000000,0.000000}%
\pgfsetstrokecolor{currentstroke}%
\pgfsetdash{}{0pt}%
\pgfsys@defobject{currentmarker}{\pgfqpoint{-0.048611in}{0.000000in}}{\pgfqpoint{-0.000000in}{0.000000in}}{%
\pgfpathmoveto{\pgfqpoint{-0.000000in}{0.000000in}}%
\pgfpathlineto{\pgfqpoint{-0.048611in}{0.000000in}}%
\pgfusepath{stroke,fill}%
}%
\begin{pgfscope}%
\pgfsys@transformshift{0.616528in}{0.472778in}%
\pgfsys@useobject{currentmarker}{}%
\end{pgfscope}%
\end{pgfscope}%
\begin{pgfscope}%
\definecolor{textcolor}{rgb}{0.000000,0.000000,0.000000}%
\pgfsetstrokecolor{textcolor}%
\pgfsetfillcolor{textcolor}%
\pgftext[x=0.448613in, y=0.430569in, left, base]{\color{textcolor}\sffamily\fontsize{8.000000}{9.600000}\selectfont 0}%
\end{pgfscope}%
\begin{pgfscope}%
\pgfsetbuttcap%
\pgfsetroundjoin%
\definecolor{currentfill}{rgb}{0.000000,0.000000,0.000000}%
\pgfsetfillcolor{currentfill}%
\pgfsetlinewidth{0.803000pt}%
\definecolor{currentstroke}{rgb}{0.000000,0.000000,0.000000}%
\pgfsetstrokecolor{currentstroke}%
\pgfsetdash{}{0pt}%
\pgfsys@defobject{currentmarker}{\pgfqpoint{-0.048611in}{0.000000in}}{\pgfqpoint{-0.000000in}{0.000000in}}{%
\pgfpathmoveto{\pgfqpoint{-0.000000in}{0.000000in}}%
\pgfpathlineto{\pgfqpoint{-0.048611in}{0.000000in}}%
\pgfusepath{stroke,fill}%
}%
\begin{pgfscope}%
\pgfsys@transformshift{0.616528in}{0.680920in}%
\pgfsys@useobject{currentmarker}{}%
\end{pgfscope}%
\end{pgfscope}%
\begin{pgfscope}%
\definecolor{textcolor}{rgb}{0.000000,0.000000,0.000000}%
\pgfsetstrokecolor{textcolor}%
\pgfsetfillcolor{textcolor}%
\pgftext[x=0.377921in, y=0.638710in, left, base]{\color{textcolor}\sffamily\fontsize{8.000000}{9.600000}\selectfont 25}%
\end{pgfscope}%
\begin{pgfscope}%
\pgfsetbuttcap%
\pgfsetroundjoin%
\definecolor{currentfill}{rgb}{0.000000,0.000000,0.000000}%
\pgfsetfillcolor{currentfill}%
\pgfsetlinewidth{0.803000pt}%
\definecolor{currentstroke}{rgb}{0.000000,0.000000,0.000000}%
\pgfsetstrokecolor{currentstroke}%
\pgfsetdash{}{0pt}%
\pgfsys@defobject{currentmarker}{\pgfqpoint{-0.048611in}{0.000000in}}{\pgfqpoint{-0.000000in}{0.000000in}}{%
\pgfpathmoveto{\pgfqpoint{-0.000000in}{0.000000in}}%
\pgfpathlineto{\pgfqpoint{-0.048611in}{0.000000in}}%
\pgfusepath{stroke,fill}%
}%
\begin{pgfscope}%
\pgfsys@transformshift{0.616528in}{0.889061in}%
\pgfsys@useobject{currentmarker}{}%
\end{pgfscope}%
\end{pgfscope}%
\begin{pgfscope}%
\definecolor{textcolor}{rgb}{0.000000,0.000000,0.000000}%
\pgfsetstrokecolor{textcolor}%
\pgfsetfillcolor{textcolor}%
\pgftext[x=0.377921in, y=0.846852in, left, base]{\color{textcolor}\sffamily\fontsize{8.000000}{9.600000}\selectfont 50}%
\end{pgfscope}%
\begin{pgfscope}%
\pgfsetbuttcap%
\pgfsetroundjoin%
\definecolor{currentfill}{rgb}{0.000000,0.000000,0.000000}%
\pgfsetfillcolor{currentfill}%
\pgfsetlinewidth{0.803000pt}%
\definecolor{currentstroke}{rgb}{0.000000,0.000000,0.000000}%
\pgfsetstrokecolor{currentstroke}%
\pgfsetdash{}{0pt}%
\pgfsys@defobject{currentmarker}{\pgfqpoint{-0.048611in}{0.000000in}}{\pgfqpoint{-0.000000in}{0.000000in}}{%
\pgfpathmoveto{\pgfqpoint{-0.000000in}{0.000000in}}%
\pgfpathlineto{\pgfqpoint{-0.048611in}{0.000000in}}%
\pgfusepath{stroke,fill}%
}%
\begin{pgfscope}%
\pgfsys@transformshift{0.616528in}{1.097203in}%
\pgfsys@useobject{currentmarker}{}%
\end{pgfscope}%
\end{pgfscope}%
\begin{pgfscope}%
\definecolor{textcolor}{rgb}{0.000000,0.000000,0.000000}%
\pgfsetstrokecolor{textcolor}%
\pgfsetfillcolor{textcolor}%
\pgftext[x=0.377921in, y=1.054994in, left, base]{\color{textcolor}\sffamily\fontsize{8.000000}{9.600000}\selectfont 75}%
\end{pgfscope}%
\begin{pgfscope}%
\pgfsetbuttcap%
\pgfsetroundjoin%
\definecolor{currentfill}{rgb}{0.000000,0.000000,0.000000}%
\pgfsetfillcolor{currentfill}%
\pgfsetlinewidth{0.803000pt}%
\definecolor{currentstroke}{rgb}{0.000000,0.000000,0.000000}%
\pgfsetstrokecolor{currentstroke}%
\pgfsetdash{}{0pt}%
\pgfsys@defobject{currentmarker}{\pgfqpoint{-0.048611in}{0.000000in}}{\pgfqpoint{-0.000000in}{0.000000in}}{%
\pgfpathmoveto{\pgfqpoint{-0.000000in}{0.000000in}}%
\pgfpathlineto{\pgfqpoint{-0.048611in}{0.000000in}}%
\pgfusepath{stroke,fill}%
}%
\begin{pgfscope}%
\pgfsys@transformshift{0.616528in}{1.305345in}%
\pgfsys@useobject{currentmarker}{}%
\end{pgfscope}%
\end{pgfscope}%
\begin{pgfscope}%
\definecolor{textcolor}{rgb}{0.000000,0.000000,0.000000}%
\pgfsetstrokecolor{textcolor}%
\pgfsetfillcolor{textcolor}%
\pgftext[x=0.307229in, y=1.263136in, left, base]{\color{textcolor}\sffamily\fontsize{8.000000}{9.600000}\selectfont 100}%
\end{pgfscope}%
\begin{pgfscope}%
\pgfsetbuttcap%
\pgfsetroundjoin%
\definecolor{currentfill}{rgb}{0.000000,0.000000,0.000000}%
\pgfsetfillcolor{currentfill}%
\pgfsetlinewidth{0.803000pt}%
\definecolor{currentstroke}{rgb}{0.000000,0.000000,0.000000}%
\pgfsetstrokecolor{currentstroke}%
\pgfsetdash{}{0pt}%
\pgfsys@defobject{currentmarker}{\pgfqpoint{-0.048611in}{0.000000in}}{\pgfqpoint{-0.000000in}{0.000000in}}{%
\pgfpathmoveto{\pgfqpoint{-0.000000in}{0.000000in}}%
\pgfpathlineto{\pgfqpoint{-0.048611in}{0.000000in}}%
\pgfusepath{stroke,fill}%
}%
\begin{pgfscope}%
\pgfsys@transformshift{0.616528in}{1.513487in}%
\pgfsys@useobject{currentmarker}{}%
\end{pgfscope}%
\end{pgfscope}%
\begin{pgfscope}%
\definecolor{textcolor}{rgb}{0.000000,0.000000,0.000000}%
\pgfsetstrokecolor{textcolor}%
\pgfsetfillcolor{textcolor}%
\pgftext[x=0.307229in, y=1.471277in, left, base]{\color{textcolor}\sffamily\fontsize{8.000000}{9.600000}\selectfont 125}%
\end{pgfscope}%
\begin{pgfscope}%
\definecolor{textcolor}{rgb}{0.000000,0.000000,0.000000}%
\pgfsetstrokecolor{textcolor}%
\pgfsetfillcolor{textcolor}%
\pgftext[x=0.251673in,y=1.076389in,,bottom,rotate=90.000000]{\color{textcolor}\sffamily\fontsize{8.000000}{9.600000}\selectfont \(\displaystyle f(x)\)}%
\end{pgfscope}%
\begin{pgfscope}%
\pgfpathrectangle{\pgfqpoint{0.616528in}{0.472778in}}{\pgfqpoint{2.163472in}{1.207222in}}%
\pgfusepath{clip}%
\pgfsetbuttcap%
\pgfsetroundjoin%
\pgfsetlinewidth{0.501875pt}%
\definecolor{currentstroke}{rgb}{0.172549,0.627451,0.172549}%
\pgfsetstrokecolor{currentstroke}%
\pgfsetdash{}{0pt}%
\pgfpathmoveto{\pgfqpoint{0.616528in}{0.472778in}}%
\pgfpathlineto{\pgfqpoint{0.616528in}{0.472778in}}%
\pgfusepath{stroke}%
\end{pgfscope}%
\begin{pgfscope}%
\pgfpathrectangle{\pgfqpoint{0.616528in}{0.472778in}}{\pgfqpoint{2.163472in}{1.207222in}}%
\pgfusepath{clip}%
\pgfsetbuttcap%
\pgfsetroundjoin%
\pgfsetlinewidth{0.501875pt}%
\definecolor{currentstroke}{rgb}{0.172549,0.627451,0.172549}%
\pgfsetstrokecolor{currentstroke}%
\pgfsetdash{}{0pt}%
\pgfpathmoveto{\pgfqpoint{0.710592in}{0.568690in}}%
\pgfpathlineto{\pgfqpoint{0.710592in}{0.651114in}}%
\pgfusepath{stroke}%
\end{pgfscope}%
\begin{pgfscope}%
\pgfpathrectangle{\pgfqpoint{0.616528in}{0.472778in}}{\pgfqpoint{2.163472in}{1.207222in}}%
\pgfusepath{clip}%
\pgfsetbuttcap%
\pgfsetroundjoin%
\pgfsetlinewidth{0.501875pt}%
\definecolor{currentstroke}{rgb}{0.172549,0.627451,0.172549}%
\pgfsetstrokecolor{currentstroke}%
\pgfsetdash{}{0pt}%
\pgfpathmoveto{\pgfqpoint{0.804656in}{0.636793in}}%
\pgfpathlineto{\pgfqpoint{0.804656in}{0.712557in}}%
\pgfusepath{stroke}%
\end{pgfscope}%
\begin{pgfscope}%
\pgfpathrectangle{\pgfqpoint{0.616528in}{0.472778in}}{\pgfqpoint{2.163472in}{1.207222in}}%
\pgfusepath{clip}%
\pgfsetbuttcap%
\pgfsetroundjoin%
\pgfsetlinewidth{0.501875pt}%
\definecolor{currentstroke}{rgb}{0.172549,0.627451,0.172549}%
\pgfsetstrokecolor{currentstroke}%
\pgfsetdash{}{0pt}%
\pgfpathmoveto{\pgfqpoint{0.992784in}{0.771836in}}%
\pgfpathlineto{\pgfqpoint{0.992784in}{0.854260in}}%
\pgfusepath{stroke}%
\end{pgfscope}%
\begin{pgfscope}%
\pgfpathrectangle{\pgfqpoint{0.616528in}{0.472778in}}{\pgfqpoint{2.163472in}{1.207222in}}%
\pgfusepath{clip}%
\pgfsetbuttcap%
\pgfsetroundjoin%
\pgfsetlinewidth{0.501875pt}%
\definecolor{currentstroke}{rgb}{0.172549,0.627451,0.172549}%
\pgfsetstrokecolor{currentstroke}%
\pgfsetdash{}{0pt}%
\pgfpathmoveto{\pgfqpoint{1.369040in}{0.937766in}}%
\pgfpathlineto{\pgfqpoint{1.369040in}{1.001874in}}%
\pgfusepath{stroke}%
\end{pgfscope}%
\begin{pgfscope}%
\pgfpathrectangle{\pgfqpoint{0.616528in}{0.472778in}}{\pgfqpoint{2.163472in}{1.207222in}}%
\pgfusepath{clip}%
\pgfsetbuttcap%
\pgfsetroundjoin%
\pgfsetlinewidth{0.501875pt}%
\definecolor{currentstroke}{rgb}{0.172549,0.627451,0.172549}%
\pgfsetstrokecolor{currentstroke}%
\pgfsetdash{}{0pt}%
\pgfpathmoveto{\pgfqpoint{1.745296in}{1.024603in}}%
\pgfpathlineto{\pgfqpoint{1.745296in}{1.112023in}}%
\pgfusepath{stroke}%
\end{pgfscope}%
\begin{pgfscope}%
\pgfpathrectangle{\pgfqpoint{0.616528in}{0.472778in}}{\pgfqpoint{2.163472in}{1.207222in}}%
\pgfusepath{clip}%
\pgfsetbuttcap%
\pgfsetroundjoin%
\pgfsetlinewidth{0.501875pt}%
\definecolor{currentstroke}{rgb}{0.172549,0.627451,0.172549}%
\pgfsetstrokecolor{currentstroke}%
\pgfsetdash{}{0pt}%
\pgfpathmoveto{\pgfqpoint{1.933424in}{1.031430in}}%
\pgfpathlineto{\pgfqpoint{1.933424in}{1.116352in}}%
\pgfusepath{stroke}%
\end{pgfscope}%
\begin{pgfscope}%
\pgfpathrectangle{\pgfqpoint{0.616528in}{0.472778in}}{\pgfqpoint{2.163472in}{1.207222in}}%
\pgfusepath{clip}%
\pgfsetbuttcap%
\pgfsetroundjoin%
\pgfsetlinewidth{0.501875pt}%
\definecolor{currentstroke}{rgb}{0.172549,0.627451,0.172549}%
\pgfsetstrokecolor{currentstroke}%
\pgfsetdash{}{0pt}%
\pgfpathmoveto{\pgfqpoint{2.667123in}{1.150654in}}%
\pgfpathlineto{\pgfqpoint{2.667123in}{1.248064in}}%
\pgfusepath{stroke}%
\end{pgfscope}%
\begin{pgfscope}%
\pgfpathrectangle{\pgfqpoint{0.616528in}{0.472778in}}{\pgfqpoint{2.163472in}{1.207222in}}%
\pgfusepath{clip}%
\pgfsetrectcap%
\pgfsetroundjoin%
\pgfsetlinewidth{1.003750pt}%
\definecolor{currentstroke}{rgb}{0.121569,0.466667,0.705882}%
\pgfsetstrokecolor{currentstroke}%
\pgfsetdash{}{0pt}%
\pgfpathmoveto{\pgfqpoint{0.616528in}{0.472778in}}%
\pgfpathlineto{\pgfqpoint{0.635341in}{0.481103in}}%
\pgfpathlineto{\pgfqpoint{0.654153in}{0.489429in}}%
\pgfpathlineto{\pgfqpoint{0.672966in}{0.497755in}}%
\pgfpathlineto{\pgfqpoint{0.691779in}{0.506080in}}%
\pgfpathlineto{\pgfqpoint{0.710592in}{0.514406in}}%
\pgfpathlineto{\pgfqpoint{0.729405in}{0.522732in}}%
\pgfpathlineto{\pgfqpoint{0.748217in}{0.531057in}}%
\pgfpathlineto{\pgfqpoint{0.767030in}{0.539383in}}%
\pgfpathlineto{\pgfqpoint{0.785843in}{0.547709in}}%
\pgfpathlineto{\pgfqpoint{0.804656in}{0.556034in}}%
\pgfpathlineto{\pgfqpoint{0.823469in}{0.564360in}}%
\pgfpathlineto{\pgfqpoint{0.842281in}{0.572686in}}%
\pgfpathlineto{\pgfqpoint{0.861094in}{0.581011in}}%
\pgfpathlineto{\pgfqpoint{0.879907in}{0.589337in}}%
\pgfpathlineto{\pgfqpoint{0.898720in}{0.597663in}}%
\pgfpathlineto{\pgfqpoint{0.917533in}{0.605989in}}%
\pgfpathlineto{\pgfqpoint{0.936345in}{0.614314in}}%
\pgfpathlineto{\pgfqpoint{0.955158in}{0.622640in}}%
\pgfpathlineto{\pgfqpoint{0.973971in}{0.630966in}}%
\pgfpathlineto{\pgfqpoint{0.992784in}{0.639291in}}%
\pgfpathlineto{\pgfqpoint{1.011597in}{0.647617in}}%
\pgfpathlineto{\pgfqpoint{1.030409in}{0.655943in}}%
\pgfpathlineto{\pgfqpoint{1.049222in}{0.664268in}}%
\pgfpathlineto{\pgfqpoint{1.068035in}{0.672594in}}%
\pgfpathlineto{\pgfqpoint{1.086848in}{0.680920in}}%
\pgfpathlineto{\pgfqpoint{1.105661in}{0.689245in}}%
\pgfpathlineto{\pgfqpoint{1.124473in}{0.697571in}}%
\pgfpathlineto{\pgfqpoint{1.143286in}{0.705897in}}%
\pgfpathlineto{\pgfqpoint{1.162099in}{0.714222in}}%
\pgfpathlineto{\pgfqpoint{1.180912in}{0.722548in}}%
\pgfpathlineto{\pgfqpoint{1.199725in}{0.730874in}}%
\pgfpathlineto{\pgfqpoint{1.218537in}{0.739199in}}%
\pgfpathlineto{\pgfqpoint{1.237350in}{0.747525in}}%
\pgfpathlineto{\pgfqpoint{1.256163in}{0.755851in}}%
\pgfpathlineto{\pgfqpoint{1.274976in}{0.764176in}}%
\pgfpathlineto{\pgfqpoint{1.293789in}{0.772502in}}%
\pgfpathlineto{\pgfqpoint{1.312601in}{0.780828in}}%
\pgfpathlineto{\pgfqpoint{1.331414in}{0.789153in}}%
\pgfpathlineto{\pgfqpoint{1.350227in}{0.797479in}}%
\pgfpathlineto{\pgfqpoint{1.369040in}{0.805805in}}%
\pgfpathlineto{\pgfqpoint{1.387853in}{0.814130in}}%
\pgfpathlineto{\pgfqpoint{1.406665in}{0.822456in}}%
\pgfpathlineto{\pgfqpoint{1.425478in}{0.830782in}}%
\pgfpathlineto{\pgfqpoint{1.444291in}{0.839107in}}%
\pgfpathlineto{\pgfqpoint{1.463104in}{0.847433in}}%
\pgfpathlineto{\pgfqpoint{1.481917in}{0.855759in}}%
\pgfpathlineto{\pgfqpoint{1.500729in}{0.864084in}}%
\pgfpathlineto{\pgfqpoint{1.519542in}{0.872410in}}%
\pgfpathlineto{\pgfqpoint{1.538355in}{0.880736in}}%
\pgfpathlineto{\pgfqpoint{1.557168in}{0.889061in}}%
\pgfpathlineto{\pgfqpoint{1.575981in}{0.897387in}}%
\pgfpathlineto{\pgfqpoint{1.594793in}{0.905713in}}%
\pgfpathlineto{\pgfqpoint{1.613606in}{0.914038in}}%
\pgfpathlineto{\pgfqpoint{1.632419in}{0.922364in}}%
\pgfpathlineto{\pgfqpoint{1.651232in}{0.930690in}}%
\pgfpathlineto{\pgfqpoint{1.670045in}{0.939015in}}%
\pgfpathlineto{\pgfqpoint{1.688857in}{0.947341in}}%
\pgfpathlineto{\pgfqpoint{1.707670in}{0.955667in}}%
\pgfpathlineto{\pgfqpoint{1.726483in}{0.963992in}}%
\pgfpathlineto{\pgfqpoint{1.745296in}{0.972318in}}%
\pgfpathlineto{\pgfqpoint{1.764109in}{0.980644in}}%
\pgfpathlineto{\pgfqpoint{1.782921in}{0.988969in}}%
\pgfpathlineto{\pgfqpoint{1.801734in}{0.997295in}}%
\pgfpathlineto{\pgfqpoint{1.820547in}{1.005621in}}%
\pgfpathlineto{\pgfqpoint{1.839360in}{1.013946in}}%
\pgfpathlineto{\pgfqpoint{1.858173in}{1.022272in}}%
\pgfpathlineto{\pgfqpoint{1.876986in}{1.030598in}}%
\pgfpathlineto{\pgfqpoint{1.895798in}{1.038923in}}%
\pgfpathlineto{\pgfqpoint{1.914611in}{1.047249in}}%
\pgfpathlineto{\pgfqpoint{1.933424in}{1.055575in}}%
\pgfpathlineto{\pgfqpoint{1.952237in}{1.063900in}}%
\pgfpathlineto{\pgfqpoint{1.971050in}{1.072226in}}%
\pgfpathlineto{\pgfqpoint{1.989862in}{1.080552in}}%
\pgfpathlineto{\pgfqpoint{2.008675in}{1.088877in}}%
\pgfpathlineto{\pgfqpoint{2.027488in}{1.097203in}}%
\pgfpathlineto{\pgfqpoint{2.046301in}{1.105529in}}%
\pgfpathlineto{\pgfqpoint{2.065114in}{1.113854in}}%
\pgfpathlineto{\pgfqpoint{2.083926in}{1.122180in}}%
\pgfpathlineto{\pgfqpoint{2.102739in}{1.130506in}}%
\pgfpathlineto{\pgfqpoint{2.121552in}{1.138831in}}%
\pgfpathlineto{\pgfqpoint{2.140365in}{1.147157in}}%
\pgfpathlineto{\pgfqpoint{2.159178in}{1.155483in}}%
\pgfpathlineto{\pgfqpoint{2.177990in}{1.163808in}}%
\pgfpathlineto{\pgfqpoint{2.196803in}{1.172134in}}%
\pgfpathlineto{\pgfqpoint{2.215616in}{1.180460in}}%
\pgfpathlineto{\pgfqpoint{2.234429in}{1.188785in}}%
\pgfpathlineto{\pgfqpoint{2.253242in}{1.197111in}}%
\pgfpathlineto{\pgfqpoint{2.272054in}{1.205437in}}%
\pgfpathlineto{\pgfqpoint{2.290867in}{1.213762in}}%
\pgfpathlineto{\pgfqpoint{2.309680in}{1.222088in}}%
\pgfpathlineto{\pgfqpoint{2.328493in}{1.230414in}}%
\pgfpathlineto{\pgfqpoint{2.347306in}{1.238739in}}%
\pgfpathlineto{\pgfqpoint{2.366118in}{1.247065in}}%
\pgfpathlineto{\pgfqpoint{2.384931in}{1.255391in}}%
\pgfpathlineto{\pgfqpoint{2.403744in}{1.263716in}}%
\pgfpathlineto{\pgfqpoint{2.422557in}{1.272042in}}%
\pgfpathlineto{\pgfqpoint{2.441370in}{1.280368in}}%
\pgfpathlineto{\pgfqpoint{2.460182in}{1.288693in}}%
\pgfpathlineto{\pgfqpoint{2.478995in}{1.297019in}}%
\pgfpathlineto{\pgfqpoint{2.497808in}{1.305345in}}%
\pgfpathlineto{\pgfqpoint{2.516621in}{1.313670in}}%
\pgfpathlineto{\pgfqpoint{2.535434in}{1.321996in}}%
\pgfpathlineto{\pgfqpoint{2.554246in}{1.330322in}}%
\pgfpathlineto{\pgfqpoint{2.573059in}{1.338648in}}%
\pgfpathlineto{\pgfqpoint{2.591872in}{1.346973in}}%
\pgfpathlineto{\pgfqpoint{2.610685in}{1.355299in}}%
\pgfpathlineto{\pgfqpoint{2.629498in}{1.363625in}}%
\pgfpathlineto{\pgfqpoint{2.648310in}{1.371950in}}%
\pgfpathlineto{\pgfqpoint{2.667123in}{1.380276in}}%
\pgfusepath{stroke}%
\end{pgfscope}%
\begin{pgfscope}%
\pgfpathrectangle{\pgfqpoint{0.616528in}{0.472778in}}{\pgfqpoint{2.163472in}{1.207222in}}%
\pgfusepath{clip}%
\pgfsetrectcap%
\pgfsetroundjoin%
\pgfsetlinewidth{1.003750pt}%
\definecolor{currentstroke}{rgb}{1.000000,0.498039,0.054902}%
\pgfsetstrokecolor{currentstroke}%
\pgfsetdash{}{0pt}%
\pgfpathmoveto{\pgfqpoint{0.616528in}{0.472778in}}%
\pgfpathlineto{\pgfqpoint{0.804656in}{0.497755in}}%
\pgfpathlineto{\pgfqpoint{0.992784in}{0.539383in}}%
\pgfpathlineto{\pgfqpoint{1.180912in}{0.597663in}}%
\pgfpathlineto{\pgfqpoint{1.557168in}{0.764176in}}%
\pgfpathlineto{\pgfqpoint{1.933424in}{0.997295in}}%
\pgfpathlineto{\pgfqpoint{2.667123in}{1.643450in}}%
\pgfusepath{stroke}%
\end{pgfscope}%
\begin{pgfscope}%
\pgfpathrectangle{\pgfqpoint{0.616528in}{0.472778in}}{\pgfqpoint{2.163472in}{1.207222in}}%
\pgfusepath{clip}%
\pgfsetbuttcap%
\pgfsetroundjoin%
\definecolor{currentfill}{rgb}{1.000000,0.498039,0.054902}%
\pgfsetfillcolor{currentfill}%
\pgfsetlinewidth{1.003750pt}%
\definecolor{currentstroke}{rgb}{1.000000,0.498039,0.054902}%
\pgfsetstrokecolor{currentstroke}%
\pgfsetdash{}{0pt}%
\pgfsys@defobject{currentmarker}{\pgfqpoint{-0.034722in}{-0.034722in}}{\pgfqpoint{0.034722in}{0.034722in}}{%
\pgfpathmoveto{\pgfqpoint{-0.034722in}{0.000000in}}%
\pgfpathlineto{\pgfqpoint{0.034722in}{0.000000in}}%
\pgfpathmoveto{\pgfqpoint{0.000000in}{-0.034722in}}%
\pgfpathlineto{\pgfqpoint{0.000000in}{0.034722in}}%
\pgfusepath{stroke,fill}%
}%
\begin{pgfscope}%
\pgfsys@transformshift{0.616528in}{0.472778in}%
\pgfsys@useobject{currentmarker}{}%
\end{pgfscope}%
\begin{pgfscope}%
\pgfsys@transformshift{0.804656in}{0.497755in}%
\pgfsys@useobject{currentmarker}{}%
\end{pgfscope}%
\begin{pgfscope}%
\pgfsys@transformshift{0.992784in}{0.539383in}%
\pgfsys@useobject{currentmarker}{}%
\end{pgfscope}%
\begin{pgfscope}%
\pgfsys@transformshift{1.180912in}{0.597663in}%
\pgfsys@useobject{currentmarker}{}%
\end{pgfscope}%
\begin{pgfscope}%
\pgfsys@transformshift{1.557168in}{0.764176in}%
\pgfsys@useobject{currentmarker}{}%
\end{pgfscope}%
\begin{pgfscope}%
\pgfsys@transformshift{1.933424in}{0.997295in}%
\pgfsys@useobject{currentmarker}{}%
\end{pgfscope}%
\begin{pgfscope}%
\pgfsys@transformshift{2.667123in}{1.643450in}%
\pgfsys@useobject{currentmarker}{}%
\end{pgfscope}%
\end{pgfscope}%
\begin{pgfscope}%
\pgfpathrectangle{\pgfqpoint{0.616528in}{0.472778in}}{\pgfqpoint{2.163472in}{1.207222in}}%
\pgfusepath{clip}%
\pgfsetbuttcap%
\pgfsetroundjoin%
\definecolor{currentfill}{rgb}{0.172549,0.627451,0.172549}%
\pgfsetfillcolor{currentfill}%
\pgfsetlinewidth{0.501875pt}%
\definecolor{currentstroke}{rgb}{0.172549,0.627451,0.172549}%
\pgfsetstrokecolor{currentstroke}%
\pgfsetdash{}{0pt}%
\pgfsys@defobject{currentmarker}{\pgfqpoint{-0.027778in}{-0.000000in}}{\pgfqpoint{0.027778in}{0.000000in}}{%
\pgfpathmoveto{\pgfqpoint{0.027778in}{-0.000000in}}%
\pgfpathlineto{\pgfqpoint{-0.027778in}{0.000000in}}%
\pgfusepath{stroke,fill}%
}%
\begin{pgfscope}%
\pgfsys@transformshift{0.616528in}{0.472778in}%
\pgfsys@useobject{currentmarker}{}%
\end{pgfscope}%
\begin{pgfscope}%
\pgfsys@transformshift{0.710592in}{0.568690in}%
\pgfsys@useobject{currentmarker}{}%
\end{pgfscope}%
\begin{pgfscope}%
\pgfsys@transformshift{0.804656in}{0.636793in}%
\pgfsys@useobject{currentmarker}{}%
\end{pgfscope}%
\begin{pgfscope}%
\pgfsys@transformshift{0.992784in}{0.771836in}%
\pgfsys@useobject{currentmarker}{}%
\end{pgfscope}%
\begin{pgfscope}%
\pgfsys@transformshift{1.369040in}{0.937766in}%
\pgfsys@useobject{currentmarker}{}%
\end{pgfscope}%
\begin{pgfscope}%
\pgfsys@transformshift{1.745296in}{1.024603in}%
\pgfsys@useobject{currentmarker}{}%
\end{pgfscope}%
\begin{pgfscope}%
\pgfsys@transformshift{1.933424in}{1.031430in}%
\pgfsys@useobject{currentmarker}{}%
\end{pgfscope}%
\begin{pgfscope}%
\pgfsys@transformshift{2.667123in}{1.150654in}%
\pgfsys@useobject{currentmarker}{}%
\end{pgfscope}%
\end{pgfscope}%
\begin{pgfscope}%
\pgfpathrectangle{\pgfqpoint{0.616528in}{0.472778in}}{\pgfqpoint{2.163472in}{1.207222in}}%
\pgfusepath{clip}%
\pgfsetbuttcap%
\pgfsetroundjoin%
\definecolor{currentfill}{rgb}{0.172549,0.627451,0.172549}%
\pgfsetfillcolor{currentfill}%
\pgfsetlinewidth{0.501875pt}%
\definecolor{currentstroke}{rgb}{0.172549,0.627451,0.172549}%
\pgfsetstrokecolor{currentstroke}%
\pgfsetdash{}{0pt}%
\pgfsys@defobject{currentmarker}{\pgfqpoint{-0.027778in}{-0.000000in}}{\pgfqpoint{0.027778in}{0.000000in}}{%
\pgfpathmoveto{\pgfqpoint{0.027778in}{-0.000000in}}%
\pgfpathlineto{\pgfqpoint{-0.027778in}{0.000000in}}%
\pgfusepath{stroke,fill}%
}%
\begin{pgfscope}%
\pgfsys@transformshift{0.616528in}{0.472778in}%
\pgfsys@useobject{currentmarker}{}%
\end{pgfscope}%
\begin{pgfscope}%
\pgfsys@transformshift{0.710592in}{0.651114in}%
\pgfsys@useobject{currentmarker}{}%
\end{pgfscope}%
\begin{pgfscope}%
\pgfsys@transformshift{0.804656in}{0.712557in}%
\pgfsys@useobject{currentmarker}{}%
\end{pgfscope}%
\begin{pgfscope}%
\pgfsys@transformshift{0.992784in}{0.854260in}%
\pgfsys@useobject{currentmarker}{}%
\end{pgfscope}%
\begin{pgfscope}%
\pgfsys@transformshift{1.369040in}{1.001874in}%
\pgfsys@useobject{currentmarker}{}%
\end{pgfscope}%
\begin{pgfscope}%
\pgfsys@transformshift{1.745296in}{1.112023in}%
\pgfsys@useobject{currentmarker}{}%
\end{pgfscope}%
\begin{pgfscope}%
\pgfsys@transformshift{1.933424in}{1.116352in}%
\pgfsys@useobject{currentmarker}{}%
\end{pgfscope}%
\begin{pgfscope}%
\pgfsys@transformshift{2.667123in}{1.248064in}%
\pgfsys@useobject{currentmarker}{}%
\end{pgfscope}%
\end{pgfscope}%
\begin{pgfscope}%
\pgfpathrectangle{\pgfqpoint{0.616528in}{0.472778in}}{\pgfqpoint{2.163472in}{1.207222in}}%
\pgfusepath{clip}%
\pgfsetrectcap%
\pgfsetroundjoin%
\pgfsetlinewidth{1.003750pt}%
\definecolor{currentstroke}{rgb}{0.172549,0.627451,0.172549}%
\pgfsetstrokecolor{currentstroke}%
\pgfsetdash{}{0pt}%
\pgfpathmoveto{\pgfqpoint{0.616528in}{0.472778in}}%
\pgfpathlineto{\pgfqpoint{0.710592in}{0.594499in}}%
\pgfpathlineto{\pgfqpoint{0.804656in}{0.680920in}}%
\pgfpathlineto{\pgfqpoint{0.992784in}{0.802641in}}%
\pgfpathlineto{\pgfqpoint{1.369040in}{0.956083in}}%
\pgfpathlineto{\pgfqpoint{1.745296in}{1.057073in}}%
\pgfpathlineto{\pgfqpoint{1.933424in}{1.097203in}}%
\pgfpathlineto{\pgfqpoint{2.667123in}{1.216427in}}%
\pgfusepath{stroke}%
\end{pgfscope}%
\begin{pgfscope}%
\pgfpathrectangle{\pgfqpoint{0.616528in}{0.472778in}}{\pgfqpoint{2.163472in}{1.207222in}}%
\pgfusepath{clip}%
\pgfsetbuttcap%
\pgfsetroundjoin%
\definecolor{currentfill}{rgb}{0.172549,0.627451,0.172549}%
\pgfsetfillcolor{currentfill}%
\pgfsetlinewidth{0.501875pt}%
\definecolor{currentstroke}{rgb}{0.172549,0.627451,0.172549}%
\pgfsetstrokecolor{currentstroke}%
\pgfsetdash{}{0pt}%
\pgfsys@defobject{currentmarker}{\pgfqpoint{-0.017361in}{-0.017361in}}{\pgfqpoint{0.017361in}{0.017361in}}{%
\pgfpathmoveto{\pgfqpoint{0.000000in}{-0.017361in}}%
\pgfpathcurveto{\pgfqpoint{0.004604in}{-0.017361in}}{\pgfqpoint{0.009020in}{-0.015532in}}{\pgfqpoint{0.012276in}{-0.012276in}}%
\pgfpathcurveto{\pgfqpoint{0.015532in}{-0.009020in}}{\pgfqpoint{0.017361in}{-0.004604in}}{\pgfqpoint{0.017361in}{0.000000in}}%
\pgfpathcurveto{\pgfqpoint{0.017361in}{0.004604in}}{\pgfqpoint{0.015532in}{0.009020in}}{\pgfqpoint{0.012276in}{0.012276in}}%
\pgfpathcurveto{\pgfqpoint{0.009020in}{0.015532in}}{\pgfqpoint{0.004604in}{0.017361in}}{\pgfqpoint{0.000000in}{0.017361in}}%
\pgfpathcurveto{\pgfqpoint{-0.004604in}{0.017361in}}{\pgfqpoint{-0.009020in}{0.015532in}}{\pgfqpoint{-0.012276in}{0.012276in}}%
\pgfpathcurveto{\pgfqpoint{-0.015532in}{0.009020in}}{\pgfqpoint{-0.017361in}{0.004604in}}{\pgfqpoint{-0.017361in}{0.000000in}}%
\pgfpathcurveto{\pgfqpoint{-0.017361in}{-0.004604in}}{\pgfqpoint{-0.015532in}{-0.009020in}}{\pgfqpoint{-0.012276in}{-0.012276in}}%
\pgfpathcurveto{\pgfqpoint{-0.009020in}{-0.015532in}}{\pgfqpoint{-0.004604in}{-0.017361in}}{\pgfqpoint{0.000000in}{-0.017361in}}%
\pgfpathclose%
\pgfusepath{stroke,fill}%
}%
\begin{pgfscope}%
\pgfsys@transformshift{0.616528in}{0.472778in}%
\pgfsys@useobject{currentmarker}{}%
\end{pgfscope}%
\begin{pgfscope}%
\pgfsys@transformshift{0.710592in}{0.594499in}%
\pgfsys@useobject{currentmarker}{}%
\end{pgfscope}%
\begin{pgfscope}%
\pgfsys@transformshift{0.804656in}{0.680920in}%
\pgfsys@useobject{currentmarker}{}%
\end{pgfscope}%
\begin{pgfscope}%
\pgfsys@transformshift{0.992784in}{0.802641in}%
\pgfsys@useobject{currentmarker}{}%
\end{pgfscope}%
\begin{pgfscope}%
\pgfsys@transformshift{1.369040in}{0.956083in}%
\pgfsys@useobject{currentmarker}{}%
\end{pgfscope}%
\begin{pgfscope}%
\pgfsys@transformshift{1.745296in}{1.057073in}%
\pgfsys@useobject{currentmarker}{}%
\end{pgfscope}%
\begin{pgfscope}%
\pgfsys@transformshift{1.933424in}{1.097203in}%
\pgfsys@useobject{currentmarker}{}%
\end{pgfscope}%
\begin{pgfscope}%
\pgfsys@transformshift{2.667123in}{1.216427in}%
\pgfsys@useobject{currentmarker}{}%
\end{pgfscope}%
\end{pgfscope}%
\begin{pgfscope}%
\pgfsetrectcap%
\pgfsetmiterjoin%
\pgfsetlinewidth{0.803000pt}%
\definecolor{currentstroke}{rgb}{0.000000,0.000000,0.000000}%
\pgfsetstrokecolor{currentstroke}%
\pgfsetdash{}{0pt}%
\pgfpathmoveto{\pgfqpoint{0.616528in}{0.472778in}}%
\pgfpathlineto{\pgfqpoint{0.616528in}{1.680000in}}%
\pgfusepath{stroke}%
\end{pgfscope}%
\begin{pgfscope}%
\pgfsetrectcap%
\pgfsetmiterjoin%
\pgfsetlinewidth{0.803000pt}%
\definecolor{currentstroke}{rgb}{0.000000,0.000000,0.000000}%
\pgfsetstrokecolor{currentstroke}%
\pgfsetdash{}{0pt}%
\pgfpathmoveto{\pgfqpoint{2.780000in}{0.472778in}}%
\pgfpathlineto{\pgfqpoint{2.780000in}{1.680000in}}%
\pgfusepath{stroke}%
\end{pgfscope}%
\begin{pgfscope}%
\pgfsetrectcap%
\pgfsetmiterjoin%
\pgfsetlinewidth{0.803000pt}%
\definecolor{currentstroke}{rgb}{0.000000,0.000000,0.000000}%
\pgfsetstrokecolor{currentstroke}%
\pgfsetdash{}{0pt}%
\pgfpathmoveto{\pgfqpoint{0.616528in}{0.472778in}}%
\pgfpathlineto{\pgfqpoint{2.780000in}{0.472778in}}%
\pgfusepath{stroke}%
\end{pgfscope}%
\begin{pgfscope}%
\pgfsetrectcap%
\pgfsetmiterjoin%
\pgfsetlinewidth{0.803000pt}%
\definecolor{currentstroke}{rgb}{0.000000,0.000000,0.000000}%
\pgfsetstrokecolor{currentstroke}%
\pgfsetdash{}{0pt}%
\pgfpathmoveto{\pgfqpoint{0.616528in}{1.680000in}}%
\pgfpathlineto{\pgfqpoint{2.780000in}{1.680000in}}%
\pgfusepath{stroke}%
\end{pgfscope}%
\begin{pgfscope}%
\pgfsetbuttcap%
\pgfsetmiterjoin%
\definecolor{currentfill}{rgb}{1.000000,1.000000,1.000000}%
\pgfsetfillcolor{currentfill}%
\pgfsetfillopacity{0.800000}%
\pgfsetlinewidth{1.003750pt}%
\definecolor{currentstroke}{rgb}{0.800000,0.800000,0.800000}%
\pgfsetstrokecolor{currentstroke}%
\pgfsetstrokeopacity{0.800000}%
\pgfsetdash{}{0pt}%
\pgfpathmoveto{\pgfqpoint{0.694306in}{1.089431in}}%
\pgfpathlineto{\pgfqpoint{1.577299in}{1.089431in}}%
\pgfpathquadraticcurveto{\pgfqpoint{1.599521in}{1.089431in}}{\pgfqpoint{1.599521in}{1.111653in}}%
\pgfpathlineto{\pgfqpoint{1.599521in}{1.602222in}}%
\pgfpathquadraticcurveto{\pgfqpoint{1.599521in}{1.624444in}}{\pgfqpoint{1.577299in}{1.624444in}}%
\pgfpathlineto{\pgfqpoint{0.694306in}{1.624444in}}%
\pgfpathquadraticcurveto{\pgfqpoint{0.672083in}{1.624444in}}{\pgfqpoint{0.672083in}{1.602222in}}%
\pgfpathlineto{\pgfqpoint{0.672083in}{1.111653in}}%
\pgfpathquadraticcurveto{\pgfqpoint{0.672083in}{1.089431in}}{\pgfqpoint{0.694306in}{1.089431in}}%
\pgfpathclose%
\pgfusepath{stroke,fill}%
\end{pgfscope}%
\begin{pgfscope}%
\pgfsetrectcap%
\pgfsetroundjoin%
\pgfsetlinewidth{1.003750pt}%
\definecolor{currentstroke}{rgb}{0.121569,0.466667,0.705882}%
\pgfsetstrokecolor{currentstroke}%
\pgfsetdash{}{0pt}%
\pgfpathmoveto{\pgfqpoint{0.716528in}{1.534470in}}%
\pgfpathlineto{\pgfqpoint{0.938750in}{1.534470in}}%
\pgfusepath{stroke}%
\end{pgfscope}%
\begin{pgfscope}%
\definecolor{textcolor}{rgb}{0.000000,0.000000,0.000000}%
\pgfsetstrokecolor{textcolor}%
\pgfsetfillcolor{textcolor}%
\pgftext[x=1.027639in,y=1.495582in,left,base]{\color{textcolor}\sffamily\fontsize{8.000000}{9.600000}\selectfont \(\displaystyle 10x-10\)}%
\end{pgfscope}%
\begin{pgfscope}%
\pgfsetrectcap%
\pgfsetroundjoin%
\pgfsetlinewidth{1.003750pt}%
\definecolor{currentstroke}{rgb}{1.000000,0.498039,0.054902}%
\pgfsetstrokecolor{currentstroke}%
\pgfsetdash{}{0pt}%
\pgfpathmoveto{\pgfqpoint{0.716528in}{1.367479in}}%
\pgfpathlineto{\pgfqpoint{0.938750in}{1.367479in}}%
\pgfusepath{stroke}%
\end{pgfscope}%
\begin{pgfscope}%
\pgfsetbuttcap%
\pgfsetroundjoin%
\definecolor{currentfill}{rgb}{1.000000,0.498039,0.054902}%
\pgfsetfillcolor{currentfill}%
\pgfsetlinewidth{1.003750pt}%
\definecolor{currentstroke}{rgb}{1.000000,0.498039,0.054902}%
\pgfsetstrokecolor{currentstroke}%
\pgfsetdash{}{0pt}%
\pgfsys@defobject{currentmarker}{\pgfqpoint{-0.034722in}{-0.034722in}}{\pgfqpoint{0.034722in}{0.034722in}}{%
\pgfpathmoveto{\pgfqpoint{-0.034722in}{0.000000in}}%
\pgfpathlineto{\pgfqpoint{0.034722in}{0.000000in}}%
\pgfpathmoveto{\pgfqpoint{0.000000in}{-0.034722in}}%
\pgfpathlineto{\pgfqpoint{0.000000in}{0.034722in}}%
\pgfusepath{stroke,fill}%
}%
\begin{pgfscope}%
\pgfsys@transformshift{0.827639in}{1.367479in}%
\pgfsys@useobject{currentmarker}{}%
\end{pgfscope}%
\end{pgfscope}%
\begin{pgfscope}%
\definecolor{textcolor}{rgb}{0.000000,0.000000,0.000000}%
\pgfsetstrokecolor{textcolor}%
\pgfsetfillcolor{textcolor}%
\pgftext[x=1.027639in,y=1.328590in,left,base]{\color{textcolor}\sffamily\fontsize{8.000000}{9.600000}\selectfont \(\displaystyle x^2-1\)}%
\end{pgfscope}%
\begin{pgfscope}%
\pgfsetbuttcap%
\pgfsetroundjoin%
\pgfsetlinewidth{0.501875pt}%
\definecolor{currentstroke}{rgb}{0.172549,0.627451,0.172549}%
\pgfsetstrokecolor{currentstroke}%
\pgfsetdash{}{0pt}%
\pgfpathmoveto{\pgfqpoint{0.827639in}{1.148838in}}%
\pgfpathlineto{\pgfqpoint{0.827639in}{1.259949in}}%
\pgfusepath{stroke}%
\end{pgfscope}%
\begin{pgfscope}%
\pgfsetbuttcap%
\pgfsetroundjoin%
\definecolor{currentfill}{rgb}{0.172549,0.627451,0.172549}%
\pgfsetfillcolor{currentfill}%
\pgfsetlinewidth{0.501875pt}%
\definecolor{currentstroke}{rgb}{0.172549,0.627451,0.172549}%
\pgfsetstrokecolor{currentstroke}%
\pgfsetdash{}{0pt}%
\pgfsys@defobject{currentmarker}{\pgfqpoint{-0.027778in}{-0.000000in}}{\pgfqpoint{0.027778in}{0.000000in}}{%
\pgfpathmoveto{\pgfqpoint{0.027778in}{-0.000000in}}%
\pgfpathlineto{\pgfqpoint{-0.027778in}{0.000000in}}%
\pgfusepath{stroke,fill}%
}%
\begin{pgfscope}%
\pgfsys@transformshift{0.827639in}{1.148838in}%
\pgfsys@useobject{currentmarker}{}%
\end{pgfscope}%
\end{pgfscope}%
\begin{pgfscope}%
\pgfsetbuttcap%
\pgfsetroundjoin%
\definecolor{currentfill}{rgb}{0.172549,0.627451,0.172549}%
\pgfsetfillcolor{currentfill}%
\pgfsetlinewidth{0.501875pt}%
\definecolor{currentstroke}{rgb}{0.172549,0.627451,0.172549}%
\pgfsetstrokecolor{currentstroke}%
\pgfsetdash{}{0pt}%
\pgfsys@defobject{currentmarker}{\pgfqpoint{-0.027778in}{-0.000000in}}{\pgfqpoint{0.027778in}{0.000000in}}{%
\pgfpathmoveto{\pgfqpoint{0.027778in}{-0.000000in}}%
\pgfpathlineto{\pgfqpoint{-0.027778in}{0.000000in}}%
\pgfusepath{stroke,fill}%
}%
\begin{pgfscope}%
\pgfsys@transformshift{0.827639in}{1.259949in}%
\pgfsys@useobject{currentmarker}{}%
\end{pgfscope}%
\end{pgfscope}%
\begin{pgfscope}%
\pgfsetrectcap%
\pgfsetroundjoin%
\pgfsetlinewidth{1.003750pt}%
\definecolor{currentstroke}{rgb}{0.172549,0.627451,0.172549}%
\pgfsetstrokecolor{currentstroke}%
\pgfsetdash{}{0pt}%
\pgfpathmoveto{\pgfqpoint{0.716528in}{1.204393in}}%
\pgfpathlineto{\pgfqpoint{0.938750in}{1.204393in}}%
\pgfusepath{stroke}%
\end{pgfscope}%
\begin{pgfscope}%
\pgfsetbuttcap%
\pgfsetroundjoin%
\definecolor{currentfill}{rgb}{0.172549,0.627451,0.172549}%
\pgfsetfillcolor{currentfill}%
\pgfsetlinewidth{0.501875pt}%
\definecolor{currentstroke}{rgb}{0.172549,0.627451,0.172549}%
\pgfsetstrokecolor{currentstroke}%
\pgfsetdash{}{0pt}%
\pgfsys@defobject{currentmarker}{\pgfqpoint{-0.017361in}{-0.017361in}}{\pgfqpoint{0.017361in}{0.017361in}}{%
\pgfpathmoveto{\pgfqpoint{0.000000in}{-0.017361in}}%
\pgfpathcurveto{\pgfqpoint{0.004604in}{-0.017361in}}{\pgfqpoint{0.009020in}{-0.015532in}}{\pgfqpoint{0.012276in}{-0.012276in}}%
\pgfpathcurveto{\pgfqpoint{0.015532in}{-0.009020in}}{\pgfqpoint{0.017361in}{-0.004604in}}{\pgfqpoint{0.017361in}{0.000000in}}%
\pgfpathcurveto{\pgfqpoint{0.017361in}{0.004604in}}{\pgfqpoint{0.015532in}{0.009020in}}{\pgfqpoint{0.012276in}{0.012276in}}%
\pgfpathcurveto{\pgfqpoint{0.009020in}{0.015532in}}{\pgfqpoint{0.004604in}{0.017361in}}{\pgfqpoint{0.000000in}{0.017361in}}%
\pgfpathcurveto{\pgfqpoint{-0.004604in}{0.017361in}}{\pgfqpoint{-0.009020in}{0.015532in}}{\pgfqpoint{-0.012276in}{0.012276in}}%
\pgfpathcurveto{\pgfqpoint{-0.015532in}{0.009020in}}{\pgfqpoint{-0.017361in}{0.004604in}}{\pgfqpoint{-0.017361in}{0.000000in}}%
\pgfpathcurveto{\pgfqpoint{-0.017361in}{-0.004604in}}{\pgfqpoint{-0.015532in}{-0.009020in}}{\pgfqpoint{-0.012276in}{-0.012276in}}%
\pgfpathcurveto{\pgfqpoint{-0.009020in}{-0.015532in}}{\pgfqpoint{-0.004604in}{-0.017361in}}{\pgfqpoint{0.000000in}{-0.017361in}}%
\pgfpathclose%
\pgfusepath{stroke,fill}%
}%
\begin{pgfscope}%
\pgfsys@transformshift{0.827639in}{1.204393in}%
\pgfsys@useobject{currentmarker}{}%
\end{pgfscope}%
\end{pgfscope}%
\begin{pgfscope}%
\definecolor{textcolor}{rgb}{0.000000,0.000000,0.000000}%
\pgfsetstrokecolor{textcolor}%
\pgfsetfillcolor{textcolor}%
\pgftext[x=1.027639in,y=1.165505in,left,base]{\color{textcolor}\sffamily\fontsize{8.000000}{9.600000}\selectfont \(\displaystyle 25\log_{2}(x)\)}%
\end{pgfscope}%
\end{pgfpicture}%
\makeatother%
\endgroup%

    \caption{\texttt{matplotlib}.\label{fig:matplotlib}}
  \end{subfigure}
  \caption{Exemplos de gráficos gerados externamente}\label{fig:graficos}
\end{figure}

Ao trabalhar com \emph{floats}, tenha duas coisas em mente:

\begin{itemize}

  \item \emph{Floats} em geral incluem uma legenda e um \emph{label}.
  Prefira sempre colocar o comando \textsf{\textbackslash{}label} de uma
  figura ou tabela dentro do comando \textsf{\textbackslash{}caption};
  não fazê-lo muitas vezes funciona, mas às vezes causa problemas.

  \item \LaTeX{} nem sempre posiciona os \emph{floats} no melhor lugar
  possível; leia a discussão a respeito na Seção~\ref{sec:limitations}.

\end{itemize}

\section{Tabelas}\index{Floats}

Talvez você precise organizar a apresentação da informação na forma de
tabelas\index{Floats}\footnote{Para defini-las com \LaTeX{}, pode valer a pena usar o
sítio \url{www.tablesgenerator.com}.}; um exemplo simples é a Tabela~\ref{tab:amino_acidos}.
Para um resultado visual excelente, não deixe de ler a documentação da
\emph{package} \pkg{booktabs}.

Normalmente, o fim de cada linha de uma tabela é indicado por
\textsf{\textbackslash\textbackslash}. No entanto, se sua tabela causar
erros misteriosos, experimente usar \textsf{\textbackslash{}tabularnewline}
ao invés de \textsf{\textbackslash\textbackslash}.

%%%%%%%% Tabelas lado-a-lado %%%%%%%%

\begin{table}
\centering

  \hspace*{\fill}
  \begin{subtable}[b]{0.45\textwidth}
    % \rowcolors é definida pela package xcolor;
    % veja também os recursos da package colortbl
    \rowcolors{2}{lightgray!70}{white}
    \centering
    \begin{tabular}{ccl}
      \toprule
      Código      & Abreviatura  & \makecell{Nome\\completo} \\
      \midrule
      \texttt{A}  & Ala          & Alanina \\
      \texttt{C}  & Cys          & Cisteína \\
      ...         & ...          & ... \\
      \texttt{W}  & Trp          & Triptofano \\
      \texttt{Y}  & Tyr          & Tirosina \\
      \bottomrule
    \end{tabular}
    \caption{Com linhas de cores alternadas.}
  \end{subtable}
  % Como mencionado mais acima, não deixe linhas em branco aqui
  \hspace*{\fill}\hspace*{\fill}\hspace*{\fill}
  \begin{subtable}[b]{0.37\textwidth}
    \centering
    \begin{tabular}{ccl}
      \rothead{Código} & \rothead{Abreviatura} & \rothead{Nome\\completo} \\
      \midrule
      \texttt{A}       & Ala                   & Alanina \\
      \texttt{C}       & Cys                   & Cisteína \\
      ...              & ...                   & ... \\
      \texttt{W}       & Trp                   & Triptofano \\
      \texttt{Y}       & Tyr                   & Tirosina \\
      \bottomrule
    \end{tabular}
    \caption{Com cabeçalhos girados.}
  \end{subtable}
  \hspace*{\fill}

  \caption{Exemplos de tabelas (códigos, abreviaturas e nomes dos aminoácidos).\label{tab:amino_acidos}}
\end{table}

Se a tabela tem muitas linhas e, portanto, não cabe em uma única página, é
possível fazê-la continuar ao longo de várias páginas com a \textit{package}
\textsf{longtable}, como é o caso da Tabela~\ref{tab:numeros}. Nesse caso,
a tabela não é um \textit{float} e, portanto, ela aparece de acordo com a
sequência normal do texto. Se, além de muito longa, a tabela for também
muito larga, você pode usar o comando \textsf{landscape} (da
\textit{package} \textsf{pdflscape}) em conjunto com \textsf{longtable}
para imprimi-la em modo paisagem ao longo de várias páginas. A
Tabela~\ref{tab:numeros} tem essa configuração comentada; experimente
des-comentar as linhas correspondentes\footnote{Observe que, nesse caso,
vai sempre haver uma quebra de página no texto para fazer a tabela
começar em uma página em modo paisagem.}. Ela também demonstra o uso
da \emph{package} \pkg{siunitx} para alinhar as colunas numéricas pelo
separador decimal.

%%%%%%%% Tabela longa em várias páginas %%%%%%%%

% As colunas do tipo "S" são definidas pela package siunitx e servem
% para alinhar os números pelo separador decimal. \sisetup{} configura
% alguns parâmetros de siunitx; colocamos dentro de um grupo para que
% esses parâmetros só afetem esta tabela.
\bgroup
\sisetup{
  round-precision = 4,
  table-number-alignment = left,
  table-format = 1.4, % uma casa antes da vírgula e 4 depois
}

%%%% É possível fazer esta mesma tabela em modo paisagem des-comentando
%%%% esta linha e a correspondente no final da tabela
%\begin{landscape}
\begin{longtable}[c]{
  r
  S
  S
  S
  S[table-format = 2.4] % nestas colunas, duas casas antes da vírgula
  S[table-format = 2.4]
  S[table-format = 2.4]
  S[table-format = 2.4]
  }

%%%%%%%%%%%%
% O cabeçalho da tabela na primeira página em que ela aparece.
\toprule
\multicolumn{2}{c}{Ângulo} &
\multicolumn{6}{c}{Função} \\

\cmidrule(lr){1-2} \cmidrule(lr){3-8}
{graus} & {rads} &
{sen} & {cos} & {tan} & {cotan} & {sec} & {cosec} \\
\cmidrule(lr){1-2} \cmidrule(lr){3-8}

\endfirsthead % Final do cabeçalho que aparece na primeira página

%%%%%%%%%%%%
% O cabeçalho da tabela em todas as páginas em que ela aparece
% exceto a primeira; aqui, igual ao anterior
\toprule
\multicolumn{2}{c}{Ângulo} &
\multicolumn{6}{c}{Função} \\

\cmidrule(lr){1-2} \cmidrule(lr){3-8}
{graus} & {rads} &
{sen} & {cos} & {tan} & {cotan} & {sec} & {cosec} \\
\cmidrule(lr){1-2} \cmidrule(lr){3-8}

\endhead % Final do cabeçalho das páginas seguintes à primeira

%%%%%%%%%%%%
% O rodapé da tabela em todas as páginas em que ela aparece
% exceto a última

\multicolumn{8}{r}{\textit{continua}\enspace$\longrightarrow$}\\

% Como usamos \captionlistentry mais abaixo, usamos "[]" aqui.
\caption[]{Exemplo de tabela com valores numéricos.}

\endfoot % Final do rodapé que aparece em todas as páginas exceto a última

%%%%%%%%%%%%
% O rodapé da tabela na última página em que ela aparece

\bottomrule

% Como usamos \captionlistentry mais abaixo, usamos "[]" aqui.
\caption[]{Exemplo de tabela com valores numéricos.}

\endlastfoot % Final do rodapé da última página

%%%%%%%%%%%%
% O conteúdo da tabela de fato.

% Como a tabela pode se estender por várias páginas, precisamos tomar
% cuidado especial com \caption e \label: se esses comandos forem
% incluídos no cabeçalho ou rodapé que aparece em várias páginas, eles
% serão executados mais de uma vez e tanto a lista de tabelas quanto as
% referências à tabela ficarão incorretas. Há diversas soluções, mas a
% mais simples é (1) não colocar \label no cabeçalho ou rodapé, (2) usar
% "\caption[]" no cabeçalho ou rodapé, que não inclui a tabela na lista
% de tabelas, e (3) colocar \captionlistentry e \label aqui, na primeira
% célula da tabela. \captionlistentry não imprime nada, apenas acrescenta
% um item na lista de tabelas com o número de página correspondente.
\captionlistentry{Exemplo de tabela com valores numéricos.}\label{tab:numeros}

0  & 0      & 0      & 1      & 0       & {-}     & 1       & {-}     \\
3  & 0,0524 & 0,0523 & 0,9986 & 0,0524  & 19,0811 & 1,0014  & 19,1073 \\
6  & 0,1047 & 0,1045 & 0,9945 & 0,1051  & 9,5144  & 1,0055  & 9,5668  \\
9  & 0,1571 & 0,1564 & 0,9877 & 0,1584  & 6,3138  & 1,0125  & 6,3925  \\
12 & 0,2094 & 0,2079 & 0,9781 & 0,2126  & 4,7046  & 1,0223  & 4,8097  \\
15 & 0,2618 & 0,2588 & 0,9659 & 0,2679  & 3,7321  & 1,0353  & 3,8637  \\
18 & 0,3142 & 0,3090 & 0,9511 & 0,3249  & 3,0777  & 1,0515  & 3,2361  \\
21 & 0,3665 & 0,3584 & 0,9336 & 0,3839  & 2,6051  & 1,0711  & 2,7904  \\
24 & 0,4189 & 0,4067 & 0,9135 & 0,4452  & 2,2460  & 1,0946  & 2,4586  \\
27 & 0,4712 & 0,4540 & 0,8910 & 0,5095  & 1,9626  & 1,1223  & 2,2027  \\
30 & 0,5236 & 0,5000 & 0,8660 & 0,5774  & 1,7321  & 1,1547  & 2,0000  \\
33 & 0,5760 & 0,5446 & 0,8387 & 0,6494  & 1,5399  & 1,1924  & 1,8361  \\
36 & 0,6283 & 0,5878 & 0,8090 & 0,7265  & 1,3764  & 1,2361  & 1,7013  \\
39 & 0,6807 & 0,6293 & 0,7771 & 0,8098  & 1,2349  & 1,2868  & 1,5890  \\
% Como nesta página há uma nota de rodapé, a linha separadora da
% nota e o final da tabela ficam muito próximos; vamos forçar uma
% quebra de página uma linha antes para resolver isso.
\pagebreak
42 & 0,7330 & 0,6691 & 0,7431 & 0,9004  & 1,1106  & 1,3456  & 1,4945  \\
45 & 0,7854 & 0,7071 & 0,7071 & 1       & 1       & 1,4142  & 1,4142  \\
48 & 0,8378 & 0,7431 & 0,6691 & 1,1106  & 0,9004  & 1,4945  & 1,3456  \\
51 & 0,8901 & 0,7771 & 0,6293 & 1,2349  & 0,8098  & 1,5890  & 1,2868  \\
54 & 0,9425 & 0,8090 & 0,5878 & 1,3764  & 0,7265  & 1,7013  & 1,2361  \\
57 & 0,9948 & 0,8387 & 0,5446 & 1,5399  & 0,6494  & 1,8361  & 1,1924  \\
60 & 1,0472 & 0,8660 & 0,5000 & 1,7321  & 0,5774  & 2,0000  & 1,1547  \\
63 & 1,0996 & 0,8910 & 0,4540 & 1,9626  & 0,5095  & 2,2027  & 1,1223  \\
66 & 1,1519 & 0,9135 & 0,4067 & 2,2460  & 0,4452  & 2,4586  & 1,0946  \\
69 & 1,2043 & 0,9336 & 0,3584 & 2,6051  & 0,3839  & 2,7904  & 1,0711  \\
72 & 1,2566 & 0,9511 & 0,3090 & 3,0777  & 0,3249  & 3,2361  & 1,0515  \\
75 & 1,3090 & 0,9659 & 0,2588 & 3,7321  & 0,2679  & 3,8637  & 1,0353  \\
78 & 1,3614 & 0,9781 & 0,2079 & 4,7046  & 0,2126  & 4,8097  & 1,0223  \\
81 & 1,4137 & 0,9877 & 0,1564 & 6,3138  & 0,1584  & 6,3925  & 1,0125  \\
84 & 1,4661 & 0,9945 & 0,1045 & 9,5144  & 0,1051  & 9,5668  & 1,0055  \\
87 & 1,5184 & 0,9986 & 0,0523 & 19,0811 & 0,0524  & 19,1073 & 1,0014  \\
90 & 1,5708 & 1      & 0      & {-}     & 0       & {-}     & 1
\end{longtable}
%\end{landscape}
\egroup % definições especiais para siunitx

Tabelas mais complexas são um tanto trabalhosas em \LaTeX{}; a
Tabela~\ref{tab:ficha} mostra como construir uma tabela em forma de ficha.
Além de complexa, ela é larga e, portanto, deve ser impressa em modo
paisagem. No entanto, usamos um outro mecanismo para girar a tabela: o
comando \textsf{sidewaystable} (da \textit{package} \textsf{rotating}).
Com esse mecanismo, ela continua sendo um \textit{float} (e, portanto,
não força quebras de página no meio do texto), mas sempre é impressa em
uma página separada.

Resumindo:

\begin{itemize}
  \item Se uma tabela cabe em uma página, defina-a como um \textit{float}
        (\verb|\begin{table}|);
  \item Se cabe em uma página mas é muito larga e precisa ser impressa em
        modo paisagem, use \textsf{sidewaystable} (que também é um \textit{float});
  \item Se não cabe em uma página por ser muito longa, use \textsf{longtable};
  \item Se não cabe em uma página por ser muito longa e precisa ser impressa
        em modo paisagem por ser muito larga, use \textsf{longtable} em
        conjunto com \textsf{landscape}. Nesse caso, vai haver uma quebra
        de página no texto para que a tabela inicie em uma nova página em
        modo paisagem.
\end{itemize}

%%%%%%%% Tabela em forma de ficha %%%%%%%%

% Aumenta o espaçamento entre as linhas da tabela (default: 0pt)
\setlength\extrarowheight{4pt}

% sidewaystable e comandos relacionados são definidos na package rotating
\begin{sidewaystable}
\centering

\begin{tabular}{|M{0.265}|M{0.073}|M{0.084}|M{0.073}|M{0.073}|M{0.08}|M{0.082}|M{0.067}|}
  \hline
    \textbf{Experimento número:} & \multicolumn{2}{c|}{1} & \multicolumn{4}{c|}{\textbf{Data:}} & jan 2017
  \tabularnewline \hline
    \textbf{Título:} & \multicolumn{7}{c|}{Medições iniciais}
  \tabularnewline \hline
    \textbf{Tipo de experimento:} & \multicolumn{7}{c|}{Levantamento quantitativo}
  \tabularnewline \hline \hline
    \textbf{Locais}          & São Paulo & Rio de Janeiro & Porto Alegre & Recife & Manaus & Brasília & Rio Branco
  \tabularnewline \thickhline
    \textbf{Valores obtidos} & 0.2       & 0.3            & 0.2          & 0.7    & 0.5    & 0.1      & 0.4
  \tabularnewline \hline
\end{tabular}

\caption{Exemplo de tabela similar a uma ficha.\label{tab:ficha}}
\end{sidewaystable}

% Redefinindo para o valor default
\setlength\extrarowheight{0pt}



%%%%%%%%%%%%%%%%%%%%%%%%%%%% APÊNDICES E ANEXOS %%%%%%%%%%%%%%%%%%%%%%%%%%%%%%%%

% Um apêndice é algum conteúdo adicional de sua autoria que faz parte e
% colabora com a ideia geral do texto mas que, por alguma razão, não precisa
% fazer parte da sequência do discurso; por exemplo, a demonstração de um
% teorema intermediário, as perguntas usadas em uma pesquisa qualitativa etc.
%
% Um anexo é um documento que não faz parte da tese (em geral, nem é de sua
% autoria) mas é relevante para o conteúdo; por exemplo, a especificação do
% padrão técnico ou a legislação que o trabalho discute, um artigo de jornal
% apresentando a percepção do público sobre o tema da tese etc.
%
% Os comandos appendix e annex reiniciam a numeração de capítulos e passam
% a numerá-los com letras. "annex" não faz parte de nenhuma classe padrão,
% foi criado para este modelo. Se o trabalho não tiver apêndices ou anexos,
% remova estas linhas.
%
% Diferentemente de \mainmatter, \backmatter etc., \appendix e \annex não
% forçam o início de uma nova página. Em geral isso não é importante, pois
% o comando seguinte costuma ser "\chapter", mas pode causar problemas com
% a formatação dos cabeçalhos. Assim, vamos forçar uma nova página antes
% de cada um deles.

%%%% Apêndices %%%%

\makeatletter
\if@openright\cleardoublepage\else\clearpage\fi
\makeatother

\pagestyle{appendix}

\appendix

% \addappheadtotoc acrescenta a palavra "Apêndice" ao sumário; se
% só há apêndices, sem anexos, provavelmente não é necessário.
\addappheadtotoc

%!TeX root=../tese.tex
%("dica" para o editor de texto: este arquivo é parte de um documento maior)
% para saber mais: https://tex.stackexchange.com/q/78101

\chapter{Código-fonte e pseudocódigo}
\label{ap:pseudocode}

Com a \textit{package} \textsf{listings}, programas podem ser inseridos
diretamente no arquivo, como feito no caso do Programa~\ref{prog:java},
ou importados de um arquivo externo com o comando
\textsf{\textbackslash{}lstinputlisting}, como no caso
do Programa~\ref{prog:mdcinput}.

% O exemplo foi copiado da documentação de algorithmicx
\begin{program}
  \lstinputlisting[
    language=pseudocode,
    style=pseudocode,
    style=wider,
    functions={},
    specialidentifiers={},
  ]
  {conteudo/euclid.psc}

  \caption{Máximo divisor comum (arquivo importado).\label{prog:mdcinput}}
\end{program}

Trechos de código curtos (menores que uma página) podem ou não ser
incluídos como \textit{floats}; trechos longos necessariamente incluem
quebras de página e, portanto, não podem ser \textit{floats}. Com
\textit{floats}, a legenda e as linhas separadoras são colocadas pelo
comando \textsf{\textbackslash{}begin\{program\}}; sem eles, utilize o
ambiente \textsf{programruledcaption} (atenção para a colocação do
comando \textsf{\textbackslash{}label\{\}}, dentro da legenda), como
no Programa~\ref{prog:mdc}\footnote{\textsf{listings} oferece alguns
recursos próprios para a definição de \textit{floats} e legendas, mas
neste modelo não os utilizamos.}:

\begin{programruledcaption}{Máximo divisor comum (em português).\label{prog:mdc}}
  \begin{lstlisting}[
    language={[brazilian]pseudocode},
    style=pseudocode,
    style=wider,
    functions={},
    specialidentifiers={},
  ]
      funcao euclides(a, b) // O máximo divisor comum de \textbf{a} e \textbf{b}
          r := a $\bmod$ b
	  enquanto r != 0 // Atingimos a resposta se \textbf{r} é zero
              a := b
              b := r
              r := a $\bmod$ b
          fim
	  devolva b // O máximo divisor comum é \textbf{b}
      fim
  \end{lstlisting}
\end{programruledcaption}

Além do suporte às várias linguagens incluídas em \textsf{listings},
este modelo traz uma extensão para permitir o uso de pseudocódigo,
útil para a descrição de algoritmos em alto nível. Ela oferece
diversos recursos:

\begin{itemize}

    \item Comentários seguem o padrão de C++ (\lstinline{//} e
          \lstinline{/* ... */}), mas o delimitador é impresso
          como ``$\triangleright$''.

    \item ``:='', ``<>'', ``<='', ``>='' e ``!='' são substituídos
          pelo símbolo matemático adequado.

    \item É possível acrescentar palavras-chave além de ``if'', ``and''
          etc. com a opção ``\textsf{morekeywords=\{pchave1,\linebreak[0]{}pchave2\}}''
          (para um trecho de código específico) ou com o comando
          \textsf{\textbackslash{}lstset\{morekeywords=\linebreak[0]{}\{pchave1,pchave2\}\}}
          (como comando de configuração geral).

    \item É possível usar pequenos trechos de código, como nomes de variáveis,
          dentro de um parágrafo normal com \textsf{\textbackslash{}lstinline\{blah\}}.

    \item ``\$\dots\$'' ativa o modo matemático em qualquer lugar.

    \item Outros comandos \LaTeX{} funcionam apenas em comentários; fora, a
          linguagem simula alguns pré-definidos (\textsf{\textbackslash{}textit\{\}},
          \textsf{\textbackslash{}texttt\{\}} etc.).

    \item O comando \textsf{\textbackslash{}label} também funciona em
          comentários; a referência correspondente (\textsf{\textbackslash{}ref})
          indica o número da linha de código. Se quiser usá-lo numa linha sem
          comentários, use \lstinline{///}~\textsf{\textbackslash{}label\{blah\}};
          ``\lstinline{///}'' funciona como \lstinline{//}, permitindo
          a inserção de comandos \LaTeX{}, mas não imprime o delimitador
          (\ensuremath{\triangleright}).

    \item Para suspender a formatação automática, use \textsf{\textbackslash{}noparse\{blah\}}.

    \item Para forçar a formatação de um texto como função, identificador,
          palavra-chave ou comentário, use \textsf{\textbackslash{}func\{blah\}},
          \textsf{\textbackslash{}id\{blah\}}, \textsf{\textbackslash{}kw\{blah\}} ou
          \textsf{\textbackslash{}comment\{blah\}}.

    \item Palavras-chave dentro de comentários não são formatadas
          automaticamente; se necessário, use \textsf{\textbackslash{}func\{\}},
          \textsf{\textbackslash{}id\{\}} etc. ou comandos \LaTeX{} padrão.

    \item As palavras ``Program'', ``Procedure'' e ``Function'' têm formatação
          especial e fazem a palavra seguinte ser formatada como função.
          Funções em outros lugares \emph{não} são detectadas automaticamente;
          use \textsf{\textbackslash{}func\{\}}, a opção ``\textsf{functions=\{func1,func2\}}''
          ou o comando ``\textsf{\textbackslash{}lstset\{functions=\{func1,func2\}\}}''
          para que elas sejam detectadas.

    \item Além de funções, palavras-chave, strings, comentários e
          identificadores, há ``\textsf{specialidentifiers}''. Você pode
          usá-los com \textsf{\textbackslash{}specialid\{blah\}}, com a opção
          ``\textsf{specialidentifiers=\{id1,id2\}}'' ou com o comando
          ``\textsf{\textbackslash{}lstset\{specialidentifiers=\{id1,id2\}\}}''.

\end{itemize}



\par

%%%% Anexos %%%%

\makeatletter
\if@openright\cleardoublepage\else\clearpage\fi
\makeatother

\pagestyle{appendix} % repete o anterior, caso você não use apêndices

\annex

% \addappheadtotoc acrescenta a palavra "Anexo" ao sumário; se
% só há anexos, sem apêndices, provavelmente não é necessário.
\addappheadtotoc

%!TeX root=../tese.tex
%("dica" para o editor de texto: este arquivo é parte de um documento maior)
% para saber mais: https://tex.stackexchange.com/q/78101/183146

\chapter[Perguntas Frequentes sobre o Modelo]{Perguntas Frequentes sobre o Modelo\footnote{Esta
seção não é de fato um anexo, mas sim um apêndice; ela foi definida desta
forma apenas para servir como exemplo de anexo.}}

\begin{itemize}

\item \textbf{Não consigo decorar tantos comandos!}\\
Use a colinha que é distribuída juntamente com este modelo (\url{gitlab.com/ccsl-usp/modelo-latex/raw/master/pre-compilados/colinha.pdf?inline=false}).

\item \textbf{Por que tantos arquivos?}\\
O preâmbulo \LaTeX{} deste modelo é muito longo; as partes que normalmente não precisam ser modificadas foram colocadas no diretório \texttt{extras}, juntamente com alguns arquivos acessórios. Já os arquivos de conteúdo (capítulos, anexos etc.) foram divididos de maneira que seja fácil para você atualizar o modelo (copiando os novos arquivos ou com um sistema de controle de versões) sem que alterações no conteúdo de exemplo (este texto que você está lendo) causem conflitos com o seu próprio texto.\looseness=-1

\item \textbf{As figuras e tabelas são colocadas em lugares ruins.}\\
Veja a discussão a respeito na Seção~\ref{sec:limitations}.

\item \textbf{Estou tendo problemas com caracteres acentuados.}\\
Versões modernas de \LaTeX{} usam UTF-8, mas arquivos antigos podem usar outras codificações (como ISO-8859-1, também conhecido como latin1 ou Windows-1252). Nesses casos, use \textsf{\textbackslash{}usepackage[latin1]\{inputenc\}} no preâmbulo do documento. Você também pode representar os caracteres acentuados usando comandos \LaTeX{}: \textsf{\textbackslash\textquotesingle{}a} para á, \textsf{\textbackslash{}c\{c\}} para cedilha etc., independentemente da codificação usada no texto\footnote{Você pode consultar os comandos desse tipo mais comuns em \url{en.wikibooks.org/wiki/LaTeX/Special_Characters}. Observe que a dica sobre o pingo do i \emph{não} é mais válida atualmente; basta usar \textsf{\textbackslash\textquotesingle{}i}.}.

\item \textbf{Existe algo específico para citações de páginas web?}\\
Biblatex define o tipo ``online'', que deve ser usado para materiais com título, autor etc., como uma postagem ou comentário em um blog, um gráfico ou mesmo uma mensagem de email para uma lista de discussão. Bibtex\index{bibtex}, por padrão, não tem um tipo específico para isso; com ele, normalmente usa-se o campo ``howpublished'' para especificar que se trata de um recurso \textit{online}. Se o que você está citando não é algo determinado com título, autor etc. mas sim um sítio (como uma empresa ou um produto), pode ser mais adequado colocar a referência apenas como nota de rodapé e não na lista de referências; nesses casos, algumas pessoas acrescentam uma segunda lista de referências especificamente para recursos \textit{online} (biblatex\index{biblatex} permite criar múltiplas bibliografias). Já artigos disponíveis \textit{online} mas que fazem parte de uma publicação de formato tradicional (mesmo que apenas \textit{online}), como os anais de um congresso, devem ser citados por seu tipo verdadeiro e apenas incluir o campo ``url'' (não é nem necessário usar o comando \textsf{\textbackslash{}url\{\}}), aceito por todos os tipos de documento do bibtex/biblatex.

\item \textbf{Aparece uma folha em branco entre os capítulos.}\\
Essa característica foi colocada propositalmente, dado que todo capítulo deve (ou deveria) começar em uma página de numeração ímpar (lado direito do documento). Se quiser mudar esse comportamento, acrescente ``openany'' como opção da classe, i.e., \textsf{\textbackslash{}documentclass[openany,\dots]\{book\}}.

\item \textbf{É possível resumir o nome das seções/capítulos que aparece no topo das páginas e no sumário?}\\
Sim, usando a sintaxe \textsf{\textbackslash{}section[mini-titulo]\{titulo enorme\}}. Isso é especialmente útil nas legendas (\textit{captions}\index{Legendas}) das figuras e tabelas, que muitas vezes são demasiadamente longas para a lista de figuras/tabelas.

\item \textbf{Existe algum programa para gerenciar referências em formato bibtex?}\\
Sim, há vários. Uma opção bem comum é o JabRef; outra é usar Zotero\index{Zotero} ou Mendeley\index{Mendeley} e exportar os dados deles no formato .bib.

\item \textbf{Posso usar pacotes \LaTeX{} adicionais aos sugeridos?}\\
Com certeza! Você pode modificar os arquivos o quanto desejar, o modelo serve só como uma ajuda inicial para o seu trabalho.

\item \textbf{Como faço para usar o Makefile (comando make) no Windows?}\\
Lembre-se que a ferramenta recomendada para compilação do documento é o \textsf{latexmk}, então você não precisa do \textsf{make}. Mas, se quiser usá-lo, você pode instalar o MSYS2 (\url{www.msys2.org}) ou o Windows Subsystem for Linux (procure as versões de Linux disponíveis na Microsoft Store). Se você pretende usar algum dos editores sugeridos, é possível deixar a compilação a cargo deles, também dispensando o \textsf{make}.\looseness=-1

\item \textbf{Como eu faço para...}\\
Leia os comentários dos arquivos ``tese.tex'' e outros que compõem este modelo, além do tutorial (Capítulo \ref{chap:tutorial}) e dos exemplos do Capítulo \ref{chap:exemplos}; é provável que haja uma dica neles ou, pelo menos, a indicação da \textit{package} relacionada ao que você precisa.

\end{itemize}

\par


%%%%%%%%%%%%%%% SEÇÕES FINAIS (BIBLIOGRAFIA E ÍNDICE REMISSIVO) %%%%%%%%%%%%%%%%

% O comando backmatter desabilita a numeração de capítulos.
\backmatter

\pagestyle{backmatter}

% Espaço adicional no sumário antes das referências / índice remissivo
\addtocontents{toc}{\vspace{2\baselineskip plus .5\baselineskip minus .5\baselineskip}}

% A bibliografia é obrigatória

\printbibliography[
  title=\refname\label{bibliografia}, % "Referências", recomendado pela ABNT
  %title=\bibname\label{bibliografia}, % "Bibliografia"
  heading=bibintoc, % Inclui a bibliografia no sumário
]

\printindex % imprime o índice remissivo no documento (opcional)

\end{document}
