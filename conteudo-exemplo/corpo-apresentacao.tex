%!TeX root=../apresentacao.tex
%("dica" para o editor de texto: este arquivo é parte de um documento maior)
% para saber mais: https://tex.stackexchange.com/q/78101/183146

% É complicado colocar uma imagem de fundo, os logos das agências e
% o conteúdo "normal" do slide de título sem que as coisas fiquem
% bagunçadas, então definimos um comando para gerar o slide de título
\customtitlepage

% Slide com o qrcode
\showqrcode

\begin{frame}{Overview}
  \overview
\end{frame}

\section{Introduction}

\begin{frame}{Context}
  \begin{itemize}
    \item The copyright compromise seeked to balance public and private interests
    \item Nowadays, changes to the law and technological advances all but destroyed this balance
    \item[]
    \item As a reaction, the free software movement was created
    \begin{itemize}
      \item Return to sharing (of source code) and to collaboration (exchange of ideas and team work)
      \item Formalization with the GNU project
      \item Only really possible when there are favourable conditions for source code exchange
      \begin{itemize}
        \item as highlighted by the growth that accompanied the Internet boom
      \end{itemize}
    \end{itemize}
  \end{itemize}

\end{frame}

\begin{frame}[plain]
  \includegraphics[width=\textwidth]{interscity-logo}
\end{frame}

\begin{frame}[standout]
  This is a problem!
\end{frame}

\begin{frame}{Goals}
  \begin{block}{Functional requirements}
    \begin{itemize}
      \item Integration and Management of \alert{IoT} Devices
      \item Data Acquisition, Storing, and Processing
      \item Context-awareness
      \item City Resource Discovery
      \item Geolocation-based Services
      \item External data access
    \end{itemize}
  \end{block}
\end{frame}

\section{Concepts}

\begin{frame}{Concepts}
  \begin{columns}[t]
    \col
      \begin{coloredblock}{red!90!black}{Functional requirements}
        \begin{itemize}
          \item Integration and Management of IoT Devices
          \item Data Acquisition, Storing, and Processing
          \item Context-awareness
          \item City Resource Discovery
          \item Geolocation-based Services
          \item External data access
        \end{itemize}
      \end{coloredblock}

    \col
      \begin{coloredblock}{red!90!black}{Non-functional requirements}
        \begin{itemize}
          \item Interoperability
          \item Scalability
          \item Security
          \item Privacy
          \item Evolvability
          \item Adaptability
        \end{itemize}
      \end{coloredblock}
  \end{columns}
\end{frame}

\begin{frame}{Theorems and proofs}
  \pause
  \begin{theorem}[An example theorem]
    Theorem\dots
  \end{theorem}

  \pause
  \begin{example}[An example of an example]
    Example\dots
  \end{example}

  \pause
  \begin{proof}[An example proof]
    Proof\dots
  \end{proof}

  \pause
  \begin{definition}[An example definition]
    Definition\dots
  \end{definition}

  \pause
  \begin{proposition}[An example proposition]
    Proposition\dots
  \end{proposition}
\end{frame}

\section{Related Works}

\begin{frame}{Related Works}
\end{frame}

\section{Methodology}

\begin{frame}{Methodology}
\end{frame}

\section{Results}

\subsection{Validation and Analysis}

\begin{frame}{Validation}
\end{frame}

\begin{frame}{Case Study}
\end{frame}

\section{Conclusion and Future works}

\begin{frame}{Conclusion and Future works}
\end{frame}

\section{References}

\begin{frame}[allowframebreaks]{References}
  \nocite{bronevetsky02, schmidt03:MSc, FSF:GNU-GPL, CORBA:spec, MenaChalco08, natbib, biblatex, eco:09}
  \printbibliography
\end{frame}

% Recapitulando
\begin{frame}{\insertshorttitle}
  \overview

  % \begin{center} acrescenta espaço vertical;
  % como possivelmente temos bem pouco espaço aqui,
  % vamos usar centering
  {%
    \centering\noindent%
    \url{https://gitlab.com/link-of-your-repository}\par
  }

\end{frame}

\showqrcode

\appendix

\begin{frame}{Extra info}
  \begin{itemize}
    \item It is often useful to have some extra slides addressing likely questions from the audience at the end of the presentation
    \item By putting them after the ``appendix'' command, they are not counted in the page count indicator
  \end{itemize}
\end{frame}
